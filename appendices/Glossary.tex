\newglossaryentry{kollaborativ}{
    name={kollaborativ},
    description={ Als kollaborative Software oder kollaboratives System wird eine Software zur Unterstützung der computergestützten Zusammenarbeit in einer Gruppe über zeitliche und/oder räumliche Distanz hinweg bezeichnet. }
}

% %%% define the acronym and use the see= option
% \newglossaryentry{C2X}{type=\acronymtype, name={C2X}, description={Car-to-X oder Vehicle-to-X}, first={Car-to-X (C2X)\glsadd{C2Xg}}, see=[Glossary:]{C2Xg}}
% \newglossaryentry{C2Ig}{
%     name={C2I},
%     description={ direkter, drahtloser Datenaustausch zwischen Fahrzeugen jeglicher Art und infrastrukturellen Einrichtungen wie Funkbaken und Lichtsignalanlagen auf Basis von \gls{WLAN}, Bluetooth oder \gls{DSRC}}
% }

\newglossaryentry{Netzwerklatenz}{
    name={Netzwerklatenz},
    description={ Die Wartezeit, die im Netzwerk verbraucht wird bevor eine Kommunikation beginnen kann, wird als Netzwerklatenz oder nur Latenz bezeichnet}
}

\newacronym{LWW}{LWW}{Last-Write-Wins}
\newacronym{CRDT}{CRDT}{Conflict-free replicated data type}
\newacronym{OT}{OT}{Operational Transformation}
%
\newacronym{API}{API}{Application Programming Interface}
\newacronym{App}{App}{Applikation}
\newacronym{JSON}{JSON}{JavaScript Object Notation}
\newacronym{PDF}{PDF}{Portable Document Format}
\newacronym{WLAN}{WLAN}{Wireless Local Area Network}
\newacronym{VM}{VM}{Virtual Machine}
\newacronym{XML}{XML}{Extensible Markup Language}
\newacronym{SQL}{SQL}{Structured Query Language}
\newacronym{UI}{UI}{User Interface}
