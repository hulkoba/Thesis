\newglossaryentry{kollaborativ}{
  name={Kollaborativ},
  description={Als kollaborative Software oder kollaboratives System wird eine Software zur Unterstützung der computergestützten Zusammenarbeit in einer Gruppe über zeitliche und/oder räumliche Distanz hinweg bezeichnet}
}

\newglossaryentry{optimistic UI}{
  name={optimistische Bedienoberfl{\"a}che},
  description={auch: optimistic \gls{UI}, wartet nicht mit der Aktualisierung der Oberfläche auf das Ende einer Operation. Die zeigt also den gewünschten Zustand der \gls{App} an, befor die Anwendung fertig ist indem sie z.B. Fakedaten zeigt.}
}

\newglossaryentry{Queue}{
  name={Queue},
  description={auch Warteschlange, eine Datenstruktur die zur Zwischenspeicherung von Objekten dient. Hierbei wird das zuerst eingegebene Objekt auch zuerst verarbeitet (wie bei einer Warteschlange).}
}

\newglossaryentry{Middleware}{
  name={Middleware},
  description={Schicht zwischen Anwendung und Betriebssystem.}
}

\newglossaryentry{PWAg}{
  name={Progressive Web App},
  description={Oder ``fortschrittliche Web App`` eine mobil nutzbare Webseite, erstellt mit den Webstandards, die als Symbiose aus einer nativen, mobilen Anwendung und einer responsiven Webseite beschrieben werden kann. Die Idee dahinter ist, dass Apps zukünftig nicht mehr über einen App Store, sondern über den Browser installiert werden kann.}}
\newglossaryentry{PWA}{
  type=\acronymtype,
  name={PWA},
  description={Progressive Web App},
  first={Progressive Web App (PWA)\glsadd{PWAg}},
  see=[Glossary:]{PWAg}
}

\newglossaryentry{Latenz}{
  name={Latenz},
  description={ Die Wartezeit, die im Netzwerk verbraucht wird bevor eine Kommunikation beginnen kann, wird als Latenz oder als Netzwerklatenz bezeichnet.}
}
\newglossaryentry{Bandbreite}{
  name={Bandbreite},
  description={ gibt an, wie viele Daten pro festgelegter Zeitspanne über ein Netzwerk übertragen werden können.}
}

\newglossaryentry{Hashfunktion}{
  name={Hashfunktion},
  description={ TODO: ist eine Abbildung, die eine große Eingabemenge (die Schlüssel) auf eine kleinere Zielmenge (die Hashwerte) abbildet}
}

\newacronym{LWW}{LWW}{Last-Write-Wins}
\newacronym{CRDT}{CRDT}{Conflict-free replicated data type}
\newacronym{OT}{OT}{Operational Transformation}
\newacronym{UI}{UI}{User Interface}
\newacronym{JSON}{JSON}{JavaScript Object Notation}
\newacronym{App}{App}{Applikation}
\newacronym{API}{API}{Application Programming Interface}
\newacronym{PDF}{PDF}{Portable Document Format}
\newacronym{WLAN}{WLAN}{Wireless Local Area Network}
\newacronym{CAP}{CAP}{Consistency Availability Partition tolerance}
\newacronym{DBMS}{DBMS}{Datenbank Management System}
\newacronym{DB}{DB}{Datenbank}

%
% \newacronym[longplural={Lichtsignalanlagen}]{LSA}{LSA}{Lichtsignalanlage}
% /\newacronym{XML}{XML}{Extensible Markup Language}
% \newacronym{SQL}{SQL}{Structured Query Language}
