\chapter{\label{chap:fazit}Fazit}
\tt{noch aus Jans Blog:}\\
\it{\Gls{OT} ist so konzipiert, dass beliebig viele Personen gleichzeitig am selben Text arbeiten können und es eine gewisse Netzwerkinstabilität bewältigen kann. Im Allgemeinen müssen die Personen jedoch jederzeit mit dem Internet verbunden sein. Bearbeiten sie den Text ohne Internetverbindung, können Ihre Änderungen später, aber nicht unbegrenzt später, integriert werden. Darüber hinaus ist es für Text und nicht für generische Objekte konzipiert. Für echte Offline-Funktionen generischer Datenobjekte sind Operational Transforms daher weniger nützlich.}\\
\it{\Glspl{CRDT} sind spezialisierte Datenstrukturen, die für die Verwendung in verteilten Systemen entwickelt wurden. Sie verfügen über viele Eigenschaften, ..., haben aber kein Konfliktkonzept.
  Nun, sie sind spezialisierte Datenstrukturen wie \tt{sets} und \tt{counters} und keine generischen Objektrepräsentationen wie \gls{JSON}, also muss man sich in diese spezialisierten Datenstrukturen einarbeiten, und vielleicht gibt Schwierigkeiten, die verwendeten Anwendungsobjekte darauf abzubilden.}
