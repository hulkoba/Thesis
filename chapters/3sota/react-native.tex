\section{React Native Offline}
React Native ist ein Framework mit dem native, mobile Apps mit JavaScript und React gebaut werden können~\cite{rn}.\\
Die Bibliothek React Native Offline erweitert das Framework um Offlinefunktionalität.
React Native Offline unterstützt die Behandlung des Netzwerkstatus.
Dieser kann einmalig oder regelmäßig abgefragt werden, um je nach Status z.B. einen anderen Inhalt zu rendern.\\
Zusammen mit Redux implementiert, können weitere Fähigkeiten genutzt werden.
Genau wie Redux Offline hat React Native Offline eine Offline \gls{Queue}, in der Aktionen gespeichert werden können.
Allerdings nur solche Aktionen, die fehlgeschlagen sind, weil die Anwendung nicht mit dem Internet verbunden ist.
Auch hier wird der Aktion ein Metaattribut gegeben. Dieses hat die Felder \tt{retry} und \tt{dismiss}.
Das erste Feld erwartet ein Boolean. Hier kann angegeben werden ob die Aktion noch einmal bei bestehender Internetverbindung ausgeführt werden soll oder nicht.
Das Feld \tt{dismiss} erwartet ein Array. Hier können Aktionen angegeben werden, die das Wiederholen der Aktion abbrechen~\cite{rn-offline-gh}.\\\\
Ein Beispiel im folgenden Listing soll die Funktionsweise der Metaattribute besser beschreiben.
%
\lstset{language=REACT,
  caption={Beispiel einer React Native Offline Aktion},
  label={code:rno}}
\begin{lstlisting}
const action = {
  type: 'FETCH_CONTACT',
  contact,
  meta: {
    retry: true,
    dismiss: ['CANCEL']
  }
}
\end{lstlisting}
%
Bei Aufruf dieser Aktion soll ein Kontakt geladen werden.
Im Metafeld ist \tt{retry} auf \tt{true} gesetzt.
Wurde die Aktion im Offlinemodus versucht auszuführen, wird sie erneut aufgerufen sobald die Anwendung wieder einen Internetzugang hat.
Die Aktionen die in \tt{dismiss} angegeben werden, unterbrechen diesen Vorgang.
Wurde also der Kontakt im Offlinemodus angefragt und dann die Aktion ''CANCEL'' gefeuert, wird ''FETCH\_CONTACT'' aus der Queue gelöscht und nicht erneut ausgeführt.
