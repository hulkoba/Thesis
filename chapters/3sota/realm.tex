\section{\label{sub:realm}Realm}
Realm ist eine Backendtechnologie für mobile Anwendungen und umfasst die Realm Datenbank und den Realm Object Server. Beide Technologien sind quelloffen, jedoch nicht kostenfrei~\cite{realm}.\\
Die Realm Datenbank ist eine objektorientierte, plattformübergreifende lokale Datenbank die eine Echtzeitsynchronisation mit dem Realm Object Server.
Der Object Server fungiert als \gls{Middleware}-Komponente in der mobilen \gls{App} und handhabt unter anderem die Ereignisbehandlung und Datensynchronisation. Im Zusammenspiel ermöglichen die beiden Technologien die Erstellung von offlinefähigen, kollaborativen, mobilen Anwendungen~\cite{realm_whitepaper}.\\\\
Zur Offline First Funktionalität stellt Realm eine umfassende Lösung bereit.
Die lokale Realm Datenbank unterstützt die Echtzeitsynchronisation von Daten sodass alle Änderungen sofort automatisch gesendet werden. Das Synchronisationsprotokoll komprimiert statt dem gesamten Objekt nur die marginalen Änderungen und synchronisiert sie auf dem Endgerät und dem Server. Zusätzlich zu den Daten werden die spezifischen Operationen erfasst. \todo{Siehe OT}. 
Wird beispielsweise ein Kontakt bearbeitet, wird neben den geänderten Daten die Information \it{update} mitgesendet.
Dank dieser zusätzlichen Information kann der Aktionswunsch genau erfasst werden sodass das System eventuelle Konflikte automatisch auflösen kann. Das hat zur Folge, dass die Synchronisation keinen manuellen Eingriff bedarf, der die Leistung des Systems beeinträchtigen könnte.
Zusätzlich zu dem \gls{OT} Algorithmus benutzt Realm vorgegebene Regeln zur automatischen Konfliktlösung. Es besteht die Möglichkeit eigene Regeln zu definieren.\\
Darüber hinaus passiert die interne Konfliktlösung auf Transaktionsebene. Das heißt der Vorgang ist nur erfolgreich wenn er auch vollständig und fehlerfrei ist. Andererseits wird er zurückgesetzt. Das gewährleistet die Konsistenz der Daten und verhindert deren Verlust wenn Änderungen aufgrund einer unterbrochenen Netzwerkverbindung nicht stattfinden können~\cite{realm_offline_whitepaper}.