\section{\label{sub:realm}Realm}
Realm ist eine Backendtechnologie für mobile Anwendungen und umfasst die Realm Datenbank und den Realm Object Server. Die Datenbank ist quelloffen, der Object Server jedoch nicht -- dieser ist außerdem nicht kostenfrei~\cite{realm}.\\
Die Realm Datenbank ist eine objektorientierte, plattformübergreifende lokale Datenbank die eine Echtzeitsynchronisation mit dem Realm Object Server bereitstellt.
Der Object Server fungiert als \gls{Middleware}-Komponente in der mobilen \gls{App} und handhabt unter anderem die Ereignisbehandlung und Datensynchronisation. Im Zusammenspiel ermöglichen die beiden Technologien die Erstellung von offlinefähigen, kollaborativen, mobilen Anwendungen~\cite{realm_whitepaper}.\\\\
Zur Offline First Funktionalität stellt Realm eine umfassende Lösung bereit.
Die lokale Realm Datenbank unterstützt die Echtzeitsynchronisation von Daten sodass alle Änderungen sofort automatisch gesendet werden. Das Synchronisationsprotokoll komprimiert statt dem gesamten Objekt nur die marginalen Änderungen und synchronisiert sie auf dem Endgerät und dem Server. Zusätzlich zu den Daten werden die spezifischen Operationen erfasst. 
Wird beispielsweise ein Kontakt bearbeitet, wird neben den geänderten Daten die Information \it{update} mitgesendet.
Dank dieser zusätzlichen Information kann der Aktionswunsch genau erfasst werden sodass das System eventuelle Konflikte automatisch auflösen kann.
Das hat zur Folge, dass die Synchronisation keinen manuellen Eingriff bedarf ~\cite{realm_offline_whitepaper}.\\
Zusätzlich zu dem \gls{OT} Algorithmus benutzt Realm vorgegebene Regeln zur automatischen Konfliktlösung. Es gibt drei Grundregeln die die hauptsächlichen Aktivitäten abdecken. Neue Einträge in Listen werden zeitlich sortiert. Für den Fall dass zwei Objekte gleichzeitig zur selben Liste hinzugefügt werden, wird das mit dem neueren Zeitstempel hinter dem älteren Objekt gespeichert.
Löschungen haben immer Vorrang; auch dann wenn das auf dem einen Gerät gelöschte Objekt auf einem anderen zu einem späteren Zeitpunkt bearbeitet wurde. Für die Aktualisierungen wird \gls{LWW} angewandt. Wird ein Objekt auf zwei Geräten bearbeitet, wird das mit dem neueren Zeitstempel behalten.\\ 
Es besteht auch die Möglichkeit eigene Regeln zu definieren oder die bestehenden zu überschreiben ~\cite{realm_conflict}.
% Eine Regel könnte zum Beispiel sein, dass alle Änderungen die von der Nutzerin, nennen wir sie Amilia Pond, gemacht werden den Vorrang haben. Das ist keine gute Regel, denn dabei würden definitiv Daten verloren gehen wenn eine andere Person eine andere Änderung an derselben Stelle wie Amilia Pond macht, aber sie soll ja nur als Beispiel dienen.\\
Darüber hinaus passiert die interne Konfliktlösung auf Transaktionsebene. Das heißt der Vorgang ist nur erfolgreich wenn er auch vollständig und fehlerfrei ist. Andernfalls wird er zurückgesetzt.
Das gewährleistet die Konsistenz der Daten und verhindert deren Verlust wenn Änderungen aufgrund einer unterbrochenen Netzwerkverbindung nicht stattfinden können~\cite{realm_offline_whitepaper}.