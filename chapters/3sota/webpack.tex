\section{Offline plugin für webpack}
Webpack ist ein JavaScript `Bundler` und bündelt alle Skripte, Bilder und \gls{Assets} für die Verwendung in Browsern.\\
Das Offline Plugin bietet Offlinefunktionalität für Webpackprojekte, indem es die gebündelten, also von Webpack generierten, \gls{Assets} cached.
Dazu benutzt es intern den ServiceWorker und AppCache als Fallback, für den Fall dass der Browser ServiceWorker nicht unterstützt~\cite{webpack-gh}.\\
Auch ungebündelte \gls{Assets} können über das Plugin gecached werden. Diese Dateien müssen dann in den Optionen explizit angegeben werden. Auch der ServiceWorker und der AppCache lassen sich über die Optionen konfigurieren oder auch ausschalten~\cite{webpack-opt}.\\
Es werden allerdings nur die \gls{Assets} und nicht die von BenutzerInnen generierten Daten gecached. Diese müssen manuell gespeichert werden.