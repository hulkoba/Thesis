\sub{offline-plugin für webpack}
webpack ist ein JavaScript `Bundler`, packt JavaScript-Dateien und oder \gls{Assets} für die Verwendung in Browsern.blabla\\
Bietet offline experience für webpack Projekte. Benutzt \sc{ServiceWorker} und \sc{AppCache} unter der Haube --> cached nur (gebündelten) von webpack generierten \gls{Assets}. Für die anderen Dateien (z.B. index.html die nicht gebundled wird oder Modul von CDN) braucht mal ein \tt{html-plugin} oder benutzt die \tt{externals} :\\\\
\tt{const OfflinePlugin = require('offline-plugin')\\
  const offline = new OfflinePlugin\\\\
  const offline = new OfflinePlugin({\\
    externals: ['index.html'],\\
    })}\\
Es gibt diverse config-options für \sc{ServiceWorker} und AppCache...~\cite{webpack-gh}
% dev\cite{webpack-dev}
