\chapter{\label{chap:anforderungen}Anforderungsdefinition}
Dieses Kapitel beschreibt die Anforderungen an eine Offline First Anwendung unter Berücksichtigung von Funktionalität, Konfliktmanagement, Tests und der Bedienoberfläche.
Nach der konkreten Beschreibung dessen, was das \it{Projekt} umfassen soll, folgen die aus den oben genannten \hyperref[chap:szenarien]{Szenarien} hergeleiteten Anforderungen, die eine offlinefähige Anwendung erfüllen soll.
%
% Scope
%
Das Ziel dieser Arbeit ist die Untersuchung des Konfliktmanagements offlinefähiger Technologien.
Dazu soll die beispielhafte Anwendung entwickelt werden, welche an dem Beispiel eines kollaborativen Adressbuchs die Offlinekompatibilität mit dem Schwerpunkt auf das Konfliktmanagement der verwendeten Technologien illustriert.
Ein offlinefähiges, kollaboratives Adressbuch ist eine Anwendung, mit der mehrere Personen Kontakte verwalten können.
Dass diese Anwendung auch ohne Internetzugang funktioniert, ist obligatorisch.
Das Adressbuch beinhaltet eine Liste von Kontakteinträgen, welche jederzeit -- unabhängig von der Internetverbindung -- von den verwendenden Personen gelesen, bearbeitet, erstellt und gelöscht werden können.
Die Offlinefunktionalität soll durch die verwendeten Technologien gegeben sein und wird nicht ausgiebig getestet.
Geschieht eine dieser Operationen offline, werden die Daten bei wieder bestehender Internetverbindung synchronisiert. Im einfachen Fall erfolgt die Synchronisation zwischen dem Server und Client.
Da die Beispielanwendung kollaborativ ist, erfolgt sie zwischen allen beteiligten Parteien.
% Synchronisation erfordert in jedem Fall den Umgang mit Konflikten.
Der Schwerpunkt liegt auf den Konflikten, die entstehen können wenn keine Internetverbindung besteht.
Dabei sind Aspekte aus unterschiedlichen Rollen zu betrachten. So werden die Rollen der AnwenderInnen, der EntwicklerInnen und die der TesterInnen berücksichtigt.\\\\
%
%
Beim ersten Start der Anwendung müssen, sofern vorhanden, alle Kontakte geladen werden. Sobald sie einmal geladen sind, sollen sie auch offline verfügbar sein.
Damit ein Datensatz, wie zum Beispiel ein Adressbucheintrag, offline erreichbar ist, sollte er wenigstens so lange auf dem Client gespeichert werden, bis er vollständig beim Server angekommen ist.
Im aktuellen Anwendungsfall bedeutet das, es gibt zwei Kopien des Adressbucheintrags. Eine auf dem Anwendungsgerät, eine auf dem Server.\\
Danach sollen nur die Einträge geladen werden, die nicht auf dem Gerät existieren.
Andernfalls werden unnötige Mengen von daten über das Netzwerk gesendet. Das soll vermieden werden.
% Die Daten würden sonst doppelt geladen werden und der Server hätte mehr zu arbeiten, was wiederum die Antwortzeit verlängern würde.
Dazu muss ermittelt werden können, welche Daten neu angelegt oder aktualisiert wurden.
Der Server muss also in der Lage sein die Einträge zu sortieren und nur bestimmte Einträge zu versenden und die Anwendung muss wissen, welche Daten sie bereits hat. Wird ein Eintrag angelegt, bearbeitet oder gelöscht, müssen alle Parteien wissen, um welchen Kontakt es sich handelt.
Dazu muss jeder Kontakt mittels einer eindeutigen ID identifizierbar sein.
Wenn es mehrere Kontakteinträge mit derselben ID gibt, muss feststellbar sein, welcher Eintrag der aktuellere ist.\\
Gibt es mehr als zwei Einträge, müssen diese sortiert werden, sodass ersichtlich wird welcher der aktuellste oder älteste ist, welcher Eintrag vor oder nach welchem kommt. Dazu muss jeder Kontakt versioniert werden.\\\\
Um eine Aussage darüber zu treffen, welche der untersuchten Technologien besser für die Entwicklung einer offlinefähigen Anwendung geeignet ist und warum, ist eine Testumgebung mitzuentwickeln. Hier soll die Möglichkeit geschaffen werden, Konflikte durch gleichzeitiges Bearbeiten eines Kontakts im Offlinestatus herbeizuführen und die Testfälle auswerten zu können.\\
Es wird ermittelt welche Strategie zur Konfliktlösung von der Technologie verwendet wird, ob die Funktionalität der Anwendung auch bei auftretenden Konflikten gewährleistet ist und ob dabei Daten verloren gehen.
Dabei wird auch der Implementierungsaufwand betrachtet, der aufgebracht werden muss, um die Technologie zu verwenden und wie leicht der geschriebene Quellcode zu verstehen ist.\\\\
%
% was Dein Projekt umfassen soll und was nicht. Im Prinzip eine konkretere Version Deiner Zielstellung.
% Hier legst Du auch fest, was Du untersuchen willst und was nicht.
%
Um den Rahmen dieser Arbeit nicht zu sprengen, werden die Testdurchläufe auf nur zwei Geräten durchgeführt.
Dort werden durch das Bearbeiten eines Kontakts auf beiden Geräten einfache Konflikte erzeugt.
Auf das Zusammenführen von zwei konfliktbehafteten Versionen wird verzichtet.
Gibt es zwei unterschiedliche Versionen eines Kontakts auf beiden Geräten, wird eine davon behalten.\\
Die verwendeten Technologien werden nur an die Anwendung angepasst wenn es unbedingt notwendig ist, wenn die Grundfunktionalität der Anwendung sonst eingeschränkt wäre. Sie werden nur benutzt, da nur so ein unverfälschtes Ergebnis erzielt werden kann.
Es wird ebenso auf ein komplexes \gls{UI} und ein perfektes Nutzungserlebnis verzichtet.\\\\
%
%
Die in \autoref{chap:konfliktszenarien} erarbeiteten Szenarien zeigen, dass Konflikte immer auftreten können.
Werden Konflikte falsch oder gar nicht behandelt, kann das zu Datenverlust führen.
Aus diesem Grund müssen sie als Teil der Anwendung betrachtet, statt ignoriert zu werden.
Im einfachen Konfliktfall kann das System entscheiden welches die konfliktfreie Version ist.
So kann zum Beispiel der Kontakt `Amilia Pond` von einer Person eine neue Telefonnummer, von einer anderen eine neue Adresse bekommen.
Die Aktualisierungen finden in unterschiedlichen Bereichen statt und sind deshalb unproblematisch zuzuordnen.\\\\
Die oben erarbeiteten \hyperref[chap:konfliktszenarien]{Konfliktszenarien} beschreiben die Konflikte die nicht vom System gelöst werden können.
Diese sollen effizient gespeichert werden.
% \it{Wichtig hierbei ist die Möglichkeit immer zu einem konfliktfreien Status zu gelangen -- unabhängig davon wie viele Konflikte es gibt.}\\
Jeder Fehlerfall muss kommuniziert werden. Wenn es konliktbehaftete Daten gibt muss dies mitgeteilt und angeboten werden die Konflikte zu lösen.
% Nur so kann sichergestellt werden, dass keine Daten verloren gehen.
%
% User-Stories
%
\section{User-Stories}
Aus den in Kapitel \ref{chap:szenarien} erarbeiteten Szenarien ergeben sich die folgenden User-Stories, die von der offlinefähigen Adressbuchanwendung erfüllt werden sollen.
Ein Ziel der Arbeit ist es, EntwicklerInnen bei der Wahl einer Technologie mit der eine offlinefähige Anwendung entwickelt werden kann, zu unterstützen. Dafür wird untersucht, inwieweit die ausgewählten Technologien die Erwartungen an diese erfüllen.
Deswegen werden zunächst die Anforderungen aus Sicht der NutzerInnen definiert, danach die aus Entwicklingsperspektive und dann die aus der Perspektive der TesterInnen.
% Tabellen: User Stories
Die folgende \hyperref[tab:user]{Tabelle} zeigt die Software-Anforderungen an eine offlinefähige Kontaktliste aus der NutzerInnenperspektive.
\begin{longtable}[c]{@{}
>{\columncolor[HTML]{CFFCC2}}l ll@{}}
\toprule
    \multicolumn{1}{p{0.05\textwidth}}{\cellcolor[HTML]{cffcc2}\textbf{ID}}
    & \multicolumn{1}{p{0.9\textwidth}}{\cellcolor[HTML]{cffcc2}\textbf{Anforderung aus NutzerInnenperspektive}}\\ \hline \noalign{\vskip 0.1cm}
\endfirsthead
\endhead
%
% 
  \multicolumn{1}{p{0.05\textwidth}}{\cellcolor[HTML]{cffcc2}\textbf{U1}} &
  \multicolumn{1}{p{0.9\textwidth}}
  {Ich als NutzerIn möchte über die Anwendung immer und überall, auch ohne Internetzugang verfügen.}\\
  \midrule
  %
  \multicolumn{1}{p{0.05\textwidth}}{\cellcolor[HTML]{cffcc2}\textbf{U2}} &
  \multicolumn{1}{p{0.9\textwidth}}
  {Ich als NutzerIn möchte, dass die Kontaktliste schnell und effizient geladen wird, um Zeit zu sparen.}\\
  \midrule
  %
  % \multicolumn{1}{p{0.05\textwidth}}{\cellcolor[HTML]{cffcc2}\textbf{U3}} &
  % \multicolumn{1}{p{0.9\textwidth}}
  % {Ich als NutzerIn möchte jeden Kontakteintrag finden, um zu wissen ob ich ihn schon gespeichert habe.}\\
  % \midrule
  %
  \multicolumn{1}{p{0.05\textwidth}}{\cellcolor[HTML]{cffcc2}\textbf{U3}} &
  \multicolumn{1}{p{0.9\textwidth}}
  {Ich als NutzerIn möchte Einträge immer und überall, auch ohne Internetzugang, erstellen, editieren oder löschen können, um meine Liste zu verwalten.}\\
  \midrule
  %
  \multicolumn{1}{p{0.05\textwidth}}{\cellcolor[HTML]{cffcc2}\textbf{U4}} &
  \multicolumn{1}{p{0.9\textwidth}}
  {Ich als NutzerIn möchte keine in der Anwendung gespeicherten Daten verlieren.}\\
  % end
  \bottomrule \cellcolor[HTML]{FFFFFF}
  \vspace{0.1cm}\\
  \noalign{\hspace{0.0525\textwidth}\grayRule}
  \caption{Anforderungen aus NutzerInnenperspektive}
  \label{tab:user}\\
\end{longtable}
Die folgende Tabelle zeigt die Software-Anforderungen an eine offlinefähige Kontaktliste aus der Perspektive der EntwicklerInnen.
\begin{longtable}[c]{@{}
	>{\columncolor[HTML]{CFFCC2}}l ll@{}}
	\toprule
	\multicolumn{1}{p{0.05\textwidth}}{\cellcolor[HTML]{cffcc2}\textbf{ID}}
	&
	\multicolumn{1}{p{0.9\textwidth}}{\cellcolor[HTML]{cffcc2}\textbf{Anforderung aus Entwicklungsperspektive}} \\
	\hline \noalign{\vskip 0.1cm}
	\endfirsthead
	%
	\endhead
	%
	\multicolumn{1}{p{0.05\textwidth}}{\cellcolor[HTML]{cffcc2}\textbf{D1}} & 
	\multicolumn{1}{p{0.9\textwidth}}
	{Ich als EntwicklerIn möchte die Daten lokal und auf dem Server speichern, um deren Erreichbarkeit unabhängig vom Internetstatus zu gewährleisten.}\\
	\midrule
	%
	\multicolumn{1}{p{0.05\textwidth}}{\cellcolor[HTML]{cffcc2}\textbf{D2}} & 
	\multicolumn{1}{p{0.9\textwidth}}
	{Ich als EntwicklerIn möchte nur die Adressbucheinträge oder deren Aktualisierungen laden, die sich nicht schon auf dem Endgerät befinden, um Datentraffic und Ladezeiten zu sparen.}\\
	\midrule
	%
	\multicolumn{1}{p{0.05\textwidth}}{\cellcolor[HTML]{cffcc2}\textbf{D3}} &
	\multicolumn{1}{p{0.9\textwidth}}
	{Ich als EntwicklerIn möchte jeden Eintrag identifizieren, um jedem Adressbucheintrag Operationen zuzuweisen und einzelne Kontakte zu finden.}\\
	\midrule
	%
	\multicolumn{1}{p{0.05\textwidth}}{\cellcolor[HTML]{cffcc2}\textbf{D4}} &
	\multicolumn{1}{p{0.9\textwidth}}
	{Ich als EntwicklerIn möchte, dass jeden Eintrag versioniert ist, um zu wissen ob wann ein Eintrag bearbeitet wurde.}\\
	\midrule
	%
	\multicolumn{1}{p{0.05\textwidth}}{\cellcolor[HTML]{cffcc2}\textbf{D5}} & 
	\multicolumn{1}{p{0.9\textwidth}}
	{Ich als EntwicklerIn möchte, dass alle von NutzerInnen vorgenommenen Änderungen beim System ankommen und keine Daten verloren gehen.}\\
	\midrule
	%
	\multicolumn{1}{p{0.05\textwidth}}{\cellcolor[HTML]{cffcc2}\textbf{D6}} &
	\multicolumn{1}{p{0.9\textwidth}}
	{Ich als EntwicklerIn möchte auftretende Konflikte effizient speichern, um mit ihnen umgehen können.
	% \todo{Mit ihnen umgehen heißt: selbstständig oder von User lösen, zum konfliktfreien Zustand gelangen}
	}\\
	\midrule
	%
	\multicolumn{1}{p{0.05\textwidth}}{\cellcolor[HTML]{cffcc2}\textbf{D7}} &
	\multicolumn{1}{p{0.9\textwidth}}
	{Ich als EntwicklerIn möchte eine Technologie verwenden die leicht zu verstehen und implemenieren ist, um den Arbeitsaufwand gering zu halten.}\\
	\midrule
	%
	\multicolumn{1}{p{0.05\textwidth}}{\cellcolor[HTML]{cffcc2}\textbf{D8}} &
	\multicolumn{1}{p{0.9\textwidth}}
	{Ich als EntwicklerIn möchte sauberen und verständlichen Code schreiben, um die Les-- und Wartbarkeit zu erhöhen.}\\
	% end
	\bottomrule \cellcolor[HTML]{FFFFFF}
	\vspace{0.1cm}\\
	\noalign{\hspace{0.0525\textwidth}\grayRule}
	\caption{Anforderungen aus Entwicklungsperspektive}
	\label{tab:dev}\\
\end{longtable}
\sub{TesterInnen Perspektive}
Die folgende Tabelle zeigt die Software-Anforderungen an eine offlinefähige Kontaktliste aus Perspektive der TesterInnen.
\begin{longtable}[c]{@{}
	>{\columncolor[HTML]{CFFCC2}}l ll@{}}
	\toprule
	\multicolumn{1}{p{0.05\textwidth}}{\cellcolor[HTML]{cffcc2}\textbf{ID}}
	                                                                   & \multicolumn{1}{p{0.95\textwidth}}{\cellcolor[HTML]{cffcc2}\textbf{Anforderung aus TesterInnenperspektive}} \\ \hline \noalign{\vskip 0.1cm}
	\endfirsthead
	%
	\endhead
	%
	\multicolumn{1}{l}{\cellcolor[HTML]{cffcc2}\textbf{T1}} &
	\multicolumn{1}{p{0.95\textwidth}}
	{Ich als TesterIn möchte sicherstellen, dass der Netzwerkstatus der Anwendung änderbar ist, um zwischen offline und online zu wechseln.}\\
  \midrule
	%
	\multicolumn{1}{l}{\cellcolor[HTML]{cffcc2}\textbf{T2}} &
	\multicolumn{1}{p{0.95\textwidth}}
	{Ich als TesterIn möchte wissen ob die Anwendung mit dem Internet verbunden ist oder nicht.}\\
	\midrule
	%
	\multicolumn{1}{l}{\cellcolor[HTML]{cffcc2}\textbf{T3}} &
	\multicolumn{1}{p{0.95\textwidth}}
	{Ich als TesterIn möchte die Anwendung auf mindestens zwei Geräten verwenden, um Kontakte gleichzeitig zu bearbeiten zu können.}\\
	\midrule
	%
	\multicolumn{1}{l}{\cellcolor[HTML]{cffcc2}\textbf{T4}} &
	\multicolumn{1}{p{0.95\textwidth}}
	{Ich als TesterIn möchte Konflikte forcieren, um das Verhalten der Anwendung zu evaluieren.}\\
	\midrule
	%
	\multicolumn{1}{l}{\cellcolor[HTML]{cffcc2}\textbf{T5}} &
	\multicolumn{1}{p{0.95\textwidth}}
	{Ich als TesterIn möchte einen Eintrag editieren \it{können} wenn 1. beide Client und Server online sind, 2. entweder Client oder Server offline ist oder 3. beide Parteien offine sind. \todo{daraus 3 Stories machen? oder nur testen online--offline?}}\\
	\midrule
	%
	\multicolumn{1}{l}{\cellcolor[HTML]{cffcc2}\textbf{T6}} &
	\multicolumn{1}{p{0.95\textwidth}}
	{Ich als TesterIn möchte die Testfälle detailliert dokumentieren \it{können}, um sie auswerten zu können.}\\
	% end
	\bottomrule \cellcolor[HTML]{FFFFFF}
	\vspace{0.1cm}\\
	\noalign{\hspace{0.0525\textwidth}\grayRule}
	\caption{Anforderungen aus TesterInnenperspektive}
	\label{tab:test}\\
\end{longtable}

%
% Funktionalität
%
%
\begin{longtable}[c]{@{}
>{\columncolor[HTML]{CFFCC2}}l ll@{}}
\toprule
    \multicolumn{1}{p{0.05\textwidth}}{\cellcolor[HTML]{cffcc2}\textbf{ID}}
    & \multicolumn{1}{p{0.95\textwidth}}{\cellcolor[HTML]{cffcc2}\textbf{Funktionale Anforderungen}}\\ \hline \noalign{\vskip 0.1cm}
\endfirsthead
\endhead
%
% 
  \multicolumn{1}{l}{\cellcolor[HTML]{cffcc2}\textbf{F1}} &
  \multicolumn{1}{p{0.95\textwidth}}
  {Ich als NutzerIn möchte die Anwendung immer und überall, auch ohne Internetzugang zu nutzen.}\\
  \midrule
  %
  \multicolumn{1}{l}{\cellcolor[HTML]{cffcc2}\textbf{F2}} &
  \multicolumn{1}{p{0.95\textwidth}}
  {Ich als Nutzerin möchte, dass die Kontaktliste schnell und effizient geladen wird, um Zeit zu sparen.}\\
  \midrule
  %
  \multicolumn{1}{l}{\cellcolor[HTML]{cffcc2}\textbf{F3}} &
  \multicolumn{1}{p{0.95\textwidth}}
  {Ich als NutzerIn möchte jeden Kontakteintrag finden, um zu wissen ob ich ihn schon gespeichert habe.}\\
  \midrule
  %
  \multicolumn{1}{l}{\cellcolor[HTML]{cffcc2}\textbf{F4}} &
  \multicolumn{1}{p{0.95\textwidth}}
  {Ich als NutzerIn möchte Einträge immer und überall, auch ohne Internetzugang, erstellen, editieren oder löschen können, um meine Liste zu verwalten.}\\
  \midrule
  %
  \multicolumn{1}{l}{\cellcolor[HTML]{cffcc2}\textbf{F5}} &
  \multicolumn{1}{p{0.95\textwidth}}
  {Ich als NutzerIn möchte keine in der Anwendung gespeicherten Daten verlieren.}\\
  % end
  \bottomrule \cellcolor[HTML]{FFFFFF}
  \vspace{0.1cm}\\
  \noalign{\hspace{0.0525\textwidth}\grayRule}
  \caption{Funktionale Anforderungen}
  \label{tab:funcreq}\\
\end{longtable}


%
%
\sub{Konfliktmanagement}

    % \subitem Operationen müssen dem Objekt/Eintrag zugeordnet werden
  % \item Delta berechnen [alle Kontakte -- lokal existierende Kontakte]
%
% UI
%
\section{Bedienoberfläche}
Da der Schwerpunkt dieser Arbeit auf dem Umgang mit Konflikten der zu testenden offlineunterstützenden Technologien liegt, soll die Bedienoberfläche der Anwendung möglichst einfach gehalten werden.\\
Alle Adressbucheinträge sollen in einer Liste angezeigt werden. Zum Anlegen, Editieren und Löschen eines einzelnen Eintrags soll es eine zweite Ansicht geben, auf die man per Klick auf den entsprechenden Eintrag in der Liste gelangt.
Wenn es zum Konflikt kommt, kann dieser über ein Dialogfenster aufgelöst werden. Im Dialog muss erkennbar sein wo, bei welchem Kontakteintrag, der Konflikt auftrat.
Außerdem müssen sich die entsprechenden Bereiche beider Versionen unterscheiden lassen und auswählbar sein. Abbildung \ref{fig:dialog} zeigt wie so ein Dialogfenster aussehen könnte.
\begin{figure}[H]
	\centering
	\includegraphics[width=0.8\textwidth]{Konfliktdialog}
	\grayRule
	\caption{Dialogfenster im Konfliktfall}
	\label{fig:dialog}
\end{figure}
Wurde beispielsweise Amilias Telefonnummer von zwei Personen gleichzeitig bearbeitet, bildet der Dialog zwei Bereiche mit der Nummer in den unterschiedlichen Versionen ab.
Durch Klick auf die korrekte Nummer kann entschieden werden welche Version die richtige ist und behalten wird.