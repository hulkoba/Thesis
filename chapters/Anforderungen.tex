\chapter{\label{chap:anforderungen}Anforderungsdefinition}
Dieses Kapitel beschreibt die Anforderungen an eine Offline First Anwendung unter Berücksichtigung von Konfliktmanegement und Funktionalität.
Die von BenutzerInnen generierten Daten sollten wenigstens so lange auf dem Client gespeichert werden, bis sie vollständig beim Server angekommen sind. Jeder Fehlersfall muss kommuniziert werden. Wenn es konliktbehaftete Daten gibt muss dies mitgeteilt, und angeboten werden die Konflikte zu lösen (Welche Telefonnummer ist die richtige).\\
Aus den oben genannten \hyperref[chap:szenarien]{Szenarien} werden im Folgenden die Anforderungen hergeleitet, die eine offlinefähige Anwendung \highlight{unter Berücksichtigung von...} erfüllen soll.
%
% Use Cases  \hyperref[sec:conflict]{oben}
%
\section{Anwendungsfälle}
Aus den in Kapitel \ref{chap:szenarien} erarbeiteten Szenarien ergeben sich die folgenden \b{drei} Use-Cases, die von der Anwendung erfüllt werden sollen.\\
\begin{table}[H]
\centering
  \begin{tabular}{@{}>{\columncolor[HTML]{cffcc2}}l ll@{} p{0.1\textwidth}p{0.4\textwidth}p{0.4\textwidth}} \toprule
\multicolumn{1}{c}{\cellcolor[HTML]{cffcc2}\textbf{ID}}
& \multicolumn{1}{c}{\cellcolor[HTML]{cffcc2}\textbf{Anwendungsfall}}
& \multicolumn{1}{c}{\cellcolor[HTML]{cffcc2}\textbf{Beschreibung}} \\
\hline
% UC1
\multicolumn{1}{l}{\cellcolor[HTML]{cffcc2}\textbf{UC1}} & \multicolumn{1}{p{0.35\textwidth}}{Ich.., um ...}
& \multicolumn{1}{p{0.55\textwidth}}{Es passiert das und das.} \\
\midrule
% UC2
\multicolumn{1}{l}{\cellcolor[HTML]{cffcc2}\textbf{UC2}} & \multicolumn{1}{p{0.35\textwidth}}{Ich.., um ...}
& \multicolumn{1}{p{0.55\textwidth}}{Es passiert das und das.} \\
\midrule
% UC3
\multicolumn{1}{l}{\cellcolor[HTML]{cffcc2}\textbf{UC3}} & \multicolumn{1}{p{0.35\textwidth}}{Ich ..., um d...}
& \multicolumn{1}{p{0.55\textwidth}}{Es passiert das und das.}\\
%\midrule
\bottomrule \cellcolor[HTML]{FFFFFF} \vspace{0.1cm}
\end{tabular}
\grayRule
  \caption{Anwendungsfälle}
  \label{tab:usecase}
\end{table}

Dann Use-Case-Diagramm
%
% Funktionalität
%
\section{Funktionalität}
\todo{siehe Anforderungen PWA?} Plus kein Datenverlust, und 'just work'\\
Daten sollen, sobald einmal geladen, auch offline verfügbar sein.
Daten sollen jederzeit (offline und online) lesbar und bearbeitbar (löschbar) sein.
%
% UI
%
\section{Die graphische Oberfläche}
\Gls{optimistic UI}
\gls{UI} soll mich nicht mit Meldungen darüber nerven, dass ich offline bin. (Bsp. Chat)\\
\gls{UI} soll sagen wenn es einen Konflikt gab / gibt und mich entscheiden lassen. Bzw ihn lösen lassen.
Auf keinen Fall selber lösen und mich nichts davon wissen lassen.\\
Im besten Fall soll due UI mir sagen \b{warum} es zum Konflikt gekommen ist.