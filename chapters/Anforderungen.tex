\chapter{\label{chap:anforderungen}Anforderungsdefinition}
Dieses Kapitel beschreibt die Anforderungen an eine Offline First Anwendung unter Berücksichtigung von Konfliktmanegement und Funktionalität.  Aus den oben genannten \hyperref[chap:szenarien]{Szenarien} werden im Folgenden die Anforderungen hergeleitet, die eine offlinefähige Anwendung \highlight{unter Berücksichtigung von...} erfüllen soll.
%
% Use Cases  \hyperref[sec:conflict]{oben}
%
\section{Anwendungsfälle}
Aus den in Kapitel \ref{chap:szenarien} erarbeiteten Szenarien ergeben sich die folgenden \b{drei} Use-Cases, die von der Anwendung erfüllt werden sollen.\\
\begin{table}[H]
\centering
  \begin{tabular}{@{}>{\columncolor[HTML]{cffcc2}}l ll@{} p{0.1\textwidth}p{0.4\textwidth}p{0.4\textwidth}} \toprule
\multicolumn{1}{c}{\cellcolor[HTML]{cffcc2}\textbf{ID}}
& \multicolumn{1}{c}{\cellcolor[HTML]{cffcc2}\textbf{Anwendungsfall}}
& \multicolumn{1}{c}{\cellcolor[HTML]{cffcc2}\textbf{Beschreibung}} \\
\hline
% UC1
\multicolumn{1}{l}{\cellcolor[HTML]{cffcc2}\textbf{UC1}} & \multicolumn{1}{p{0.35\textwidth}}{Ich.., um ...}
& \multicolumn{1}{p{0.55\textwidth}}{Es passiert das und das.} \\
\midrule
% UC2
\multicolumn{1}{l}{\cellcolor[HTML]{cffcc2}\textbf{UC2}} & \multicolumn{1}{p{0.35\textwidth}}{Ich.., um ...}
& \multicolumn{1}{p{0.55\textwidth}}{Es passiert das und das.} \\
\midrule
% UC3
\multicolumn{1}{l}{\cellcolor[HTML]{cffcc2}\textbf{UC3}} & \multicolumn{1}{p{0.35\textwidth}}{Ich ..., um d...}
& \multicolumn{1}{p{0.55\textwidth}}{Es passiert das und das.}\\
%\midrule
\bottomrule \cellcolor[HTML]{FFFFFF} \vspace{0.1cm}
\end{tabular}
\grayRule
  \caption{Anwendungsfälle}
  \label{tab:usecase}
\end{table}

Dann Use-Case-Diagramm
%
% Funktionalität
%
\section{Funktionalität}
Damit ein Datensatz, wie zum Beispiel ein Adressbucheintrag, offline erreichbar ist, sollte er wenigstens so lange auf dem Client gespeichert werden, bis er vollständig beim Server angekommen sind. Im aktuellen Anwendungsfall bedeutet das, es gibt zwei Kopien des Adressbucheintrags. Eine auf dem Anwendungsgerät, eine auf dem Server.\\
Jeder Fehlerfall muss kommuniziert werden. Wenn es konliktbehaftete Daten gibt muss dies mitgeteilt, und angeboten werden die Konflikte zu lösen (Welche Telefonnummer ist die richtige).\\
\begin{enumerate}
  \item Nur die Einträge laden die ich noch nicht hab
    \subitem kostet Bandbreite und Serverarbeitszeit
    \subitem doppelt (geladen)
    \subitem dauert länger (response)
  \item Einträge identifizieren
    \subitem Operationen müssen dem Objekt/Eintrag zugeordnet werden
  \item Einträge verionieren
  \item Delta berechnen
  \item lokal und auf dem Server gespeichert sein
  \item 2 Objekte mit derselben ID ? welches ist das aktuellste
  \item mehr als 2 Objekte mit derselben ID ? sortieren
  \item (Liste von inhaltsbasierten Versionen muss festgelegte Länge haben)
  \item Konflikte treten auf, deswegen müssen sie effizient gespeichert werden
  \item in Baumstruktur? So kann leicht zu einem konfliktfreiem Zustand (nur 1 Knoten) `navigiert` werden
  \item rekursiv lösen?
\end{enumerate}
\todo{Um Delta zu kalkulieren müssen Daten auf dem Client gespeichert werden (localStorage, lokale Datenbank oder Datei...) | Server muss die Daten sortieren können und in der Lage sein nur bestimmte Daten zu liefern.}
\todo{siehe Anforderungen \hyperref[sub:pwa]{PWA}?} Plus kein Datenverlust, und 'just work'\\
Daten sollen, sobald einmal geladen, auch offline verfügbar sein.
Daten sollen jederzeit (offline und online) lesbar und bearbeitbar (löschbar) sein.
\todo{Nochmal speziell auf Konfliktmanagement eingehen}
%
% UI
%
\section{User experience? Bedienoberfläche?}
\Gls{optimistic UI}\\
\gls{UI} soll mich nicht mit Meldungen darüber nerven, dass ich offline bin\\
\gls{UI} soll sagen wenn es einen Konflikt gab / gibt und mich entscheiden lassen. Bzw ihn lösen lassen.
Auf keinen Fall selber lösen und mich nichts davon wissen lassen.\\
Im besten Fall soll die \gls{UI} mir sagen \b{warum} es zum Konflikt gekommen ist.
