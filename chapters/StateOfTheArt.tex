\chapter{\label{chap:state}Bestehende Konzepte}
Sowohl integriert in Radwegen, als auch für den Einsatz in Kraftfahrzeugen gibt es bereits Projekte zu Ampelassistenten in Bordcomputern, Navigationssystemen, oder als mobile Applikation. Solche Anwendungen erkennen rote Ampeln und ermitteln die optimale Fahrgeschwindigkeit für die Grüne Welle. Auch Erweiterungen für Fahrräder werden immer vielfältiger --- vom einfachen Navigationssystem bis hin zu intelligenten Aufsätzen, die an das \gls{Smartphone} gekoppelt sind.


%%% KOLIBRI %%%
\subsection*{\label{sec:kolibri}Projekt Kolibri}
In Bayern wurde im April 2011 das Pilot-Projekt \textsc{Kolibri}\footnote{ Kooperative Lichtsignaloptimierung -- Bayrisches Pilotprojekt} mit den Teststrecken der B13 bei München mit sieben und der St2145 in der Nähe von Regensburg mit acht ampelgeregelten Kreuzungen gestartet. Gemeinsam untersuchten die Projektpartner\footnote{ \url{http://www.kolibri-projekt.de/Sites/kolibri3.html}} Funktionen und Auswirkungen eines Ampelassistenten außerhalb von Ortschaften. \cite{cobi}\\
"'Per Mobilfunk übertragen die Ampeln ihre Daten an die Zentrale der TRANSVER GmbH. Dort wertet sie ein Computer aus und sendet die Ergebnisse an die Fahrzeuge. Ein Anzeigefeld im Bordcomputer oder eine Applikation auf dem \gls{Smartphone} zeigt an, ob sich das Fahrzeug in der Grünen Welle bewegt."'\\
Die FahrerInnen wurden sowohl fahrzeugintegriert\footnote{ On-Board-Computer} als auch via \gls{Smartphone}, wie Abbildung zeigt, über die Schaltung der nächsten Ampel informiert und erhielten Empfehlungen über die aktuelle \gls{Progressionsgeschwindigkeit}.

%
% FAHRRADERWEITERUNGEN %
%
\clearpage
\section{Fahrraderweiterungen}
Um die Fahrräder intelligenter und attraktiver zu machen, gibt es verschiedene Erweiterungen mit zahlreichen Funktionen. Die hier aufgeführten Fahrraderweiterungen führen zum gewünschten Ziel und ergänzen die Navigation um zusätzliche Eigenschaften, die unter anderem den Weg dorthin erleichtern.
\subsection{Displaylose Fahrradnavigation}
Das \textsc{Hammerhead} ist ein Gerät in der Form eines "'Hammers"' und wird an den Fahrradlenker angebracht. Mit verschiedenfarbigen \glspl{LED} bestückt zeigt es den Weg, warnt vor Hindernissen und ersetzt die vorderen Scheinwerfer.\\
Via Bluetooth ist \textsc{Hammerhead} an das \gls{Smartphone} gekoppelt, auf dem die zugehörige Navigationsanwendung läuft mit der man Routen
Ein sehr ähnliches Prinzip verfolgt das \textsc{CycleNav} von der Firma Schwinn. Unterschiede findet man hier im Design und einem integriertem Lautsprecher, der Abbiegehinweise ausgibt und auf Wunsch wiederholt. \cite{CycleNav}
% ### COBI ###
\subsection{Der COBI Fahrradcomputer}
Ein Projekt aus Frankfurt am Main entwickelt das System \textsc{COBI} (Connected Biking), das alle standardisierten Fahrradsysteme wie Lampen, Navigation, Tachometer etc. vereinen soll. \textsc{COBI} ist ein Modul mit integrierter Frontleuchte, in das man das \gls{Smartphone}, welches dann mit der installierten \textsc{COBI}-\Gls{App} als Fahrradcomputer dient, legt. Durch eine wasser- und stoßfeste Hülle ist es vor Umwelteinflüssen geschützt. Zu dem Lenkersystem gibt es auch Rückstrahler die beim Bremsen intensiver leuchten und eine Blinkfunktion haben.\\
% \begin{figure}[H]
%         \centering
%         \begin{subfigure}[b]{0.49\textwidth}
%                 \includegraphics[width=\textwidth]{COBI_01}
%                 \caption{Frontlicht und Smartphonehalterung}
%                 \label{fig:cobi1}
%         \end{subfigure}%
%         ~ %add desired spacing between images (~, \quad, \qquad, \hfill)
%         \begin{subfigure}[b]{0.49\textwidth}
%                 \includegraphics[width=\textwidth]{COBI_02}
%                 \caption{Bremslicht und Blinker}
%                 \label{fig:cobi2}
%         \end{subfigure}
%         \grayRule
%         \caption[COBI]{COBI -- Das smarte Fahrradsystem. Quelle: \cite{cobi_pic}}
%         \label{fig:cobi}
% \end{figure}
Möchte man das \gls{Smartphone} trotzdem nicht am Lenker haben, bleibt die Verbindung zum Modul über Funk bestehen. Steuern lässt sich das System dann über einen Controller, den man am Lenker angebracht, mit dem Daumen bedienen kann. Ist es jedoch in der Halterung, wird das \gls{Smartphone} über den E-Bike-Akku oder einen zusätzlich integrierten Akku aufgeladen. Wie bei den anderen genannten Systemen ist in der \textsc{COBI}-\Gls{App} eine Navigationsanwendung, wie auch die tracking\&share Funktion inklusive. Darüber hinaus verfügt es über einen Diebstahlschutz, Fitnesstracker sowie die Möglichkeit einer Anbindung an Spotify\footnote{ Digitaler Musikstreaming Dienst}.\\
Das Projekt ist bereits voll finanziert und der Versand der vorbestellten Systeme beginnt voraussichtlich im Frühjahr 2015. \cite{cobi}
