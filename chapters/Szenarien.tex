\chapter{\label{chap:szenarien}Szenarien ...}
Alle in Kapitel \ref{chap:state} angeführten ...  haben die Gemeinsamkeit... \\
Prinzipiell sollte die Anwendung in der Lage sein, ..., die sich aus den Szenarien ergeben. Folgende Situationen können eintreten:\\
\note{langsames Internet, 3G LTE, Timeout während Request}
\begin{description}[leftmargin=0.7cm,style=nextline]
\item[Szenario R1:]
blabla  \\
\item[Szenario R2:]
blabla \\
\item[Szenario R3:]
blabla\\
%
\item[Szenario G1:]
blabla.\\
\item[Szenario G2:]
blabla.\\
\item[Szenario G3:]
blabla.\\
\end{description}
%
% ERGEBNIS
%
\subsection*{Ergebnis}
Da die Szenarien \textit{R2} und \textit{G2}, die Szenarien \textit{R3} und \textit{G3} sowie die Szenazien\textit{V1} und \textit{V2} zusammengefasst werden können, ergeben sich aus den acht Szenarien im Ampelbereich die fünf nun aufgezählten Fälle.
\begin{itemize}
  \item Fall a: blabla
  \item Fall b: blabla
  \item Fall c: blabla
  \item Fall d: blabla
  \item Fall e: blabla
\end{itemize}
Im weiteren Verlauf dieser Arbeit wird unter anderem beschrieben, wie diese Fälle in die Konzeption der zu \it{entwickelnden Anwendung} eingebunden werden.
