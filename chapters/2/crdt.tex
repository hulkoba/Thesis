\sub{Conflict-free replicated data type}
Die Idee von \glspl{CRDT} ist, dass jeder "Typ" (wie ein Einkaufswagen) mit Intelligenz handelt, um Konflikte automatisch zu lösen.\\\\
CRDTs sind Objekte, die ohne teure Synchronisation aktualisiert werden können. Sie konvergieren schließlich, wenn alle gleichzeitigen Aktualisierungen kommutativ\footnote{Unverändert bei Vertauschen der Operanden} sind und wenn alle Aktualisierungen schließlich von jeder Replik ausgeführt werden ~\cite{crdt_shapiro}.
Um diese Garantien zu geben, müssen diese Objekte bestimmte Kriterien erfüllen, welche im Folgenden beschrieben werden ~\cite{crdt_shapiro2}.
% In ihren Arbeiten betrachten Shapiro et. al zwei Replikationsmodelle in einem letztendlich konsistentem verteilten System: den zusatndsbasierten und den operationsbasierten Ansatz. Basierend auf dem Replikationsmodell definieren Sie zwei Arten von CRDTs: CvRDT (vonvergent replicated data type) und CmRDT (kommutativ replizierter Datantyp). Interessanterweise zeigen sie, dass diese beiden Replikationsmodelle und diese beiden Arten von CRDTs äquivalent sind.
\subsub{Zustandbasierter Ansatz}
Wenn ein Replikat ein Update von einem Client empfängt, aktualisiert es zuerst seinen lokalen Status und dann, einige Zeit später, seinen \b{vollständigen Status}. So sendet jedes Replikat gelegentlich seinen vollständigen Status an ein anderes Replikat im System. Um ein Replikat, das den Status eines anderen Replikats empfängt, wendet eine \b{Zusammenführungsfunktion} (merge) an, um den empfangenen Status mit dem lokalen Status zusammenzuführen. Entsprechend sendet dieses Replikat gelegentlich auch seinen Status an ein anderes Replikat, sodass jedes Update schließlich alle Replikate im System erreicht.
\subsub{Operationsbasierter Ansatz (OT?)}
Bei diesem Ansatz sendet ein Replikat seinen vollständigen Status (kann groß sein) nicht an ein anderes Replikat. Stattdessen sendet es nur den \b{Aktualisierungsvorgang} an \b{alle} anderen Replikate im System und erwartet von ihnen, dass sie das Update auf sich anwenden.\\
Da es sich um einen Sendevorgang handelt, wenn zwei Updates $u1$ und $u2$, bei einem Replikat $i$ angewendet werden und diese Updates an zwei Replikate $r1$ und $r2$ gesendet werden, können diese Updates in unterschiedlicher Reihenfolge bei diesen replikaten ankommen. $r1$ kann sie in der Reihenfolge $u1, u2$ empfangen, während bei $r2$ die Updates in umgekehrter Reihenfolge ($u2, u1$) ankommen können. Sind die Aktualisierungen \b{kommutativ}, können die Repliken zussamengeführt werden, egal in welcher Reihenfolge die Updates bei ihnen ankommen - der resultierende Zustand ist derselbe. In diesem Modell wird ein Objekt, für das alle gleichzeitigen Aktualisierungen kommutativ sind, CmRDT (commutative replicated data type - kommutativ replizierter Datentyp) genannt. \\\\
\b{Beispiel:...}\\
CRDTs befassen sich mit einem interessanten und grundlegendem Problem in verteilten Systemen, haben jedoch eine wichtige Einschränkung: "Da ein CRDT konstruktionsbedingt keinen Konsens verwendet, hat der Ansatz starke Einschränkungen; Dennoch sind einige interessante und nicht-triviale CRDTs bekannt" ~\cite{crdt_shapiro2}. Die Einschränkung ist, dass die CRDT-Adresse nur einen Teil des Problemraums betrifft, da nicht alle möglichen Aktualisierungsoperationen kommutativ sind und daher nicht alle Probleme in CRDTs umgewandelt werden können. Auf der anderen Seite können CRDTs für einige Arten von Anwendungen durchaus nützlich sein, da sie eine nette Abstraktion zur Implementierung repliziter verteilter Systeme bieten und gleichzeitig theoretische Konsistenzgarantien bieten.