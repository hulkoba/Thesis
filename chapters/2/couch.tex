
Aufgabe der Replikation von CouchDB ist die Synchronisation 2+n Datenbanken. Lösungen: Zuverlässige \b{Synchronisation} von Datenbanken auf verschiedenen Geräten. \b{Verteilung} der Daten über ein Cluster von DB-Instanzen die jeweils einen Teil des requests beantworten (Lastverteilung) und \b{Spiegelung} der Daten über geografisch weit verteilte Standorte.\\
Durch die inkrementelle (schrittweise) Arbeitsweise kann CouchDB genau dort weitermachen wo es unterbrochen wurde wenn während der Replikation ein Fehler auftritt, beispielsweise durch eine ausfallende Netzwerkverbindung
\it{Es werden auch nur die Daten übertragen, die notwendig sind, um die Datenbanken zu synchronisieren.}\\
Das Besondere an CouchDB ist, dass es darauf ausgerichtet ist, Fehler/Konflikte vernünftig zu behandeln statt anznehmen es träten keine auf (vgl. \cite{couchDB} S. 7f). Wie \hyperref[sec:conflict]{oben} beschrieben, gibt es in Verteilten Systemen einige Fehler die auftreten können.\\\\
\it{Das CouchDB Replikationsmodell erlaubt eine nahtlose, peer-to-peer (direkte) Datensynchronisation zwischen beliebig vielen Geräten. Das CouchDB Replikationsprotokoll ist in CouchDB selbst implementiert, das die Serverkomponente abdeckt. Dann gibt es das PouchDB-Projekt, das dasselbe Protokoll in JavaScript implementiert, das auf Browser- und Node.js-Anwendungen abzielt. das deckt Ihre Kunden und dev-Server ab. Schließlich gibt es Couchbase Mobile und Cloudant Sync, die auf iOS und Android laufen und das CouchDB Synchronisationsprotokoll in Objective-C bzw. Java implementieren.}\\
Vektoruhr~\footnote{\url{https://en.wikipedia.org/wiki/Vector_clock}} \\
content addressable versions: Idee: Nimm den Objektinhalt (content) und jag ihn durch eine \gls{Hashfunktion}\\
%
Diese Art von Konflikten sollten von Menschen gelöst werden. Nur so kann sichergestellt werden, dass die korrekte Änderung gespeichert wird und keine Daten verloren gehen.
%
\subsub{schlussendliche Konsistenz}
\subsub{Lokale Konsistenz}
\subsub{Verteilte Konsistenz}

\subsub{Replikation?}
\subsub{Konfliktmanagement}
