CouchDB verwendet Replikation um Änderungen an Dokumenten zwischen einzelnen Knoten zu synchronisieren.
Die Aufgabe der Replikation von CouchDB ist die Synchronisation zweier oder mehrerer Datenbanken.
Die Replikation in CouchDB erfolgt schrittweise. Alle Änderungen an Dokumenten werden periodisch zwischen den Servern kopiert.
So muss die Verbindung zwischen den Servern nicht permanent bestehen, denn im Feherfall weiß CouchDB wo es unterbrochen wurde und kann an diesem Punkt die Arbeit fortsetzen.
Außerdem ist so jeder Knoten unabhängig von den anderen und das System fällt nicht aus wenn dieser eine Knoten fehlerbehaftet ist.\\
%
Das Besondere an CouchDB ist, dass es darauf ausgerichtet ist, Konflikte vernünftig zu behandeln statt anznehmen es träten keine auf.
Das interne Replikationssystem besitzt eine automatische Konflikterkennung und --lösung.
Wichtig für diesen Mechanismus sind die Revisionsnummern.
Dokumente verden mit Revisionsnummern versioniert. Mit jeder Aktualisierung bekommt es eine neue Revision, die neben der alten gespeichert wird.
Hat ein Dokument durch gleichzeitiges Bearbeiten in zwei unterschiedlichen datenbanke dieselbe Revisionsnumer, erkennt CouchDB den Konflikt und markiert es.
Auch wenn CouchDb die gewinnende und die verlierende Revision festlegt, werden beide Versionen gespeichert.
So kann im Entwicklungsprozess der Anwendung entschieden werden, wie mit den Konflikten umgegangen wird.
Eine Variante wäre, sie von den AnwenderInnen lösen zu lassen (vgl. \cite{couchDB} S. 15ff).\\\\
%
%
%
\it{Das CouchDB Replikationsmodell erlaubt eine nahtlose, peer-to-peer (direkte) Datensynchronisation zwischen beliebig vielen Geräten. Das CouchDB Replikationsprotokoll ist in CouchDB selbst implementiert, das die Serverkomponente abdeckt. Dann gibt es das PouchDB-Projekt, das dasselbe Protokoll in JavaScript implementiert, das auf Browser- und Node.js-Anwendungen abzielt. das deckt Ihre Kunden und dev-Server ab. Schließlich gibt es Couchbase Mobile und Cloudant Sync, die auf iOS und Android laufen und das CouchDB Synchronisationsprotokoll in Objective-C bzw. Java implementieren.}\\
content addressable versions: Idee: Nimm den Objektinhalt (content) und jag ihn durch eine \gls{Hashfunktion}\\
%
Diese Art von Konflikten sollten von Menschen gelöst werden. Nur so kann sichergestellt werden, dass die korrekte Änderung gespeichert wird und keine Daten verloren gehen.
%
%
%
\subsub{Replikation}
%
%
%
\subsub{Konfliktmanagement}
%
% \subsub{schlussendliche Konsistenz}
\subsub{Eventual Consistency}
Das \gls{CAP} Theorem, veranschaulicht in \autoref{fig:cap}, besagt, dass jedes System mit dem Daten über das Netzwerk gesendet werden, nur zwei von den drei möglichen Eigenschaften, Konsistenz, Verfügbarkeit und Partitionstoleranz, garantieren kann.
Konsitzenz der gespeicherten Daten bedeutet, es muss sichergestellt werden dass nach Abschluss der Transaktion auch alle Replikate des manipulierten Datensatzes aktualisiert werden. Der Datensatz ist in jeder Datenbank identisch.
Das System ist besitzt eine hohe Verfügbarkeit wenn alle Anfragen an das System stets beantwortet werden. Die Verfügbarkeit ist gering, wenn die Antwortzeiten des Systems lang sind.
Partitionstoleranz ist gleichzusetzen mit Ausfalltoleranz. Die Datenbank kann auf mehreren Servern verteilt sein. Trotzdem ein Server oder eine Partition ausfällt, kann das System weiterhin funktionieren.
\begin{figure}[H]
  \centering
  \includegraphics[width=0.6\textwidth]{cap}
  \grayRule
  \caption{Das CAP Theorem}
  \label{fig:cap}
\end{figure}
%
 Eventual Consistency ist eine abgeschwächte Variante der Konsistenz, die häufig bei verteilten Datenbanken zur Anwendung kommt. Dabei verzichtet man aus Performancegründen bei Schreiboperationen darauf, Daten sofort auf alle Server/Partitionen zu verteilen.
%
%
Stattdessen kommen Algorithmen zum Einsatz, die sicherstellen, dass nach Beendigung der Schreiboperationen die Daten konsistent gemacht werden, in der Regel ohne Aussage darüber, in welchem Zeitraum der Vorgang abgeschlossen sein wird. In der Zwischenzeit sind unterschiedliche Datenbestände auf den einzelnen Servern. Das kann dazu führen, dass identische, zeitgleiche Abfragen von mehreren Benutzern unterschiedliche Ergebnisse liefern können. Man kann lediglich darauf vertrauen, dass die Daten letztendlich konsistent sind, daher der Name diese Konzeptes.

(vgl. ~\cite{couchDB} S. 11 ff.)
%Zuverlässige Synchronisation von Datenbanken auf verschiedenen Geräten.
%Verteilung der Daten über ein Cluster von DB-Instanzen die jeweils einen Teil des requests beantworten (Lastverteilung) und \b{Spiegelung} der Daten über geografisch weit verteilte Standorte.\\
% Durch die inkrementelle (schrittweise) Arbeitsweise kann CouchDB genau dort weitermachen wo es unterbrochen wurde wenn während der Replikation ein Fehler auftritt, beispielsweise durch eine ausfallende Netzwerkverbindung
% \it{Es werden auch nur die Daten übertragen, die notwendig sind, um die Datenbanken zu synchronisieren.}\\