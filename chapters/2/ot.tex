\def \maya { 3D Design Software. steht unter \url{https://www.autodesk.com/products/maya/free-trial} zum Download bereit}
\def \flash { Browserplugin, steht unter \url{https://get.adobe.com/de/flashplayer/} zum Download bereit}
\def \cad { CAD, engl.: computer-aided design, deutsch: rechnerunterstütztes Konstruieren}
\def \gdocs { Webanwendung, mit der Dokumente erstellt und kollaborativ bearbeitet werden können, \url{https://docs.google.com}}
\def \gwave { Google Wave war eine von Google vorgestelle Webanwendung zur Kommunikation und kollaborativer Zusammenarbeit}
\def \msoffice { Bürosoftware, steht unter \url{https://products.office.com/de-de/compare-all-microsoft-office-products?tab=1} zum Download bereit}
%
\gls{OT} ist eine weit verbreitete Technik zur Unterstützung von Funktionalitäten in \gls{kollaborativ}er Software.
Sie stammt aus einer 1989 veröffentlichten Forschungsarbeit und wurde ursprünglich nur für die gemeinsame Bearbeitung von Klartext-Dokumenten entwickelt~\cite{ot_paper}. Etwas später wurden einige Unvollständigkeiten im Algorithmus entdeckt und es wurden unabhängig voneinander verschiedene Lösungsvorschläge erarbeitet ~\cite{ot-later}.
Heute unterstützt \gls{OT} zusätzlich das \gls{kollaborativ}e Bearbeiten von \gls{HTML}, RTF und XML Dokumenten, Adobe Flash\footnote{\flash} Grafiken und Dokumenten in CAD\footnote{\cad} Tools wie Autodesk Maya\footnote{\maya}.
Des Weiteren können mit \gls{OT} Dateien die in Microsoft Office\footnote{\msoffice} enthalten sind wie Word Dokumente, PowerPoint Folien und Excel Tabellen kollaborativ bearbeitet werden, aber auch Dokumente in webbasierten Anwendungen wie Google Docs\footnote{\gdocs} und Google Wave\footnote{\gwave}~\cite{ot-faq}.\\\\
%
%
Operational Transform ist ein Algorithmus für die Transformationen von Operationen, die auf Dokumente mit unterschiedlichen Zustand angewendet werden, um diese Dokumente in den identischen Zustand zu versetzen.
Jede Änderung an einem freigegebenen Dokument wird als Operation dargestellt.
Eine Operation ist die Repräsentation einer Änderung an einem Dokument und zeichnet im Wesentlichen den Unterschied zwischen der aktuellen und der nachfolgenden Version eines Dokuments auf.
Die Anwendung einer Operation auf das aktuelle Dokument führt zu einem neuen Dokumentstatus.
Es gibt beispielsweise die Operation ''Einfügen''. 
Das Einfügen besteht aus dem eingefügten Text und dessen Position im Dokument. Die Operation, die beschreibt ''Füge den Buchstaben x an dritter Stelle im Dokument ein'' sieht so aus: (\tt{insert('x', 3)}).
Für die Position kann ein Koordinatensystem ermittelt werden, Zeilennummer und die Position in einer Zeile, oder das Dokument wie eine Folge von Zeichen behandeln werden.\\
%
\Gls{kollaborativ}e Systeme, die \gls{OT} verwenden, benutzen normalerweise den replizierten Dokumentenspeicher.
Das heißt, auf jedem Gerät befindet sich eine eigene Kopie des Dokuments.
Die Operationen erfolgen auf der lokalen Kopie und die Änderungen werden an alle anderen Geräte weitergegeben.\\\\
%
Um konkurrierende Operationen zu behandeln, gibt es die \tt{transform} Funktion.
Wenn ein Client die Änderungen von einem anderen empfängt, werden die Änderungen vor ihrer Ausführung transformiert.
Sie nimmt zwei Operationen, die auf demselben Dokument, aber auf unterschiedlichen Geräten angewendet wurden und berechnet daraus eine neue.
Die neue Operation wird dann nach der zweiten Operation angewendet. Die erste, beabsichtigte Operation wird beibehalten~\cite{ot_paper}.\\
%
Die Grundidee des \gls{OT} Algorithmus wird anhand eines Beispiels, das die \autoref{fig:ot} veranschaulicht, illustriert.\\
Es gibt den initialen Text ''abc'', der auf den Geräten der kollaborativ arbeitenden Personen, Amilia und Rory, besteht.
Amilia fügt lokal vor dem ''abc'' ein ''x'' ein, Rory löscht auf seinem Gerät den Buchstaben ''c'''.
%
\begin{figure}[h]
  \centering
  \includegraphics[width=0.8\textwidth]{ot-j}
  \grayRule
  \caption{Die Grundidee von \gls{OT}}
  \label{fig:ot}
\end{figure}
%
Die konkurrierenden Operationen sind \tt{insert(0, ''x'')}, ''Füge das Zeichen ''x'' an Stelle Null ein'', und \tt{delete(1, 2)}, ''lösche ein Zeichen an der zweiten Stelle''.
In \gls{OT} werden lokale Änderungen angewandt, so wie sie passieren.
Operationen über das Netzwerk werden, bevor sie auf zuvor ausgeführte Operationen angewandt werden, transformiert.\\
%
Wenn die Operation von Rory zuerst ausgeführt wird, gibt es kein Problem. Der Buchstabe an zweiter Stelle wird gelöscht und danach wird an Stelle Null ein ''x'' eingefügt. Das Ergebnis ist das gewünschte ''xab''.
Im Bild ist auf der rechten Seite der Ablauf dieses Vorgangs abzulesen, wie er in \gls{OT} stattfindet.
Rory löscht den Buchstaben ''c'' und im Dokument steht ein ''ab''.
Amilias Editierung, \it{O2} kommt an und wird gegen \it{O1} transformiert. In diesem Fall ist das Ergebnis der Transformation dasselbe wie vorher.
Die Ausführung von Rories Änderung hat keinen Einfluss auf die von Amilia.
Wird Amilias Änderung auf ''ab'' angewendet, wird das ''x'' an erster Stelle hinzugefügt und das Ergebnis, ''xab'' ist identisch mit dem auf Amilias Seite.\\
%
Würde allerdings die Operation von Amilia zuerst ausgeführt werden, gäbe es ein Problem.
Dann würde der Buchstabe ''b'' gelöscht, da er sich, nachdem das ''x'' eingefügt wurde, an zweiter Stelle steht.
Das Ergebnis nach der ersten Operation wäre ''xac''. Die Dokumentenstatus wären nicht identisch.\\
%
Auf der linken Seite wird die Operation \it{O2} zuerst ausgeführt im Dokument steht ''xabc''.
Dann kommt \it{O1} auf Amilias Gerät an und wird zu \it{O2'} transformiert. Die neue Operation ist nun \tt{delete(1, 3)}, ''lösche ein Zeichen an dritter Stelle''. Der Positionsparameter wurde um eins hochgesetzt, weil das Einfügen des Buchstabens ''x'' durch die konkurrierenden Operation beachtet wurde.
Wird nun \it{O2'} auf der Operation \it{O1} angewendet, wird der korrekte Buchstabe gelöscht und das Ergebnis ist, wie auf Rories Seite ''xab''.
Wird die Operation von Rory, \tt{delete(1, 2)}, gegen die Operation von Amilia transformiert, wird berücksichtigt, dass Amilia ein Zeichen vor der Position zwei, der Poition des von Rory gelöschten Zeichens, eingefügt hat.\\\\
%
%
Zuammenfassend besteht die Grundidee von \gls{OT} darin, eine Operation in eine neue Form umzuwandeln.
Die neue Form ergibt sich aus den Auswirkungen zuvor durchgeführter Operationen.
So können die transformierten Operationen die korrekte Wirkung erzielen und eine Konsistenz der replizierten Dokumente sicherstellen~\cite{ot-later}.
%
% Kritik
%
%False-Tie puzzle? \it{A Generic Operation Transformation Scheme for Consistency Maintenance in Real-time Cooperative Editing Systems} und \it{Achieving convergence, causality-preservation, and intention-preservation in real-time cooperative editing systems}\\\\
%Während der klassische OT-Ansatz, Operationen durch ihre Versätze im Text zu definieren, einfach und natürlich zu sein scheint, werfen real verteilte Systeme ernsthafte Probleme auf. Nämlich, dass sich die Operationen mit endlicher Geschwindigkeit fortpflanzen, die Zustände der TeilnehmerInnen sind oft verschieden, so dass die resultierenden Kombinationen von Zuständen und Operationen extrem schwer vorherzusehen und zu verstehen sind.
%Wie Li und Li es ausdrückten: ``Aufgrund der Notwendigkeit, eine komplizierte Fallabdeckung in Betracht zu ziehen, sind formale Beweise sehr kompliziert und fehleranfällig, selbst für OT-Algorithmen, die nur zwei charakteristische Primitive behandeln (Einfügen und Löschen)``~\cite{ot-critic}.\\
%Damit OT funktioniert, muss jede einzelne Änderung an den Daten erfasst werden: ``Einen Schnappschuss des Zustands zu erhalten, ist normalerweise trivial, aber das Erfassen von Bearbeitungen ist eine ganz andere Sache. [...] Der Reichtum moderner Benutzerschnittstellen kann dies problematisch machen, besonders in einer browserbasierten Umgebung`` ~\cite{diff_sync}.
%
% It is designed to let any number of people collaborate on text at the same time and it can deal with a certain level of network instability, but generally, it requires clients to be connected at all times. If you go and keep editing, your changes can be integrated later, but not indefinitely later.
%In addition, it is designed for text, not for generic objects, so for true offline capabilities of generic data objects, Operational Transforms are less useful. Check them out if you need a solution for mostly connected text.