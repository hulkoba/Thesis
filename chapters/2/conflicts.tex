% 1. Definition
% 2. Konfliktmanagement
\section{\label{sec:conflict}Konflikte}
\todo{was ist ein Konflikt? Ab wann kann ein Konflikt vom System gelöst werden, ab wann nicht? (siehe git: gleichzeitige Änderung an unterschiedlichen Stellen: kein Problem. Gleichzeitige Änderung an derselben Stelle: Konflikt!)}\\
Verteilte Systeme: Das ist ein mächtiger Begriff für viele Ideen und Konzepten, aber es läuft in der Regel darauf hinaus: Da sind zwei oder mehr Computer, die durch ein Netzwerk verbunden sind und es wird versucht, dass einige der Daten auf beiden Computern gleich aussehen. ==> Ein System das zuverlässig über ein Netzwerk funktioniert.\\
Zwei Geräte, ein Server, über Netzwerk verbunden.\\\\
Spezielle Eigenschaft von Netzwerken: Verbindung kann jederzeit abbrechen:
Acht Irrtümer der verteilten Datenverarbeitung:
% \begin{enumerate}
%   \item Das Netzwerk ist zuverlässig
%   \item Die \gls{Latenz}zeit ist gleich null
%   \item Die Bandbreite ist unendlich
%   \item Das Netzwerk ist (informations)sicher
%   \item Die Netzwerkstruktur wird sich nicht ändern
%   \item Es gibt eineN AdministratorIn
%   \item Die Datentransportkosten sind gleich null
%   \item Das Netzwerk ist homogen
% \end{enumerate}
\begin{description}[leftmargin=0.5cm,style=nextline]
	\item[1. Das Netzwerk ist zuverlässig] ~ Der Strom kann ausfallen oder Glasfaserkabel können kaputt sein --- Das Netzwerk ist nicht zuverlässig.
	\item[2. Die \gls{Latenz} ist gleich null] ~ Glasfaserkabel werden durch Mikrowellen (oder andere Technologien) ersetzt um Millisekunden an Zeit zu sparen. Das würde nicht passieren, wäre die \gls{Latenz} bei null. Es dauert nun mal eine gewisse Zeit(ms) wenn ein Signal eine (geografisch)weite Strecke zurücklegen muss --- Die Latenz ist nicht gleich null.
	\item[3. Die \gls{Bandbreite} ist unendlich] ~ Daten können nicht schneller fließen als die Komponenten die sie verarbeiten (\gls{Middleware}, Datenbank \ldots) --- Die Bandbreite ist nicht unendlich.
	\item[4. Das Netzwerk ist sicher] ~ Der \sc{Heartbeat-bug}\footnote{\url{http://heartbleed.com/} -- Zugriff: 07.04.2018}, der im Jahr 2014 behoben wurde und die Sicherheitslücke im ICE-\gls{WLAN} im Jahr 2016\footnote{\url{https://netzpolitik.org/2016/datenschutz-im-zug-deutsche-bahn-will-sicherheitsluecke-in-neuem-ice-wlan-schliessen/} -- Zugriff: 07.04.2018} sind nur zwei Beispiele die zeigen, dass das Netzwerk nicht sicher ist.
	\item[5. Die Netzwerkstruktur wird sich nicht ändern] ~ Eine Datenbank kann beispielsweise über mehrere Server verteilt sein, die (teilweise) voneinander abhängig sind. Ein Server mit Abhängigkeiten kann ausfallen, es kann eine Aktualisierung für einen anderen Server geben --- die Struktur ändert sich.
	\item[5. Die Netzwerkstruktur wird sich nicht ändern] ~ Eine Datenbank kann beispielsweise über mehrere Server verteilt sein, die (teilweise) voneinander abhängig sind. Ein Server mit Abhängigkeiten kann ausfallen, es kann eine Aktualisierung für einen anderen Server geben --- die Struktur ändert sich.
	\item[6. Es gibt eineN AdministratorIn] ~ Es kann beliebig viele AdministratorInnen geben.
	\item[7. Die Datentransportkosten sind gleich null] ~ Netflix bezahlte anfang 2014 diversen InternetanbieterInnen dafür, dass Netflix KundInnen bevorzugten Internetzugang haben.
	\item[8. Das Netzwerk ist homogen] ~ Es gibt verschiedene Arten von Netzwer: 3G, 4G, LTE, WiFi. Wird beeinflusst durch Hardware (Smartphone, Tablet, PC, Laptop, Router \ldots)~\cite{fallacies}
\end{description}
\sub{Konfliktlöstungsstrategien}
\subsub{Git}
Beschreiben wie Git Konflikte löst.
\sub{\gls{CAP} Theorem?}
%
% Couch Pouch
%
CouchDB verwendet Replikation um Änderungen an Dokumenten zwischen einzelnen Knoten zu synchronisieren.

Dokumente werden versioniert

Inkrementelle Replikation


Aufgabe der Replikation von CouchDB ist die Synchronisation 2+n Datenbanken. Lösungen: Zuverlässige \b{Synchronisation} von Datenbanken auf verschiedenen Geräten. \b{Verteilung} der Daten über ein Cluster von DB-Instanzen die jeweils einen Teil des requests beantworten (Lastverteilung) und \b{Spiegelung} der Daten über geografisch weit verteilte Standorte.\\
Durch die inkrementelle (schrittweise) Arbeitsweise kann CouchDB genau dort weitermachen wo es unterbrochen wurde wenn während der Replikation ein Fehler auftritt, beispielsweise durch eine ausfallende Netzwerkverbindung
\it{Es werden auch nur die Daten übertragen, die notwendig sind, um die Datenbanken zu synchronisieren.}\\
Das Besondere an CouchDB ist, dass es darauf ausgerichtet ist, Fehler/Konflikte vernünftig zu behandeln statt anznehmen es träten keine auf (vgl. \cite{couchDB} S. 7f). Wie \hyperref[chap:conflict]{oben} beschrieben, gibt es in Verteilten Systemen einige Fehler die auftreten können.\\\\
\it{Das CouchDB Replikationsmodell erlaubt eine nahtlose, peer-to-peer (direkte) Datensynchronisation zwischen beliebig vielen Geräten. Das CouchDB Replikationsprotokoll ist in CouchDB selbst implementiert, das die Serverkomponente abdeckt. Dann gibt es das PouchDB-Projekt, das dasselbe Protokoll in JavaScript implementiert, das auf Browser- und Node.js-Anwendungen abzielt. das deckt Ihre Kunden und dev-Server ab. Schließlich gibt es Couchbase Mobile und Cloudant Sync, die auf iOS und Android laufen und das CouchDB Synchronisationsprotokoll in Objective-C bzw. Java implementieren.}\\
content addressable versions: Idee: Nimm den Objektinhalt (content) und jag ihn durch eine \gls{Hashfunktion}\\
%
Diese Art von Konflikten sollten von Menschen gelöst werden. Nur so kann sichergestellt werden, dass die korrekte Änderung gespeichert wird und keine Daten verloren gehen.
%
% \subsub{schlussendliche Konsistenz}
\subsub{Eventual Consistency}
Das \gls{CAP} Theorem, veranschaulicht in \autoref{fig:cap}, besagt, dass jedes System mit dem Daten über das Netzwerk gesendet werden, nur zwei von den drei möglichen Eigenschaften, Konsistenz, Verfügbarkeit und Partitionstoleranz, garantieren kann.
Konsitzenz der gespeicherten Daten bedeutet, es muss sichergestellt werden dass nach Abschluss der Transaktion auch alle Replikate des manipulierten Datensatzes aktualisiert werden. Der Datensatz ist in jeder Datenbank identisch.
Das System ist besitzt eine hohe Verfügbarkeit wenn alle Anfragen an das System stets beantwortet werden. Die Verfügbarkeit ist gering, wenn die Antwortzeiten des Systems lang sind.
Partitionstoleranz ist gleichzusetzen mit Ausfalltoleranz. Die Datenbank kann auf mehreren Servern verteilt sein. Trotzdem ein Server oder eine Partition ausfällt, kann das System weiterhin funktionieren.
\begin{figure}[H]
  \centering
  \includegraphics[width=0.6\textwidth]{cap}
  \grayRule
  \caption{Das CAP Theorem}
  \label{fig:cap}
\end{figure}
%
 Eventual Consistency ist eine abgeschwächte Variante der Konsistenz, die häufig bei verteilten Datenbanken zur Anwendung kommt. Dabei verzichtet man aus Performancegründen bei Schreiboperationen darauf, Daten sofort auf alle Server/Partitionen zu verteilen.
%
%
Stattdessen kommen Algorithmen zum Einsatz, die sicherstellen, dass nach Beendigung der Schreiboperationen die Daten konsistent gemacht werden, in der Regel ohne Aussage darüber, in welchem Zeitraum der Vorgang abgeschlossen sein wird. In der Zwischenzeit sind unterschiedliche Datenbestände auf den einzelnen Servern. Das kann dazu führen, dass identische, zeitgleiche Abfragen von mehreren Benutzern unterschiedliche Ergebnisse liefern können. Man kann lediglich darauf vertrauen, dass die Daten letztendlich konsistent sind, daher der Name diese Konzeptes.

(vgl. ~\cite{couchDB} S. 11 ff.)
\subsub{Lokale Konsistenz}
\subsub{Verteilte Konsistenz}

\subsub{Replikation?}
\subsub{Konfliktmanagement}

