\chapter{\label{chap:implementierung}Implementierung}
\section{Installation}
\sub{1}
\subsub{Pouch}
\subsub{Couch}
\sub{2}
%
% Installationsanleitung
%
\section{Installationsanleitung}
Beide entwickelten Prototypen sind als öffentliche Repositories auf GitHub\footnote{Software--Entwicklungs--Plattform \url{https://github.com/}} zu finden. 
Um sie zu installieren müssen folgende Schritte ausgeführt werden.
\sub{amilia-qouch}
1. Zuerst muss das Repository kopiert werden:
\lstset{language=sh, caption={},belowcaptionskip=0.3\baselineskip}
\begin{lstlisting}
git clone git@github.com:hulkoba/amilia-qouch.git
# oder
git clone https://github.com/hulkoba/amilia-qouch.git
\end{lstlisting}
2. Dann muss man in das Verzeichnis navigieren und alle Abhängigkeiten installieren.
\begin{lstlisting}
cd amilia-qouch
npm install
\end{lstlisting}
3. Mittels
\begin{lstlisting}
npm start
\end{lstlisting}
wird die Anwendung gestartet.
%
\sub{amilia-rdx}
1. Auch hier muss das Repository zuerst kopiert werden:
\lstset{language=sh, caption={},belowcaptionskip=0.3\baselineskip}
\begin{lstlisting}
git clone git@github.com:hulkoba/amilia-rdx.git
# oder
git clone https://github.com/hulkoba/amilia-rdx.git
\end{lstlisting}
2. Schritt zwei ist identisch mit dem in der \tt{amilia-qouch} Anleitung\\
3. Mittels
\begin{lstlisting}
npm run server
npm start
\end{lstlisting}
wird zuerst der Server, dann die Anwendung gestartet.