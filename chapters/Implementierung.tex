\chapter{\label{chap:implementierung}Der Prototyp der Anwendung}
In diesem Kaptitel wird nach dem in Kapitel \ref{chap:konzeption} präsentiertem Lösungsweg die detaillierte Beschreibung der technischen Realisierung der Anwendung vorgestellt.\\
Nach der Beschreibung der Konfigurationsdateien wird auf die Umsetzung der Szenarien eingegangen. Im Zuge dessen werden die implementierten Algorithmen vorgestellt, ....

\section{Die package.json Datei}

%
% Installationsanleitung
%
\section{Installationsanleitung}
Um die Anwendung zu installieren, muss deren .apk-Datei auf das Gerät geladen und von dort gestartet werden. Die .apk-Datei ist das als Zwischencode ausführbare Kompilat, welches dann auf dem Gerät zu Plattformcode kompiliert wird. Zum Auffinden und Ausführen der Datei wird ein Dateimanager, wie zum Beispiel der kostenlose Datei Manager\footnote{ Der Datei Manager (File Manager) steht im Google Play Store unter \url{https://play.google.com/store/apps/details?id=com.rhmsoft.fm} bereit.},  benötigt.