\chapter{\label{chap:grundlagen}Grundlagen}
Was bedeutet offlinefähig?\\
Native \Glspl{App} existieren und funktionieren grundsätzlich solange offline, bis sie versuchen online Daten abzurufen.
%
% Offline-First
%
\section{Offline First}
Offline-First heißt, die Bestandteile einer Anwendung so zu verwalten, dass nach der ersten Verwendung keine Internetverbindung mehr notwendig ist um deren grundlegenden Funktionen zu nutzen.\todo{Quelle}\\
Eine Anwendung, die für den Offline-Gebrauch entwickelt wurde, ist sowohl mit, als auch ohne Internetverbindung vollständig einsatzbereit. Bei einer bestehenden Internetverbindung ist das Laden der \gls{Assets} aus dem Cache schneller als aus dem Netz. Daten, die zuerst lokal gespeichert werden, gehen auch bei plötzlichen Verbindungsverlust nicht verloren.
% 1. Separate Apps from Data\\
% 2. Deliver App Code (and make it cachable) {appcache \& ServiceWorkers}\\
% 3. Save Data Offline {localStrogare / localForage, IndexedDB, other Wrapper}\\
% 4. Detect Connectivity {navigator.onLine} (Lie-fi)\\
% 5. Sync Data - Build upon existing solutions --CouchDB/PouchDB | remoteStorage\\
%
% Anforderungen
%
\sub{Anforderungen an Offline-First Anwendungen}
Eine Webanwendung besteht grundsätzlich aus zwei Bestandteilen: \it{\gls{Assets}} und benutzerInnengenerierte Daten.\\
Um offline nutzbar zu sein, müssen einige Voraussetzungen erfüllt werden.
\begin{itemize}
  \item müssen auf dem Endgerät gespeichert werden
  \item Serverseitige (nicht lokal) Datenbank \&\& Synchronisation zwischen Server ind Client
  \item Kollaborativ \&\& Sync zwischen allen Beteiligten
  \item Synchronisation erfordert den Umgang mit Konflikten
  \item Kein Datenverlust
\end{itemize}
\todo{merge mit PWA?}
%
% PWA
%
\sub{\label{sub:pwa}Progressive Web Apps}
\Gls{PWA} ist eine Bezeichnung für eine mobil nutzbare Webseite, die eine Brücke zwischen der nativen
Applikation und einer Webseite schlägt.
Der Begriff \gls{PWA} wurde im Jahr 2015  von Alex Russel und seiner Frau Frances Berriman geprägt.
Dieser beschreibt Webseiten, die die positiven Funktionen von nativen Applikationen mitbringen, aber statt über App Stores installiert zu werden, im Webbrowser existieren. Die Webseiteninhalte sind ohne die Installation sofort und jederzeit für die NutzerInnen abrufbar. Schon beim zweiten Besuch der Webseite ist die Ladezeit der Daten verkürzt und sie ist offline, oder auch bei schlechter Internetverbindung nutzbar. Nach mehrmaligem Aufruf kann die \gls{PWA} über den Browser installiert und zum Startbildschirm hinzugefügt werden. Russel und Berriman legen folgende Einenschaften einer \gls{PWA} fest:
\begin{description}[leftmargin=0.5cm,style=nextline]
    \item[Responsive] to fit any form factor
    \item[Connectivity independent] Progressively-enhanced with Service Workers to let them work offline
    \item[App-like-interactions] Adopt a Shell + Content application model to create appy navigations \& interactions
    \item[Fresh] Transparently always up-to-date thanks to the Service Worker update process
    \item[Safe] Served via TLS (a Service Worker requirement) to prevent snooping
    \item[Discoverable] Are identifiable as “applications” thanks to W3C Manifests and Service Worker registration scope a llowing search engines to find them
    \item[Re-engageable] Can access the re-engagement UIs of the OS; e.g. Push Notifications
    \item[Installable] to the home screen through browser-provided prompts, allowing users to “keep” apps they find most useful without the hassle of an app store
    \item[Linkable] meaning they’re zero-friction, zero-install, and easy to share. The social power of URLs matters.
\end{description}
\todo{Näher erläutern?}~\cite{pwa}.

%
% ServiceWorker
%
\subsub{ServiceWorker}
\subsub{localForage und AsyncStorage}
\subsub{IndexedDB}
%
% Konflikte [Jan]
%
\section{\label{sec:conflict}Konflikte}
Verteilte Systeme: Das ist ein mächtiger Begriff für viele Ideen und Konzepten, aber es läuft in der Regel darauf hinaus: Da sind zwei oder mehr Computer, die durch ein Netzwerk verbunden sind und es wird versucht, dass einige der Daten auf beiden Computern gleich aussehen. ==> Ein System das zuverlässig über ein Netzwerk funktioniert.\\
Zwei Geräte, ein Server, über Netzwerk verbunden.\\\\
Spezielle Eigenschaft von Netzwerken: Verbindung kann jederzeit abbrechen:
Acht Irrtümer der verteilten Datenverarbeitung:
% \begin{enumerate}
%   \item Das Netzwerk ist zuverlässig
%   \item Die \gls{Latenz}zeit ist gleich null
%   \item Die Bandbreite ist unendlich
%   \item Das Netzwerk ist (informations)sicher
%   \item Die Netzwerkstruktur wird sich nicht ändern
%   \item Es gibt eineN AdministratorIn
%   \item Die Datentransportkosten sind gleich null
%   \item Das Netzwerk ist homogen
% \end{enumerate}
\begin{description}[leftmargin=0.5cm,style=nextline]
  \item[1. Das Netzwerk ist zuverlässig] ~ Der Strom kann ausfallen oder Glasfaserkabel können kaputt sein --- Das Netzwerk ist nicht zuverlässig.
 \item[2. Die \gls{Latenz} ist gleich null] ~ Glasfaserkabel werden durch Mikrowellen (oder andere Technologien) ersetzt um Millisekunden an Zeit zu sparen. Das würde nicht passieren, wäre die \gls{Latenz} bei null. Es dauert nun mal eine gewisse Zeit(ms) wenn ein Signal eine (geografisch)weite Strecke zurücklegen muss --- Die Latenz ist nicht gleich null.
 \item[3. Die \gls{Bandbreite} ist unendlich] ~ Daten können nicht schneller fließen als die Komponenten die sie verarbeiten (\gls{Middleware}, Datenbank \ldots) --- Die Bandbreite ist nicht unendlich.
 \item[4. Das Netzwerk ist sicher] ~ Der \sc{Heartbeat-bug}\footnote{\url{http://heartbleed.com/} -- Zugriff: 07.04.2018}, der im Jahr 2014 behoben wurde und die Sicherheitslücke im ICE-\gls{WLAN} im Jahr 2016\footnote{\url{https://netzpolitik.org/2016/datenschutz-im-zug-deutsche-bahn-will-sicherheitsluecke-in-neuem-ice-wlan-schliessen/} -- Zugriff: 07.04.2018} sind nur zwei Beispiele die zeigen, dass das Netzwerk nicht sicher ist.
 \item[5. Die Netzwerkstruktur wird sich nicht ändern] ~ Eine Datenbank kann beispielsweise über mehrere Server verteilt sein, die (teilweise) voneinander abhängig sind. Ein Server mit Abhängigkeiten kann ausfallen, es kann eine Aktualisierung für einen anderen Server geben --- die Struktur ändert sich.
 \item[5. Die Netzwerkstruktur wird sich nicht ändern] ~ Eine Datenbank kann beispielsweise über mehrere Server verteilt sein, die (teilweise) voneinander abhängig sind. Ein Server mit Abhängigkeiten kann ausfallen, es kann eine Aktualisierung für einen anderen Server geben --- die Struktur ändert sich.
 \item[6. Es gibt eineN AdministratorIn] ~ Es kann beliebig viele AdministratorInnen geben.
 \item[7. Die Datentransportkosten sind gleich null] ~ Netflix bezahlte anfang 2014 diversen InternetanbieterInnen dafür, dass Netflix KundInnen bevorzugten Internetzugang haben.
 \item[8. Das Netzwerk ist homogen] ~ Es gibt verschiedene Arten von Netzwer: 3G, 4G, LTE, WiFi. Wird beeinflusst durch Hardware (Smartphone, Tablet, PC, Laptop, Router \ldots)~\cite{fallacies}
\end{description}
\sub{Git}
Beschreiben wie Git Konflikte löst.
\sub{\gls{CAP} Theorem}
%
% Strategien
%
\section{Sync in verteilten Systemen}
Es stellt sich heraus, dass die Implementierung dieser Art von Echtzeit-Zusammenarbeit alles andere als trivial ist.
Im Folgenden werden die drei Strategien \gls{OT}, \gls{CRDT} und \gls{LWW} vorgestellt plus CouchDBs Peplikationsmodell.
%
% Last-write-wins
%
%Der \gls{LWW} Ansatz geht davon aus, dass Schreibvorgänge immer in der richtigen Reihenfolge ausgeführt werden. Der letzte an die Datenbank gesendete Schreibvorgang wird als korrekt angenommen.
%Konflikte werden bei dieser Strategie ignoriert
Diese Strategie ist leicht zu implementieren da lediglich festgestellt werden muss welche Manipulation die neuere ist, was durch die Vergabe eines Zeitstempels problemlos zu errechnen ist ~\cite{lww}.\\
%\quote{When multiple updates are applied to the same column, Cassandra uses clientprovided timestamps to resolve conflicts}
\sub{LWW}
\sub{OT}
\sub{CRDT}

%
% Operational Transform
%
%\def \maya { 3D Design Software. steht unter \url{https://www.autodesk.com/products/maya/free-trial} zum Download bereit}
\def \flash { Browserplugin, steht unter \url{https://get.adobe.com/de/flashplayer/} zum Download bereit}
\def \cad { CAD, engl.: computer-aided design, deutsch: rechnerunterstütztes Konstruieren}
\def \gdocs { Webanwendung, mit der Dokumente erstellt und kollaborativ bearbeitet werden können, \url{https://docs.google.com}}
\def \gwave { Google Wave war eine von Google vorgestelle Webanwendung zur Kommunikation und kollaborativer Zusammenarbeit}
\def \msoffice { Bürosoftware, steht unter \url{https://products.office.com/de-de/compare-all-microsoft-office-products?tab=1} zum Download bereit}
%
\gls{OT} ist eine weit verbreitete Technik zur Unterstützung von Funktionalitäten in \gls{kollaborativ}er Software.
Sie stammt aus einer 1989 veröffentlichten Forschungsarbeit und wurde ursprünglich nur für die gemeinsame Bearbeitung von Klartext-Dokumenten entwickelt~\cite{ot_paper}. Etwas später wurden einige Unvollständigkeiten im Algorithmus entdeckt und es wurden unabhängig voneinander verschiedene Lösungsvorschläge erarbeitet ~\cite{ot-later}.
Heute unterstützt \gls{OT} zusätzlich das \gls{kollaborativ}e Bearbeiten von \gls{HTML}, RTF und XML Dokumenten, Adobe Flash\footnote{\flash} Grafiken und Dokumenten in CAD\footnote{\cad} Tools wie Autodesk Maya\footnote{\maya}.
Des Weiteren können mit \gls{OT} Dateien die in Microsoft Office\footnote{\msoffice} enthalten sind wie Word Dokumente, PowerPoint Folien und Excel Tabellen kollaborativ bearbeitet werden, aber auch Dokumente in webbasierten Anwendungen wie Google Docs\footnote{\gdocs} und Google Wave\footnote{\gwave}~\cite{ot-faq}.\\\\
%
%
Operational Transform ist ein Algorithmus für die Transformationen von Operationen, die auf Dokumente mit unterschiedlichen Zustand angewendet werden, um diese Dokumente in den identischen Zustand zu versetzen.
Jede Änderung an einem freigegebenen Dokument wird als Operation dargestellt.
Eine Operation ist die Repräsentation einer Änderung an einem Dokument und zeichnet im Wesentlichen den Unterschied zwischen der aktuellen und der nachfolgenden Version eines Dokuments auf.
Die Anwendung einer Operation auf das aktuelle Dokument führt zu einem neuen Dokumentstatus.
Es gibt beispielsweise die Operation ''Einfügen''. 
Das Einfügen besteht aus dem eingefügten Text und dessen Position im Dokument. Die Operation, die beschreibt ''Füge den Buchstaben x an dritter Stelle im Dokument ein'' sieht so aus: (\tt{insert('x', 3)}).
Für die Position kann ein Koordinatensystem ermittelt werden, Zeilennummer und die Position in einer Zeile, oder das Dokument wie eine Folge von Zeichen behandeln werden.\\
%
\Gls{kollaborativ}e Systeme, die \gls{OT} verwenden, benutzen normalerweise den replizierten Dokumentenspeicher.
Das heißt, auf jedem Gerät befindet sich eine eigene Kopie des Dokuments.
Die Operationen erfolgen auf der lokalen Kopie und die Änderungen werden an alle anderen Geräte weitergegeben.\\\\
%
Um konkurrierende Operationen zu behandeln, gibt es die \tt{transform} Funktion.
Wenn ein Client die Änderungen von einem anderen empfängt, werden die Änderungen vor ihrer Ausführung transformiert.
Sie nimmt zwei Operationen, die auf demselben Dokument, aber auf unterschiedlichen Geräten angewendet wurden und berechnet daraus eine neue.
Die neue Operation wird dann nach der zweiten Operation angewendet. Die erste, beabsichtigte Operation wird beibehalten~\cite{ot_paper}.\\
%
Die Grundidee des \gls{OT} Algorithmus wird anhand eines Beispiels, das die \autoref{fig:ot} veranschaulicht, illustriert.\\
Es gibt den initialen Text ''abc'', der auf den Geräten der kollaborativ arbeitenden Personen, Amilia und Rory, besteht.
Amilia fügt lokal vor dem ''abc'' ein ''x'' ein, Rory löscht auf seinem Gerät den Buchstaben ''c'''.
%
\begin{figure}[h]
  \centering
  \includegraphics[width=0.8\textwidth]{ot-j}
  \grayRule
  \caption{Die Grundidee von \gls{OT}}
  \label{fig:ot}
\end{figure}
%
Die konkurrierenden Operationen sind \tt{insert(0, ''x'')}, ''Füge das Zeichen ''x'' an Stelle Null ein'', und \tt{delete(1, 2)}, ''lösche ein Zeichen an der zweiten Stelle''.
In \gls{OT} werden lokale Änderungen angewandt, so wie sie passieren.
Operationen über das Netzwerk werden, bevor sie auf zuvor ausgeführte Operationen angewandt werden, transformiert.\\
%
Wenn die Operation von Rory zuerst ausgeführt wird, gibt es kein Problem. Der Buchstabe an zweiter Stelle wird gelöscht und danach wird an Stelle Null ein ''x'' eingefügt. Das Ergebnis ist das gewünschte ''xab''.
Im Bild ist auf der rechten Seite der Ablauf dieses Vorgangs abzulesen, wie er in \gls{OT} stattfindet.
Rory löscht den Buchstaben ''c'' und im Dokument steht ein ''ab''.
Amilias Editierung, \it{O2} kommt an und wird gegen \it{O1} transformiert. In diesem Fall ist das Ergebnis der Transformation dasselbe wie vorher.
Die Ausführung von Rories Änderung hat keinen Einfluss auf die von Amilia.
Wird Amilias Änderung auf ''ab'' angewendet, wird das ''x'' an erster Stelle hinzugefügt und das Ergebnis, ''xab'' ist identisch mit dem auf Amilias Seite.\\
%
Würde allerdings die Operation von Amilia zuerst ausgeführt werden, gäbe es ein Problem.
Dann würde der Buchstabe ''b'' gelöscht, da er sich, nachdem das ''x'' eingefügt wurde, an zweiter Stelle steht.
Das Ergebnis nach der ersten Operation wäre ''xac''. Die Dokumentenstatus wären nicht identisch.\\
%
Auf der linken Seite wird die Operation \it{O2} zuerst ausgeführt im Dokument steht ''xabc''.
Dann kommt \it{O1} auf Amilias Gerät an und wird zu \it{O2'} transformiert. Die neue Operation ist nun \tt{delete(1, 3)}, ''lösche ein Zeichen an dritter Stelle''. Der Positionsparameter wurde um eins hochgesetzt, weil das Einfügen des Buchstabens ''x'' durch die konkurrierenden Operation beachtet wurde.
Wird nun \it{O2'} auf der Operation \it{O1} angewendet, wird der korrekte Buchstabe gelöscht und das Ergebnis ist, wie auf Rories Seite ''xab''.
Wird die Operation von Rory, \tt{delete(1, 2)}, gegen die Operation von Amilia transformiert, wird berücksichtigt, dass Amilia ein Zeichen vor der Position zwei, der Poition des von Rory gelöschten Zeichens, eingefügt hat.\\\\
%
%
Zuammenfassend besteht die Grundidee von \gls{OT} darin, eine Operation in eine neue Form umzuwandeln.
Die neue Form ergibt sich aus den Auswirkungen zuvor durchgeführter Operationen.
So können die transformierten Operationen die korrekte Wirkung erzielen und eine Konsistenz der replizierten Dokumente sicherstellen~\cite{ot-later}.
%
% Kritik
%
%False-Tie puzzle? \it{A Generic Operation Transformation Scheme for Consistency Maintenance in Real-time Cooperative Editing Systems} und \it{Achieving convergence, causality-preservation, and intention-preservation in real-time cooperative editing systems}\\\\
%Während der klassische OT-Ansatz, Operationen durch ihre Versätze im Text zu definieren, einfach und natürlich zu sein scheint, werfen real verteilte Systeme ernsthafte Probleme auf. Nämlich, dass sich die Operationen mit endlicher Geschwindigkeit fortpflanzen, die Zustände der TeilnehmerInnen sind oft verschieden, so dass die resultierenden Kombinationen von Zuständen und Operationen extrem schwer vorherzusehen und zu verstehen sind.
%Wie Li und Li es ausdrückten: ``Aufgrund der Notwendigkeit, eine komplizierte Fallabdeckung in Betracht zu ziehen, sind formale Beweise sehr kompliziert und fehleranfällig, selbst für OT-Algorithmen, die nur zwei charakteristische Primitive behandeln (Einfügen und Löschen)``~\cite{ot-critic}.\\
%Damit OT funktioniert, muss jede einzelne Änderung an den Daten erfasst werden: ``Einen Schnappschuss des Zustands zu erhalten, ist normalerweise trivial, aber das Erfassen von Bearbeitungen ist eine ganz andere Sache. [...] Der Reichtum moderner Benutzerschnittstellen kann dies problematisch machen, besonders in einer browserbasierten Umgebung`` ~\cite{diff_sync}.
%
% It is designed to let any number of people collaborate on text at the same time and it can deal with a certain level of network instability, but generally, it requires clients to be connected at all times. If you go and keep editing, your changes can be integrated later, but not indefinitely later.
%In addition, it is designed for text, not for generic objects, so for true offline capabilities of generic data objects, Operational Transforms are less useful. Check them out if you need a solution for mostly connected text.
%
% CRDT
%
%%
% CRDT
%
Ein \gls{CRDT} ist eine spezielle Datenstruktur die in verteilten Systemen auf mehreren Geräten repliziert werden können. Jede Operation wird asynchron an alle Replikate gesendet. Jedes Replikat wendet alle ankommenden Aktualisierungen in variabler Reihenfolge an. Ein Algorithmus löst alle konfliktbehafteten Aktualisierungen auf, wodurch sichergestellt ist, dass Konflikte gar nicht erst auftreten~\cite{crdt_shapiro}. Eine Synchronisation ist nicht notwendig, da die Aktualisierung sofort ausgeführt wird ~\cite{crdt_shapiro2}.\\\\
%
In ihren Arbeiten betrachten Shapiro et. al zwei Replikationsmodelle in einem verteilten System: den zusandsbasierten und den operationsbasierten Ansatz.
Interessanterweise zeigen sie, dass diese beiden Replikationsmodelle und diese beiden Arten von CRDTs äquivalent sind.\\
Bei Operation-basierten \glspl{CRDT} muss sichergestellt werden, dass die Operationen nicht verloren gehen oder dupliziert werden wenn sie zu den anderen Replikaten übertragen werden.
Zustandsbasierte \glspl{CRDT} haben den Nachteil, dass der gesamte Zustand, statt nur der Operation, zu den anderen Replikaten übertragen werden muss.
%
%
\subsub{Zustandbasierter Ansatz}

\begin{figure}[H]
  \centering
  \includegraphics[width=0.8\textwidth]{crdt-state}
  \grayRule
  \caption[Replikation von zustandsbasierten \glspl{CRDT}]{Replikation von zustandsbasierten \glspl{CRDT}, Quelle: ~\cite{crdt_shapiro2}}
  \label{fig:crdt-state}
\end{figure}

Wenn ein Replikat ein Update von einem Client empfängt, aktualisiert es zuerst seinen lokalen Status und dann, einige Zeit später, seinen \b{vollständigen Status}.
So sendet jedes Replikat gelegentlich seinen vollständigen Status an ein anderes Replikat im System.
Um ein Replikat, das den Status eines anderen Replikats empfängt, wendet eine \b{Zusammenführungsfunktion} (merge) an, um den empfangenen Status mit dem lokalen Status zusammenzuführen.
Entsprechend sendet dieses Replikat gelegentlich auch seinen Status an ein anderes Replikat, sodass jedes Update schließlich alle Replikate im System erreicht.
%
%
\subsub{Operationsbasierter Ansatz}
\begin{figure}[H]
  \centering
  \includegraphics[width=0.8\textwidth]{crdt-op}
  \grayRule
  \caption[Replikation von Operationsbasierten \gls{CRDT}]{Replikation von Operationsbasierten \glspl{CRDT}, Quelle: ~\cite{crdt_shapiro2}}
  \label{fig:crdt-op}
\end{figure}

Bei diesem Ansatz sendet ein Replikat seinen vollständigen Status (kann groß sein) nicht an ein anderes Replikat.
Stattdessen sendet es nur den \b{Aktualisierungsvorgang} an \b{alle} anderen Replikate im System und erwartet von ihnen, dass sie das Update auf sich anwenden.\\
Da es sich um einen Sendevorgang handelt, wenn zwei Updates $u1$ und $u2$, bei einem Replikat $i$ angewendet werden und diese Updates an zwei Replikate $r1$ und $r2$ gesendet werden, können diese Updates in unterschiedlicher Reihenfolge bei diesen replikaten ankommen.
$r1$ kann sie in der Reihenfolge $u1, u2$ empfangen, während bei $r2$ die Updates in umgekehrter Reihenfolge ($u2, u1$) ankommen können.
Sind die Aktualisierungen \b{kommutativ}, können die Repliken zussamengeführt werden, egal in welcher Reihenfolge die Updates bei ihnen ankommen - der resultierende Zustand ist derselbe. In diesem Modell wird ein Objekt, für das alle gleichzeitigen Aktualisierungen kommutativ sind, CmRDT (commutative replicated data type - kommutativ replizierter Datentyp) genannt. \\\\
\b{Beispiel:...}\\
CRDTs befassen sich mit einem interessanten und grundlegendem Problem in verteilten Systemen, haben jedoch eine wichtige Einschränkung: "Da ein CRDT konstruktionsbedingt keinen Konsens verwendet, hat der Ansatz starke Einschränkungen; Dennoch sind einige interessante und nicht-triviale CRDTs bekannt" ~\cite{crdt_shapiro2}.
Die Einschränkung ist, dass die CRDT-Adresse nur einen Teil des Problemraums betrifft, da nicht alle möglichen Aktualisierungsoperationen kommutativ sind und daher nicht alle Probleme in CRDTs umgewandelt werden können. Auf der anderen Seite können CRDTs für einige Arten von Anwendungen durchaus nützlich sein, da sie eine nette Abstraktion zur Implementierung repliziter verteilter Systeme bieten und gleichzeitig theoretische Konsistenzgarantien bieten.
%
%Alas, everything in programming is trade-offs, so what do we trade for being able to have conflict-free data structures? Well, they are specialised data structures, like sets and counters, and not generic object representations like JSON, so we’ll have to buy into a whole world of these specialised data structures, and maybe we have hard time mapping our application objects to them. 
%
% Couch Pouch
%
CouchDB verwendet Replikation um Änderungen an Dokumenten zwischen einzelnen Knoten zu synchronisieren.

Dokumente werden versioniert

Inkrementelle Replikation


Aufgabe der Replikation von CouchDB ist die Synchronisation 2+n Datenbanken. Lösungen: Zuverlässige \b{Synchronisation} von Datenbanken auf verschiedenen Geräten. \b{Verteilung} der Daten über ein Cluster von DB-Instanzen die jeweils einen Teil des requests beantworten (Lastverteilung) und \b{Spiegelung} der Daten über geografisch weit verteilte Standorte.\\
Durch die inkrementelle (schrittweise) Arbeitsweise kann CouchDB genau dort weitermachen wo es unterbrochen wurde wenn während der Replikation ein Fehler auftritt, beispielsweise durch eine ausfallende Netzwerkverbindung
\it{Es werden auch nur die Daten übertragen, die notwendig sind, um die Datenbanken zu synchronisieren.}\\
Das Besondere an CouchDB ist, dass es darauf ausgerichtet ist, Fehler/Konflikte vernünftig zu behandeln statt anznehmen es träten keine auf (vgl. \cite{couchDB} S. 7f). Wie \hyperref[chap:conflict]{oben} beschrieben, gibt es in Verteilten Systemen einige Fehler die auftreten können.\\\\
\it{Das CouchDB Replikationsmodell erlaubt eine nahtlose, peer-to-peer (direkte) Datensynchronisation zwischen beliebig vielen Geräten. Das CouchDB Replikationsprotokoll ist in CouchDB selbst implementiert, das die Serverkomponente abdeckt. Dann gibt es das PouchDB-Projekt, das dasselbe Protokoll in JavaScript implementiert, das auf Browser- und Node.js-Anwendungen abzielt. das deckt Ihre Kunden und dev-Server ab. Schließlich gibt es Couchbase Mobile und Cloudant Sync, die auf iOS und Android laufen und das CouchDB Synchronisationsprotokoll in Objective-C bzw. Java implementieren.}\\
content addressable versions: Idee: Nimm den Objektinhalt (content) und jag ihn durch eine \gls{Hashfunktion}\\
%
Diese Art von Konflikten sollten von Menschen gelöst werden. Nur so kann sichergestellt werden, dass die korrekte Änderung gespeichert wird und keine Daten verloren gehen.
%
% \subsub{schlussendliche Konsistenz}
\subsub{Eventual Consistency}
Das \gls{CAP} Theorem, veranschaulicht in \autoref{fig:cap}, besagt, dass jedes System mit dem Daten über das Netzwerk gesendet werden, nur zwei von den drei möglichen Eigenschaften, Konsistenz, Verfügbarkeit und Partitionstoleranz, garantieren kann.
Konsitzenz der gespeicherten Daten bedeutet, es muss sichergestellt werden dass nach Abschluss der Transaktion auch alle Replikate des manipulierten Datensatzes aktualisiert werden. Der Datensatz ist in jeder Datenbank identisch.
Das System ist besitzt eine hohe Verfügbarkeit wenn alle Anfragen an das System stets beantwortet werden. Die Verfügbarkeit ist gering, wenn die Antwortzeiten des Systems lang sind.
Partitionstoleranz ist gleichzusetzen mit Ausfalltoleranz. Die Datenbank kann auf mehreren Servern verteilt sein. Trotzdem ein Server oder eine Partition ausfällt, kann das System weiterhin funktionieren.
\begin{figure}[H]
  \centering
  \includegraphics[width=0.6\textwidth]{cap}
  \grayRule
  \caption{Das CAP Theorem}
  \label{fig:cap}
\end{figure}
%
 Eventual Consistency ist eine abgeschwächte Variante der Konsistenz, die häufig bei verteilten Datenbanken zur Anwendung kommt. Dabei verzichtet man aus Performancegründen bei Schreiboperationen darauf, Daten sofort auf alle Server/Partitionen zu verteilen.
%
%
Stattdessen kommen Algorithmen zum Einsatz, die sicherstellen, dass nach Beendigung der Schreiboperationen die Daten konsistent gemacht werden, in der Regel ohne Aussage darüber, in welchem Zeitraum der Vorgang abgeschlossen sein wird. In der Zwischenzeit sind unterschiedliche Datenbestände auf den einzelnen Servern. Das kann dazu führen, dass identische, zeitgleiche Abfragen von mehreren Benutzern unterschiedliche Ergebnisse liefern können. Man kann lediglich darauf vertrauen, dass die Daten letztendlich konsistent sind, daher der Name diese Konzeptes.

(vgl. ~\cite{couchDB} S. 11 ff.)
\subsub{Lokale Konsistenz}
\subsub{Verteilte Konsistenz}

\subsub{Replikation?}
\subsub{Konfliktmanagement}

