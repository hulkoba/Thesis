\chapter{\label{chap:grundlagen}Grundlagen}
Was bedeutet offlinefähig?\\
\note{Harte, weiche, mittlere probleme (verschiedene Stufen von offlinefähig)}\\
Native \Glspl{App} existieren und funktionieren grundsätzlich solange offline, bis sie versuchen online Daten abzurufen.\\
bei nativen Apps ist das Problem bei der Datenverteilung
%
% Offline-First
%
\section{Offline First}
%
% Anforderungen
%
% \sub{Anforderungen an Offline-First Anwendungen}
% Eine Webanwendung besteht grundsätzlich aus zwei Bestandteilen: \it{\gls{Assets}} und benutzerInnengenerierte Daten.\\
% Um offline nutzbar zu sein, müssen einige Voraussetzungen erfüllt werden.
% \begin{itemize}
%   \item müssen auf dem Endgerät gespeichert werden
%   \item Serverseitige (nicht lokal) Datenbank \&\& Synchronisation zwischen Server ind Client
%   \item Kollaborativ \&\& Sync zwischen allen Beteiligten
%   \item Synchronisation erfordert den Umgang mit Konflikten
%   \item Kein Datenverlust
% \end{itemize}
% \todo{merge mit PWA?}
Offline First heißt, die Bestandteile einer Anwendung so zu verwalten, dass nach der ersten Verwendung keine Internetverbindung mehr notwendig ist um deren grundlegenden Funktionen zu nutzen.\todo{Quelle}\\
Eine Anwendung, die für den Offline-Gebrauch entwickelt wurde, ist sowohl mit, als auch ohne Internetverbindung vollständig einsatzbereit. Bei einer bestehenden Internetverbindung ist das Laden der Ressourcen aus dem Cache schneller als aus dem Netz. Daten, die zuerst lokal gespeichert werden, gehen auch bei plötzlichen Verbindungsverlust nicht verloren.
%
% Cache ServiceWorker AppCache
%
\sub{Lokale Speicherung im Cache}
Das Cachen der \gls{Assets} ist der erste Schritt um Daten offline verfügbar zu machen. Browser haben die Möglichkeit, diese Dateien in ihrem Cache zu speichern. Der Browser-Cache ist leider nicht persistent und kann ohne Vorwarnung Daten löschen, wenn das Limit des globalen Cache (Disk Cache) erreicht ist.
\subsub{Appcache}
Um mehr Kontrolle darüber zu bekommen, was wann und für wie lange gespeichert werden soll, wurde der Application Cache (AppCache) zur HTML-Spezifikation hinzugefügt.
Im Juni 2016\footnote{siehe \url{https://github.com/w3c/html/pull/444/commits}} wurde der AppCache wieder aus den Web-Standards entfernt, und wird nicht mehr empfohlen.
In der Theorie stellte sich der Application Cache als einfach anzuwenden und unproblematisch dar. Um eine webbasierte Anwendung offline auszuliefern wurde benötigte es eine Textdatei -- der \tt{cache manifest}-Datei -- mit der Endung \tt{.appcache}. Dort wurden alle Ressourcen aufgelistet, welche der Browser cachen sollte.
Die Datei wurde dann über das \tt{manifest}-Attribut in die \gls{HTML}-Dateien der Webanwendung eingebunden werden.
\lstset{language=HTML,caption={Beispiel einer \gls{HTML}-Datei mit einer Manifest-Attribut Einbindung},label={code:appcache_html}}
\begin{lstlisting}
<!DOCTYPE html>
<html manifest="example.appcache">
  <head>
    <title>Example Application Cache</title>
    <link rel="stylesheet" href="style.css">
    <script src="index.js"></script>
  </head>
  <body>
    ...
  </body>
</html>
\end{lstlisting}
Die über das \tt{manifest}-Attribut eingebundene Cache-Datei kann folgendermaßen aussehen:
\lstset{language=python,caption={Beispiel einer \normalfont{\tt{.appcache}}\itshape{-Datei}},label={code:appcache}}
\begin{lstlisting}
CACHE MANIFEST
# version comment for triggering updates
# v1
style.css
index.js
assets/cat.png
\end{lstlisting}
Alle Seiten, mit dem manifest-Attribut und die, die explizit in der Textdatei beschrieben wurden, wurden vom Browser gespeichert~\cite{appcache}.\\
In der Praxis jedoch zeigten sich zahlreiche Probleme mit dem AppCache. So wurde der Application Cache nur aktualisiert wenn sich der Inhalt der des Manifests geändert hat. Dann mussten alle Dateien neu heruntergeladen werden.
Wurden das Manifest und eine andere Datei geändert, wurden die geänderten Dateien nicht unbedingt erneut gespeichert. Denn wenn der Server zusammen mit den Dateien keine Cache-Header sendete, so speicherte der Browser die Datei nach einem Cache-Header-Wert den er `errät`. So konnte es passieren, dass der Browser annahm, eine Datei brauche keine Aktualisierung und weiterhin die alte, gecachte Version auslieferte~\cite{noappcache}.\\
Als Reaktion auf diese Probleme wurde der Service Worker entworfen.
%
% Service Worker
%
\subsub{Service Worker}
Ein Service Worker ist ein Skript, das zwischen Netzwerk und Browser sitzt und von Letzterem im Hintergrund ausgeführt wird. Die Kernfunktion des Service Workers ist es, Netzwerkanfragen abzufangen um sie zu verarbeiten und im Cache zu verwalten~\cite{serviceworker}.\\
Gegenwärtig besitzen -- bis auf den Internet Explorer -- sämtliche Desktop-Browser, und alle gängigen mobilen Browser eine Unterstützung für Service Worker.
\begin{figure}[H]
	\centering
	\includegraphics[width=1\textwidth]{ServiceWorker_all}
	\grayRule
	\caption{Browserkompatibilität für Service Worker, Quelle: ~\cite{caniuse-sw}}
	\label{fig:serviceworker}
\end{figure}
Mit dem Service Worker können wie mit dem App Cache statische Ressourcen sofort beim ersten Besuch der Seite im Cache gespeichert werden. Es lässt sich hierbei unterscheiden, ob die Daten vor der ersten Verwendung, oder später im Cache gespeichert werden sollen. Für den ersten Fall eignen sich statische Inhalte wie Schriften oder JS-Dateien, für den zweiten größere Ressourcen die nicht sofort benötigt werden.\\
Zusätzlich bietet der Service Worker die Möglichkeit auf Interaktionen zu reagieren. Den NutzerInnen kann angeboten werden bestimmt Inhalte der Seite, wie zum Beispiel ein Video, später, bzw. offline anzuschauen. Diese werden dann im Cache gespeichert und sind somit offline verfügbar.
Service Worker erlauben außerdem den Zugriff auf Push-Benachrichtigungen und das Background Sync \gls{API}. Die Hintergrundsynchronisation kann einmalig oder in festgelegten Intervallen stattfinden und ist besonders für nicht dringende Aktualisierungen wertvoll~\cite{offline_cookbook}.
%
% Browser
%
\sub{Lokale Speicherung im Browser}
Der zweite Schritt Daten offline verfügbar zu machen ist sie im Browser zu speichern. Einige Konzepte hierfür werden im Folgenden erläutert.
%
\subsub{Web Storage}
Das Web Storage \gls{API} ist ein Web-Standard mit dessen Hilfe Daten als Schlüssel / Wert Paare im Browser gespeichert werden können. Es wird, wie Abbildung \ref{fig:webStorage} zeigt, von allen relevanten Browsern unterstützt.
Web Storage umfasst zwei Mechanismen. Den Session Storage und den Local Storage.\\
Der Session Storage existiert nur so lange der Browser geöffnet ist.
Das heißt alle Daten die im Session Storage gespeichert werden, existieren nicht mehr sobald der Browser geschlossen wird. Daten die im Local Storage gespeichert sind, existieren dort solange bis der Browser Cache geleert wird~\cite{webstorage}.
\begin{figure}[H]
	\centering
	\includegraphics[width=\textwidth]{webStorage}
	\grayRule
	\caption{Browserkompatibilität für Web Storage, Quelle: ~\cite{caniuse-ws}}
	\label{fig:webStorage}
\end{figure}
Der von den Browsern freigegebene Cache-Space für Web Storage variiert, ist aber meist auf 10 MB begrenzt. Der größte Nachteil ist wohl, dass Web Storage synchron arbeitet und so andere Operationen, wie zum Beispiel das Rendern der Seite, blockieren kann~\cite{webstorage-con}.
%
%
\subsub{WebSQL}
Eine andere der lokalen Speicherung im Browser ist die Web SQL Datenbank.
Sie hat ein asynchrones \gls{API} und unterstützt die grundlegenden SQL Abfragen. Web SQL sollte in den W3C Standards aufgenommen werden. Aus Mangel unabhängigen Implementierungen wie eine andere \gls{DB} als SQLite im Backend wurde es abgelehnt~\cite{websql}.\\
Das Web SQL \gls{API} wird nur von den Webkit--Browsern unterstützt. Also nicht von Firefox, dem Internet Explorer oder dessen neueren Variante Edge~\cite{caniuse-websql}.
%
%
\subsub{IndexedDB}
\begin{figure}[H]
	\centering
	\includegraphics[width=\textwidth]{indexedDB}
	\grayRule
	\caption{Browserkompatibilität für IndexedDB, Quelle: ~\cite{caniuse-idb}}
	\label{fig:indexedDB}
\end{figure}
%
\subsub{IndexedDB 2}
\begin{figure}[H]
	\centering
	\includegraphics[width=\textwidth]{indexedDB2}
	\grayRule
	\caption{Browserkompatibilität für IndexedDB 2.0, Quelle: ~\cite{caniuse-idb}}
	\label{fig:indexedDB2}
\end{figure}
https://www.w3.org/TR/IndexedDB-2/
%
% Sync
%
\sub{Datenbanksynchronisation}
%
% Couch
%
\subsub{\label{sec:couch}CouchDB}
Apache CouchDB\tm ist ein \gls{DBMS} das seit 2005 als freie Software entwickelt wird. Die dokumentenorientierte \gls{DB} funktioniert sowohl als einzelne Instanz, als auch im Cluster, in dem ein Datenbanksserver auf einer beliebig großen Anzahl an Servern oder \glspl{VM} ausgeführt werden kann. So kann die Datenschicht beliebig skaliert werden, um die Anforderungen vieler BenutzerInnen zu erfüllen. CouchDB verwendet das \gls{HTTP}--Protokoll und \gls{JSON} als Datenformat, weswegen es mit jeder Webfähigen Anwendung kompatibel ist. CouchDB wird über ein \gls{REST}ful \gls{HTTP} \gls{API} angesprochen. Mit den für \gls{REST}ful Services standardisierten Methoden z. B. GET, POST, PUT, DELETE können die Daten abgerufen und. manipuliert werden.\\
Das implementierte Replikationsmodell erlaubt die Synchronisation bzw. bidirektionale Replikation zu verschiedenen Geräten ist genau die Besonderheit, die CouchDB als eine Offline--Datenbank auszeichnet. Dessen Funktionsweise wird in \autoref{sec:replication} detailliert beschrieben. Dieses Protokoll ist die Grundlage für Offline First Anwendungen.
Das Replikations-API von CouchDB bietet die Möglichkeit, eine Datenbank kontinuierlich oder selbstgesteuert mit einer anderen zu synchronisieren.
So kann beispielsweise eine CouchDB-Instanz auf dem Mobiltelefon und eine auf dem Laptop bestehen und beide können sich bei bestehender Internetverbindung synchronisieren. Da so die gespeicherten Daten aus dem lokalen Speicher gelesen werden, sind ein schnelles Interface und eine geringe Latenz die positive Folge. Wenn Konflikte auftreten, beispielsweise durch gleichzeitiges Bearbeiten eines Dokuments von zwei Personen ohne Netzwerkverbindung, werden diese als solche markiert, jedoch nicht von selbst aufgelöst. So gehen keine Daten verloren und es liegt an der benutzenden Person diese zu lösen. \todo{Referenz git?}\\
CouchDB ist für Server konzipiert. Für Browser gibt es \hyperref[sub:pouch]{PouchDB} und für native iOS- und Android--\glspl{App} wurde Couchbase Lite entwickelt. Alle können Daten miteinander replizieren und verwenden das CouchDB Replikationsprotokoll~\cite{couch}.
\todo{CouchDB Replikationsmodell hier?}
%
% Pouch
%
\subsub{\label{sub:pouch}PouchDB}
\todo{Was benutzt Poch wann? -- IndexedDB, Web SQL, LevelDB}
Als Ergänzung zu CouchDB kann PouchDB verwendet werden. PouchDB ist eine Open--Source--JavaScript--Datenbank, die so konzipiert wurde, dass sie im Browser läuft. PouchDB ermöglicht es Anwendungen zu erstellen, die sowohl offline als auch online funktionieren. Daten können lokal gespeichert werden, sodass alle Funktionen der Anwendung auch im Offline--Modus zur Verfügung stehen.
Daten werden unabhängig von der nächsten Anmeldung (des nächsten Onlinezugangs) zwischen \b{Clients}, CouchDB oder kompatiblen Servern synchronisiert.
PouchDB läuft auch in Node.js\footnote{JavaScript Laufzeitumgebung, steht unter \url{https://nodejs.org/en/download/} zum Download bereit} und kann als direkte Schnittstelle zu CouchDB--kompatiblen Servern verwendet werden~\cite{pouch}.

%
% PWA
%
\sub{\label{sub:pwa}Progressive Web Apps \todo{raus?}}
\Gls{PWA} ist eine Bezeichnung für eine mobil nutzbare Webseite, die eine Brücke zwischen der nativen
Applikation und einer Webseite schlägt.
Der Begriff \gls{PWA} wurde im Jahr 2015  von Alex Russel und seiner Frau Frances Berriman geprägt.
Dieser beschreibt Webseiten, die die positiven Funktionen von nativen Applikationen mitbringen, aber statt über App Stores installiert zu werden, im Webbrowser existieren. Die Webseiteninhalte sind ohne die Installation sofort und jederzeit für die NutzerInnen abrufbar. Schon beim zweiten Besuch der Webseite ist die Ladezeit der Daten verkürzt und sie ist offline, oder auch bei schlechter Internetverbindung nutzbar. Nach mehrmaligem Aufruf kann die \gls{PWA} über den Browser installiert und zum Startbildschirm hinzugefügt werden. Russel und Berriman legen folgende Einenschaften einer \gls{PWA} fest:
\begin{description}[leftmargin=0.5cm,style=nextline]
	\item[Responsive] to fit any form factor
	\item[Connectivity independent] Progressively-enhanced with Service Workers to let them work offline
	\item[App-like-interactions] Adopt a Shell + Content application model to create appy navigations \& interactions
	\item[Fresh] Transparently always up-to-date thanks to the Service Worker update process
	\item[Safe] Served via TLS (a Service Worker requirement) to prevent snooping
	\item[Discoverable] Are identifiable as “applications” thanks to W3C Manifests and Service Worker registration scope a llowing search engines to find them
	\item[Re-engageable] Can access the re-engagement UIs of the OS; e.g. Push Notifications
	\item[Installable] to the home screen through browser-provided prompts, allowing users to “keep” apps they find most useful without the hassle of an app store
	\item[Linkable] meaning they’re zero-friction, zero-install, and easy to share. The social power of URLs matters.
\end{description}
\todo{Näher erläutern?}~\cite{pwa}.

%
% Konflikte [Jan]
%
% 1. Definition
Als Konflikt wird die Situation beschrieben in der es ein Dokument mit unterschiedlichen Informationen in mehreren Geräten oder Datenbanken gespeichert ist (vgl. ~\cite{couchDB} S. 153).
Konflikte gehören in verteilten Systemen zur Realität und lassen sich nicht vermeiden.
Ein verteiltes System ist per Definition ein Zusammenschluss unabhängiger Computer, die sich für die NutzerInnen als ein einziges System präsentieren (vgl. ~\cite{tanenbaum} S. 2).
Einfacher gesagt besteht ein verteiltes System, sobald zwei oder mehrere Computer über das Netzwerk miteinander verbunden sind.
Die spezielle Eigenschaft von Netzwerken ist jedoch, dass die Verbindung jederzeit abbrechen kann.
Gareth Wilson beschreibt in seinem Artikel die acht Irrtümer der verteilten Datenverarbeitung ~\cite{fallacies}:
\begin{enumerate}
  \item Das Netzwerk ist zuverlässig
  \item Die \gls{Latenz}zeit ist gleich null
  \item Die Bandbreite ist unendlich
  \item Das Netzwerk ist (informations)sicher
  \item Die Netzwerkstruktur wird sich nicht ändern
  \item Es gibt eineN AdministratorIn
  \item Die Datentransportkosten sind gleich null
  \item Das Netzwerk ist homogen
\end{enumerate}
Aus diesen irrtümlichen Annahmen über das Netzwerk ergeben sich Fehlerszenarien. So kann es passieren, dass der zweite Computer entweder sehr weit entfernt, sehr beschäftigt oder ausgeschaltet ist. Diese Fehlerszenarien können dazu führen, dass ein Konflikt entsteht.\\
Anhand des folgenden Beispiels wird ein mögliches Fehlerszenario für den Fall des unzuverlässigen Netzwerks aufgegriffen.
Zwei Personen treffen sich im Zug und verstehen sich auf Anhieb sehr gut. Person A, nennen wir sie Amilia, gibt Person B, nennen wir sie Rory, ihre Telefonnummer. Der Zug fährt durch einen Tunnel und das Netzwerk bricht ab als Rory sie in ihr Adressbuch, das in Form einer \gls{App} auf ihrem Laptop gespeichert ist, schreibt.
Amilia dikiert ihre Telefonnummer falsch, mit einem Zahlendreher, weil sie die Nummer noch nicht lange hat.
Zur Sicherheit schickt Amilia Rory ihre Nummer zusätzlich per E-Mail. Rory schaltet ihren Laptop aus, weil er sich mit Amilia unterhalten möchte.
Am Abend ist Rory zu Hause angekommen und sie speichert Amilias Nummer aus der E-Mail in ihrem Adressbuch auf seinem stationären Desktop PC.
Jetzt gibt es Amilias Telefonnummer mit unterschiedlichen Informationen in Rorys Adressbuch auf verschiednene Geräten.
Wenn Rory am nächsten Tag seinen Laptop startet und das Adressbuch sich mit dem auf dem PC synchronisiert entsteht ein Konflikt. Die falsche Telefonnummer wird gespeichert und die richtige ist verloren.\\\\
%
%
Konflikte sind in zwei Kategorien einzuteilen. Es gibt solche, die vom System gelöst werden können und welche die eine spezielle Behandlung brauchen.
Gibt es eine gleichzeitige Änderung an unterschiedlichen Stellen eine Dokuments muss das kein Problem darstellen.
Das Dokument kann beide Aktualisierungen erhalten indem die Änderungen zusammengefügt werden.
Diese Prozedur wird \it{merge} genannt und wird unter anderem von Git\footnote{git steht unter \url{https://git-scm.com/downloads} zum Download bereit}, einer Software zur verteilten Versionsverwaltung, angewandt. Diese Art Konflikt kann selbstständig vom System gelöst werden.\\

Die Konflikte die durch die gleiche Änderung an ein und derselben Stelle des Dokuments entstehen, benötigen eine aufwändigere Behandlung.
Es muss festgestellt werden welche Version die korrekte ist und gespeichert werden soll.
Zur Lösung dieses Problems wurden verschiedene Managementstrategien entworfen die im Folgenden vorgestellt werden.
%
%
% \begin{description}[leftmargin=0.5cm,style=nextline]
% 	\item[1. Das Netzwerk ist zuverlässig] ~ Der Strom kann ausfallen oder Glasfaserkabel können kaputt sein --- Das Netzwerk ist nicht zuverlässig.
% 	\item[2. Die \gls{Latenz} ist gleich null] ~ Glasfaserkabel werden durch Mikrowellen (oder andere Technologien) ersetzt um Millisekunden an Zeit zu sparen. Das würde nicht passieren, wäre die \gls{Latenz} bei null. Es dauert nun mal eine gewisse Zeit(ms) wenn ein Signal eine (geografisch)weite Strecke zurücklegen muss --- Die Latenz ist nicht gleich null.
% 	\item[3. Die \gls{Bandbreite} ist unendlich] ~ Daten können nicht schneller fließen als die Komponenten die sie verarbeiten (\gls{Middleware}, Datenbank \ldots) --- Die Bandbreite ist nicht unendlich.
% 	\item[4. Das Netzwerk ist sicher] ~ Der \sc{Heartbeat-bug}\footnote{\url{http://heartbleed.com/} -- Zugriff: 07.04.2018}, der im Jahr 2014 behoben wurde und die Sicherheitslücke im ICE-\gls{WLAN} im Jahr 2016\footnote{\url{https://netzpolitik.org/2016/datenschutz-im-zug-deutsche-bahn-will-sicherheitsluecke-in-neuem-ice-wlan-schliessen/} -- Zugriff: 07.04.2018} sind nur zwei Beispiele die zeigen, dass das Netzwerk nicht sicher ist.
% 	\item[5. Die Netzwerkstruktur wird sich nicht ändern] ~ Eine Datenbank kann beispielsweise über mehrere Server verteilt sein, die (teilweise) voneinander abhängig sind. Ein Server mit Abhängigkeiten kann ausfallen, es kann eine Aktualisierung für einen anderen Server geben --- die Struktur ändert sich.
% 	\item[5. Die Netzwerkstruktur wird sich nicht ändern] ~ Eine Datenbank kann beispielsweise über mehrere Server verteilt sein, die (teilweise) voneinander abhängig sind. Ein Server mit Abhängigkeiten kann ausfallen, es kann eine Aktualisierung für einen anderen Server geben --- die Struktur ändert sich.
% 	\item[6. Es gibt eineN AdministratorIn] ~ Es kann beliebig viele AdministratorInnen geben.
% 	\item[7. Die Datentransportkosten sind gleich null] ~ Netflix bezahlte anfang 2014 diversen InternetanbieterInnen dafür, dass Netflix KundInnen bevorzugten Internetzugang haben.
% 	\item[8. Das Netzwerk ist homogen] ~ Es gibt verschiedene Arten von Netzwer: 3G, 4G, LTE, WiFi. Wird beeinflusst durch Hardware (Smartphone, Tablet, PC, Laptop, Router \ldots)~\cite{fallacies}
% \end{description}
