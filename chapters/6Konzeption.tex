\chapter{\label{chap:konzeption}Konzeption}
Die erarbeiteten Anforderungen zur Untersuchung der Konfliktmanagementstrategien offlinefähiger Systeme werden in diesem Kapitel für die Konzeption angewendet.\\
Es soll für jede zu untersuchende Technologie ein Prototyp entwickelt werden. Im Rahmen dieser Arbeit entsteht ein Prototyp der \sc{Redux Offline} verwendet und ein zweiter in dem PouchDB und CouchDB eingesetzt wird. Für letzteren könnte genauso gut HOODIE benutzt werden, da HOODIE sowohl PouchDB als auch CouchDB benutzt~\cite{hoodie-how}.
Doch da für den zu entwickelnden Prototyp lediglich diese beiden Komponenten benötigt werden, wurde sich dagegen entschieden.
Bis zu einem gewissen Status, nämlich dem der Verwendung der Technologien, sind beide Prototypen -- bis auf den Namen-- identisch.
% Beginnend mit dem Aufbau der exemplarischen Anwendung werden in den folgenden Abschnitten die  der Anwendungsaufbau, Architektur  und schließlich \todo{blabla} die graphische Oberfläche aufgeführt.
% Den Raum aller möglichen Lösungen anhand der Anforderungen auf die in irgendeinem Sinn beste / geeignetste einzuschränken:\\
% App entwickeln, in der es einen `Verbindung unterbrechen` Knopf gibt ', oder ob es aus irgendwelchen Erwägungen notwendig sein könnte, das über eine separate Instanz zu machen.
%
% Aufbau
%
\section{Anwendungsaufbau}
Die Prototypen bestehen im Frontend aus React und wurden mit \sc{Create React App} erstellt. \sc{Create React App} erstellt ein Projekt mit dem gewünschten Namen, generiert eine initiale Projektstruktur (vgl. \autoref{fig:init}) und installiert die dafür benötigten Abhängigkeiten~\cite{create-react}.
\begin{figure}[H]
  \centering
  \begin{subfigure}[t]{0.4\textwidth}
          \includegraphics[width=\textwidth]{Ordnerstruktur}
          \caption{Die initiale Projektstruktur}
          \label{fig:init}
  \end{subfigure}
  ~ 
  \begin{subfigure}[t]{0.4\textwidth}
          \includegraphics[width=\textwidth]{rca-package}
          \caption{Die initiale package.json Datei}
          \label{fig:init2}
  \end{subfigure}
  \grayRule
  \caption[Create React App: initiale Testapplikation]{einer mit Create React App erstellten Testapplikation}
  \label{fig:create-react-app}
\end{figure}
%
Diese sind im Verzeichnis node\_modules installiert.
Außerdem ist ein ServiceWorker
%und ein App Manifest (\tt{manifest.json}) enthalten, wodurch die \gls{PWA}-- Kriterien erfüllt sind.
enthalten der die \gls{Assets} cacht.
Als Template gibt es nun die \tt{public/index.html}-- Datei. In der \tt{index.js}--Datei werden die React--Komponenten und der ServiceWorker initialisiert. 
Alle \tt{App.*}--Dateien umfassen eine minimale Beispielanwendung.
In der generierten \tt{package.json}--Datei (vgl. \autoref{fig:init2}), befinden sich Informationen über die Anwendung und ihre Abhängigkeiten. Im Unterpunkt \tt{scripts} werden Kommandozeilen-Aufrufe definiert und können mit dem Befehl \tt{npm run} aufgerufen werden.
%
% React Komponenten
%
\sub{Aufbau der React Komponenten}
React ist eine Open-Source Bibliothek, die dazu dient, die View-Komponente des Model-View-Controller-Ansatzes abzudecken, also die Seite der Anwendung die für die Anzeige und Interaktion zuständig ist. Ein Vorteil von React sind die Komponentenbasierte Philosopie. Eine Komponente ermöglicht die Aufteilung der \gls{UI} in kleine Teile und ist eine abstrakte Basisklasse. Einmal implementiert, lässt sich eine Komponente immer wieder verwenden~\cite{react}.\\
Eine Komponente kann einen internen \tt{state} besitzen, oder die Daten aus den \tt{props} nehmen.
\tt{Props} sind Daten die von der Elternkomponente übergeben werden und können auch nur von dieser manipuliert werden.
Eine React Komponente hat immer eine \tt{render()}--Funktion die die Daten aus dem \tt{state} oder den \tt{props} liest und zurückgibt was dargestellt werden soll.
Hier wird das zur Komponente gehörende \gls{HTML} erzeugt. Jede Änderung des \tt{states} führt einen erneuten Aufruf der \tt{render()}--Funktion mit sich.\\\\
React Komponenten können in zwei Kategorien aufgeteilt werden. Die eine Kategorie ist die Containerkomponente. Sie dienen als Datenquelle und in ihnen steckt die Logik wie etwas funktioniert. Sie stellen außerdem Callbackfunktionen für die Viewkomponenten bereit.
Viewkomponenten haben keine andere Zuständigkeit als die Daten, die sie über ihre \tt{props} erhalten, anzuzeigen und ggf. die ebenfalls empfangenden Callbackfunktionen aufzurufen ~\cite{react-components}.\\
Zur Veranschaulichung wird anhand des Listings \ref{code:react-form} eine beispielhafte React Containerkomponente, und im \autoref{code:react-form-view} die dazugehörige Viewkomponente beschrieben.
Zusammen repräsentieren sie ein Formular in dem der Name des Kontakts angezeigt und geändert werden kann.
%
\begin{center}
  \lstinputlisting[language=REACT,
  numbers=left,xleftmargin=20pt,framexleftmargin=15pt,
  firstline=1, lastline=24,
  caption={Eine React Containerkomponente},
  label=code:react-form]{code/Form.js}
\end{center}
%
Die Formular--Containerkomponente hat ein Kontaktobjekt im internen \tt{state} gespeichert.
Auf dieses Objekt haben andere Komponenten keinen Zugriff und es ist nur via \tt{setState()} änderbar.
Initial wird das Kontaktobjekt über die \tt{props} in Zeile 5 geladen. So kann das Vorausfüllen der Eingabefelder realisiert werden.\\
In neun sieben steht die \tt{handleChange} Funktion, die den internen \tt{state} in Zeile 12 mit den reingegebenen Werten aktualisiert.
Die Viewkomponente wird wie in in den Zeilen 17 bis 19 aufgerufen. Ihr wird der interne State übergeben, also das Kontaktobjekt und die \tt{handleChange} Funktion. 
% listing
\begin{center}
  \lstinputlisting[language=REACT,
  numbers=left,xleftmargin=20pt,framexleftmargin=15pt,
  firstline=25, lastline=33,
  caption={Eine React Viewkomponente},
  label=code:react-form-view]{code/Form.js}
\end{center}
%
Die beiden Parameter sind in der Zeile 1 des folgenden Listings wiederzufinden.
Die Viewkomponente macht nichts anderes als die ihr übergebenen Daten anzuzeigen. Im Ereignisbehandler des Eingabefeldes wird die übergebene \tt{handleChange} Funktion, in der Zeile 5, aufgerufen.
Diese Komponente besitzt keinerlei Logik.
%
% Redux
%
\sub{Verwendung von Redux Offline}
Redux Offline kann nur zusammen mit Redux verwendet werden. Deswegen ist für diesen Prototypen die Implementierung von Redux vorausgesetzt.
Redux ist eine JavaScript Bibliothek die Probleme im Zusammenhang mit dem \it{Zustand} einer Anwendung löst.
Es gibt einen zentralen Ort, in dem der \it{Zustand} der App gespeichert ist, auf den von jeder Komponente aus zugegriffen werden kann.
Dieser Ort wird \sc{Store} genannt. Alle Änderungen der Daten im zentralen Speicher erfolgen ausschließlich über Aktionen.\\\\
%
Redux Offline ist eine erweiternde Bibliothek für Redux dessen Funktionsweise in Abschnitt \ref{sub:reduxoffline} detailliert beschrieben wird.\\
Nach der Installation muss der Redux \tt{store} zusammen mit dem \tt{offline "-store "-enhancer} erzeugt werden. Listing \ref{code:store} visualisiert diesen Vorgang. Ein Redux \tt{store} wird mit dem \tt{storeCreator} in Zeile 5 erzeugt. Ein \tt{store "-enhancer} ist eine Funktion die den \tt{storeCreator} neu zusammenfügt und einen neuen, erweiterten \tt{storeCreator} zurückgibt.
Redux Offline kommt mit einer Grundkonfiguration (siehe Zeile 3). Diese wird dem \tt{offline store enhancer} in Zeile 8 übergeben.
\begin{center}
  \lstinputlisting[language=REACT,
  firstline=50,lastline=62,
  numbers=left,xleftmargin=20pt,framexleftmargin=15pt,
  caption={Erstellen eines Stores mit Redux Offline},
  label=code:store]{code/Redux.js}
\end{center}
Der gesamte Kontext, der zum Synchronisieren einer Aktion erforderlich ist in einem zusätzlichen Metaattribut gespeichert.
Damit die Anwendung weiß wie die Aktionen verarbeitet werden sollen wird sie mit dem Metafeld dekoriert. Die Aktion zum Lesen der Kontakte könnte dann wie im folgenden Listing aussehen.
%
\begin{center}
  \lstinputlisting[language=REACT,
  firstline=9,lastline=20,
  numbers=left,xleftmargin=20pt,framexleftmargin=15pt,
  caption={Aktion \tt{fetchContacts} mit Metaattribut},
  label=code:react-meta]{code/Redux.js}
\end{center}
%
Das erste \tt{meta.offline} Feld beschreibt die Netzwerkaktion die ausgeführt werden soll, also den Aufruf an die angegebene URL in Zeile 6.
Bei \tt{commit} in Zeile 7 wird festgelegt welche Aktion bei erfolgreichem Netzwerkaufruf augeführt werden soll.
Für den Fall, dass von dem angefragtem \gls{API} ein 4xx oder 5xx \gls{HTTP} Statuscode zurückkommt wird die im \tt{rollback} definierte Aktion gefeuert ~\cite{redux-offline-npm}.\\
Die Aktionen beschreiben nur was passiert. Wie der Status sich ändert, wird im \tt{Reducer} beschrieben.
Das Listing \ref{code:reducer} illustriert wie das im entsprechendem Prototypen umgesetzt werden könnte.
%
\begin{center}
  \lstinputlisting[language=REACT,
  firstline=36,lastline=50,
  numbers=left,xleftmargin=20pt,framexleftmargin=15pt,
  caption={Reducer mit allen Aktionen die im Meta Feld beschrieben werden},
  label=code:reducer]{code/Redux.js}
\end{center}
%
In diesem Beispiel wird der \gls{App}status nur bei erfolgreichem Netzwerkaufruf aktualisiert. Das ist in den Zeilen 6 bis 8 nachzulesen. Und auch nur dann, wenn sich die Antwort vom Server von diesem unterscheidet.
%
%
% Architektur
%
\section{WIP Architektur}
Die zu erstellenden Prototypen erhalten die Namen \it{amilia-qouch} und \it{amilia-rdx}, wobei Amilia der Name ist, der sich in den Beispielkontakten in den Szenarien wiederfindet. Die Abkürzung \it{rdx} steht für Redux und zeigt, dass dieser Prototyp \sc{Redux Offline} verwendet. Die Endung \tt{qouch} soll die Symbiose von CouchDB und PouchDB darstellen. Der Buchstabe Q klingt wie das hart ausgesprochene C in Couch und wenn man das kleine Q horizontal spiegelt, sieht man das P für Pouch.\\\\
Beide Prototypen setzen sich aus den nachfolgend beschriebenen Komponenten zusammen, welche in \autoref{fig:uml} veranschaulicht werden. Die Abbildung stellt ein Komponentendiagramm dar. Es handelt sich hierbei nicht um das UML Komponentendiagramm, sondern um ein eigens entworfenes. Die Bezeichnung ist durch die Darstellung von React Komponenten begründet.\\\\
Jeder Kasten repräsentiert eine Komponente, deren Bezeichnung im Kopf steht.
Alle Komponenten auf der linken Seite können den Appstatus nicht manipulieren. Sie können nur die von der Elternkomponente durchgereichten Funktionen aufrufen. Anhand der Linien ist abzulesen auf welche Funktionen und Eigenschaften die View--Komponenten Zugriff haben.
%
\begin{figure}[H]
  \centering
  \includegraphics[width=\textwidth]{uml}
  \grayRule
  \caption[Komponentendiagramm]{Komponentendiagramm der Prototypen}
  \label{fig:uml}
\end{figure}
%
Die Komponente \tt{Contacts} fungiert als Container und ist das Herzstück der Anwendung. Er definiert die graphische Oberfläche und stellt alle notwendigen Funktionen bereit. Sie hat einen internen \tt{state} in dem sowohl die Kontaktliste, als auch die Daten für das Formular gespeichert sind. Im Diagramm ist der \tt{state} an der blauen Schrift zu erkennen.
Das Objekt \tt{editView} zeigt, welche Ansicht -- Liste oder Formular -- gerade aktuell ist und speichert den im Formular zu ladenden Kontakt.
Wie das Kontaktobjekt aufgebaut ist zeigt der blaue Kasten im Diagramm.
Wird eine Aktion zum Ändern der Ansicht aufgerufen, beispielsweise durch das Betätigen eines Knopfes, wird über die Funktion \tt{toggleEdit()} der interne \tt{state} aktualisiert und ein erneutes Rendern der Komponente eingeleitet. Dann wird entsprechend die Liste oder das Formular gerendert.\\
%
Der \tt{Header} implementiert das externe Modul \tt{react-detect-offline} und kann so den Netzwerkstatus anzeigen. Die Komponente hat Zugriff auf die \tt{toggleEdit}--Funktion und einen Knopf, der an diese gebunden ist. Damit kann das Rendern des Kontaktformulars eingeleitet werden.
Dieser Knopf soll nur angezeigt werden, wenn die Liste aktiv ist. Diese Information ist in dem  Attribut \tt{isOpen} abzulesen.\\
Die \tt{ContactList} repräsentiert die Kontaktliste und wird initial gerendert. Hier werden alle Kontakte als Liste dargestellt.
Es kann das Bearbeiten, durch den Aufruf der durchgereichten Funktion \tt{toggleEdit()} eingeleitet werden. Durch den Aufruf von \tt{remove"-Contact()} wird der Kontakt gelöscht.\\
%
Die Komponente \tt{ContactForm} zeigt, sofern vorhanden, alle im Kontakt gespeicherten Daten an.
Diese können hier bearbeitet werden.
Gibt es keine Kontaktdaten die geladen werden kann, kann hier ein neuer Kontakt angelegt werden.
Zusätzlich zu den Eingabefeldern für jedes Kontaktattribut hat sie zwei Knöpfe mit denen die Aktion bestätigt oder abgebrochen werden kann.
Für Ersteres hat sie Zugrifg auf die Funktionen \tt{addContact} und \tt{editContact}, für Letzteres kann sie \tt{toggleEdit()} aufrufen.\\
Sie ist neben \tt{Contacs} die einzige Komponente mit einem internen \tt{state}.
Dieser wird für die Ereignishandler benötigt, die auf die Veränderung der einzelnen Eingabefelder zu ``lauschen``. Der zu bearbeitende Kontakt wird hier zwischengespeichert und dann als Ganzes an \tt{Contacts} gegeben.\\
\todo{Konflikt--Dialog}\\\\
%
% 
%
Eine Backendimplementierung ist für den Prototypen \it{amilia-qouch} nicht notwendig, da dies bereits durch die Verwendung von CouchDB gegeben ist.
% Client - Server - Modell
\begin{figure}[H]
  \centering
  \includegraphics[width=0.6\textwidth]{qouch-model}
  \grayRule
  \caption{Client-Server-Modell}
  \label{fig:qouch-model}
\end{figure}
%
Obwohl Redux Offline nach eigener Aussage die Datenbank ersetzt~\cite{redux-offline}, stellt es keine Serverdatenbank zur Verfügung.
Deswegen wird ein Node Server erstellt der alle \gls{CRUD} Operationen unterstützt. Die Kontakte werden in einer \gls{JSON} Datei persistiert.

%
% Speicherung & Sync
%
\sub{Das Speichern der Daten}
Das Seichern von Kontakten wird in den Prototypen unterschiedlich implementiert.
%
% --------------------------------------------------------------------------------- Qouch
%
\subsub{Daten mit PochDB und CouchDB speichern}
Für den Protoyp \it{amilia-qouch} muss zunächst CouchDB installiert werden.
% sudo apt-get install couchdb
Sobald dieser Schritt erledigt ist läuft CouchDB auf \tt{localhost:5984} und ist einsatzbereit.
Das asynchrone \gls{API} von PouchDB stellt alle notwendigen Funktionen bereit die sowohl Callbacks, Promises als auch asynchrone Funktionen unterstützen. 
Das Listing \ref{code:pouch} führt alle benötigeten Funktionen auf und zeigt die notwendigen Schritte zur Synchronisation der lokalen PouchDB und CouchDB.\\
Die lokale Datenbank wird in Zeile eins erstellt. Wenn es die Datenbank mit dem Namen `contacts` bereits gibt, wird sie gestartet.\\
Um eine CouchDB Instanz zu erzeugen ist der Aufruf in Zeile zwei mit der URL zur CouchDB--Datenbank notwendig. Auch hier erstellt PouchDB die Datenbank, sofern sie noch nicht existiert. PouchDB funktioniert nun als Client zu einer online CouchDB Instanz.
Zur kontinuierlichen Synchronisation beider Instanzen, der lokalen PouchDB und der CouchDB, ist lediglich der Aufruf in Zeile vier erforderlich. Im optionalen Parameter können zum Beispiel FIlter oder Einstellungen zum wiederholten Synchronisationsversuch im Falle eines Fehlschlags gesetzt werden ~\cite{pouch_options}.
%
\begin{center}
  \lstinputlisting[language=REACT,
  numbers=left,xleftmargin=20pt,framexleftmargin=15pt,
  caption={Persistierung der Daten mit PouchDB und CouchDB}, 
  label=code:pouch]{code/Pouch.js}
\end{center}
%
% CREATE
Die Zeilen sieben bis zehn zeigen wie ein Kontakt erzeugt werden kann. Bevor das geschieht wird die ID gesetzt. PouchDB bietet zur Erstellung von Objekten auch \tt{localDB.post()} an. Bei dessen Verwendung wird \tt{\_id} von PouchDB automatisch generiert. Diese Variante wird jedoch nicht empfohlen, weil dann die IDs zufällig sind, die Objekte nicht danach sortiert werden können~\cite{pouch-create}.\\
% UPDATE
Das Aktualisieren eines Kontakts sieht ähnlich aus. Zuerst wird der entsprechende Kontakt wie in Zeile zwölf aus der Datenbank angefragt um dann in der Datenbank aktualisiert zu werden. Mit jedem Update bekommt ein Kontakt von PouchDB eine neue Revision.\\
% GET
Der Aufruf \tt{localDB.allDocs()} in Zeile 17 fragt alle in der lokalen Datenbank gespeicherten Kontakte an. Ohne den Parameter \tt{include\_docs: true} werden nur die \tt{\_id} und die \tt{\_rev} Eigenschaften eines jeden gespeicherten Kontakts zurückgegeben. Ist die Option \tt{conflicts} auf \tt{true} gesetzt, werden unter dem Attribut \tt{\_conflicts} Konfliktinformationen zu jedem Kontakt gespeichert. Ist ein Kontakt konfliktbehaftet hat er nun das Attribut \tt{\_conflicts}. Dort bedindet sich eine Liste mit allen konkurrierenden Revisionen.\\
% DELETE
Man kann einen Kontakt in PouchDB wie in Zeile 22 mittels \tt{localDB.remove(contact)} löschen. Der Kontakt ist dann nicht wirklich gelöscht sondern wird durch ein \tt{\_deleted} Attribut als solches markiert.
%Dann ist das Kontaktdokument mit all seinen Feldern gelöscht. Die lokale Datenbank soll sich mit CouchDB synchronisieren. Ist die Revision eines gelöschten Kontakts nicht mehr vorhanden, kann diese nicht repliziert werden. Deswegen wird der Kontakt wie in Zeile 19 als gelöscht markiert und aktualisiert.
%
% Redux Offline
%
\subsub{Daten mit Redux Offline speichern}
Die Idee hinter Redux Offline ist, dass der Redux Store die lokale Datenbank ersetzt. Sobald der Appstatus sich ändert, also irgendwo im Code \tt{setState()} ausgeführt wird, wird er automatisch lokal gespeichert. Dazu wird intern \tt{redux-persist} benutzt, dessen Funktionsweise in Abschnitt \ref{sub:reduxpersist} erläutert wird. Der Redux Store wird bei jeder Änderung persistiert und beim Start der Anwendung neu geladen.
% Es bedarf keiner zusätzlichen Implementierung für die lokale Speicherung der Kontaktdaten.\\
Wie die Daten mit Redux Offline gespeichert synchronisiert werden ist am folgenden Listing erklärt.
\begin{center}  \lstinputlisting[language=REACT,
  numbers=left,xleftmargin=20pt,framexleftmargin=15pt,
  caption={Speicherung der Daten mit Redux Offline}, 
  label=code:syncrdx]{code/Redux-sync.js}
\end{center}
%
Mit dem Aufruf der Aktion ADD\_CONTACT wird der Vorgang gestartet einen Kontakt hinzuzufügen. Die anzufragende URL ist im \tt{meta.offline.effect} Feld in Zeile neun festgelegt. Die Anfrage geht an den Server, welcher die Daten in der \gls{JSON} Datei persisitiert.
Der Reducer hat Zugriff auf das Aktionsobjekt. Dort ist der gerade hinzugefügte Kontakt gespeichert. Mit diesem wird in Zeile 23 der Appstate aktualisiert und so lokal gespeichert.\\
Ist die Netzwerkanfrage erfolgreich, wird die Aktion \tt{commit} in Zeile zwölf gefeuert. Die wird im Reducer in Zeile 25 behandelt. Der \tt{state} wird mit der Antwort vom Server aktualisiert und die Synchronisation ist vollzogen.\\\\
Wie die Serverimplementierung für den gerade beschriebenen Fall aussehen könnte, beschreibt das Listing \ref{code:server}.
%
\begin{center}  \lstinputlisting[language=REACT,
  numbers=left,xleftmargin=20pt,framexleftmargin=15pt,
  caption={Mögliche Serverimplementierung für das Hinzufügen eines Kontakts}, 
  label=code:server]{code/Server.js}
\end{center}
%
In Zeile zwei werden die Kontakte aus einer \gls{JSON} Datei geladen. Diese sind als Objekt in einem Array gespeichert. Bekommt der Server eine post--Anfrage, wird ein Kontaktobjekt mitgesendet. Dieser wird in Zeile fünf in einer Variable zwischengespeichert. 
Wird der Kontakt korrekt gesendet wird er in Zeile zehn dem Array hinzugefügt. Andererseits sendet der Server den \gls{HTTP}--Statuscode 400 an den Client. % bad request
Außerdem wird mithilfe des in Node integriertem Dateisystem\footnote{siehe hierzu: \url{https://nodejs.org/api/fs.html}} Moduls die \gls{JSON} Datei neu geschrieben. Nun ist der neue Kontakt persistiert. Kommt es beim Schreiben der Datei zu keinem Fehler, sendet der Server den frisch gespeicherten Kontakt zurück an den Client.
% -----------------------------------------------------------------------------
%
% Online / Offline
%
\sub{\label{sub:detect}Verbindungsstatus feststellen und ändern}
Für die Überprüfung der Verbindung zum Server wird das Modul React Detect Offline verwendet.
Es bietet zwei Komponenten, die entsprechend des Netzwerkstatus den ihnen gegebenen Inhalt rendern.
Der folgende Codeausschnitt zeigt eine Verwendung dieser beiden Komponenten. Ist die Anwendung online, wird ''you are online'' gerendert. Im anderen Fall ~''you are offline''.
%
\begin{center}
\lstinputlisting[language=REACT,caption={Beispiel einer React Detect Offline--Implementierung}, label=code:react-detect]{code/Header.js}
\end{center}
%
Das Modul verfolgt kontinuierlich den Onlinestatus des Browsers, indem es auf die Online- und Offlineereignisse der Webspezifikation reagiert. Zusätzlich fragt es alle fünf Sekunden die URL \url{https://ipv4.icanhazip.com} ab und rendert je nach Verbindungsstatus die entsprechende Komponente.
Verschiedene Parameter wie die URL oder das Intervall, in dem die URL aufgerufen wird, können konfiguriert werden~\cite{react-detect}. \\\\
%
Der Verbindungsstatus eines Gerätes kann im Browser geändert werden.
Die Prototypen, die im Rahmen dieser Arbeit entwickelt werden, sollen in den Browsern Firefox und Chrome lauffähig sein.\\
In Firefox lässt sich der Netzwerkstatus über das Einstellungsmenü ändern. Dort kann man entweder unter dem Punkt ''Sonstiges'' oder dem Punkt ''Web-Entwickler'' ''Offline arbeiten'' auswählen und ist vom Internet getrennt. Dieser Status lässt sich über den selben Weg rückgängig machen.\\
In Chrome öffnet man dazu die Entwicklertools, geht auf ''Netzwerk'' und klickt auf die Checkbox ''Offline'' am oberen Rand. Dieselbe Checkbox ist auch im ''Application''--Tab unter ''Service Workers'' zu finden.\\
Soll der Offlinestatus für zwei oder mehr Geräte hergestellt werden, kann der Server oder die CouchDB gestoppt werden. Damit der Status in der Oberfläche abgelesen werden kann, muss dem React Detect Offline--Modul die URL vom Server übergeben werden.
%
% UI
%
\section{Die graphische Oberfläche}
Aus den minimalen Anforderungen an die graphische Oberfläche ergibt sich das Design.
Anhand der folgenden Abbildungen werden die gefertigten Entwürfe der BenutzerInnenoberfläche dargestellt.\\\\
Diese Listenansicht in Abbildung \ref{fig:list} besteht aus dem Header / Kopf und den Listeneinträgen.
Sie zeigt die Kontakteinträge in beiden Netzwerkstatus: online (Abbildung \ref{fig:list-online}) und offline (\ref{fig:list-offline}).\\\\
%
%
Im Header ist abzulesen, ob die Anwendung gerade eine Verbindung zum Server hat oder nicht.
Für eine bessere Prägnanz wurden hierzu unterstützend die Farben Rot für keine Verbindung und Grün für eine bestehende Netzwerkverbindung gewählt.
Rechts im Header gibt es einen Knopf, mit dem man in die Ansicht gelangt, in der ein Kontakt hinzugefügt werden kann.\\
%
In der Liste sieht man die Namen der Person und jeweils einen Knopf zum Bearbeiten oder Löschen.
Mit der Betätigung des ''Delete''--Knopfes wird der entsprechende Eintrag in der Liste gelöscht
\begin{figure}[H]
  \centering
  \begin{subfigure}[t]{0.49\textwidth}
          \includegraphics[width=\textwidth]{list-online}
          \caption{Kontaktliste im Onlinestatus}
          \label{fig:list-online}
  \end{subfigure}
  ~ 
  \begin{subfigure}[t]{0.49\textwidth}
          \includegraphics[width=\textwidth]{list-offline}
          \caption{Kontaktliste im Offlinestatus}
          \label{fig:list-offline}
  \end{subfigure}
  \grayRule
  \caption{Die Kontaktliste in beiden Netzwerkstatus}
  \label{fig:list}
\end{figure}
Klickt man auf den Knopf zum Bearbeiten oder auf den zum Hinzufügen eines Kontakts, gelangt man in die Bearbeitungsansicht (vgl. Abbildung \ref{fig:edit}). Der Header ist bis auf den Knopf zum Hinzufügen eines Kontakts identisch zu dem der Liste. Auch hier ist abzulesen, ob die Anwendung on-- oder offline ist. Da man sich bereits in der Ansicht zum Anlegen oder Editieren eines Kontaks befindet, ist der Knopf im Header überflüssig.\\
Ein Kontakt hat einen Namen, eine E-Mailadresse und eine Telefonnummer. In dieser Ansicht gibt es für jedes Attribut ein Eingabefeld. Die Felder sind beim Bearbeiten des Kontakts vorausgefüllt. Mittels Betätigung des ''Speichern'' Knopfs werden die Änderungen übernommen, klickt man auf ''Cancel'', werden sie verworfen. In beiden Fällen gelangt man wieder zur Listenansicht.
\begin{figure}[H]
  \centering
  \begin{subfigure}[t]{0.49\textwidth}
          \includegraphics[width=\textwidth]{edit}
          \caption{Editieransicht im Onlinestatus}
          \label{fig:edit-online}
  \end{subfigure}
  ~ 
  \begin{subfigure}[t]{0.49\textwidth}
          \includegraphics[width=\textwidth]{edit-offline}
          \caption{Editieransicht im Offlinestatus}
          \label{fig:edit-offline}
  \end{subfigure}
  \grayRule
  \caption{Die Editieransicht in beiden Netzwerkstatus}
  \label{fig:edit}
\end{figure}
Sobald ein Konflikt entstanden ist, soll sich ein Dialog öffnen, der die nutzenden Personen darüber informiert, welcher Kontakteintrag konfliktbehaftet ist und mit welcher Version er konkurriert.
Anhand des Dialoginhalts kann unterschieden werden, welche Version die lokal gespeicherte ist und welche vom Server kommt.
Der Dialog beinhaltet außerdem zwei unterschiedlich farbige Knöpfe, die jeweils den Kontakteintrag in einer anderen Version anzeigen.
Durch Klick auf einen Knopf wird die bevorzugte Version des Kontakts gespeichert und die andere verworfen.
So kann ein Mensch entscheiden, welche Version behalten werden soll und es wird sichergestellt, dass keine Daten verloren gehen.\\
Für die Implementierung des Konfliktdialogs ist es notwendig, dass die zu untersuchenden Technologien die Möglichkeit bieten, Konflikte zu speichern oder wenigstens als solche zu identifizieren, sodass sie manuell gespeichert werden können.
%
% Tests
%
\section{WIP Tests}
\todo{Auch testen ob Offlinefunktionalität gewährleistet ist? -- Scope: Zeitlichen Rahmen sprengen usw.}\\
Um das Konfliktmanagement der zu testenden Technologien untersuchen zu können, müssen zunächst Konflikte erstellt werden. Dazu muss in erster Linie die Anwendunng auf mindestens zwei Geräten laufen und der Netzwerkstatus muss änderbar sein. Es ist außerdem hilfreich, wenn die Anwendung zeigt, in welchem Status sie sich befindet. Sie sollte wissen, ob sie on-- oder offline ist.\\
\todo{Kontakte editieren (auf 2 Geräten gleichzeitig) on-on, off-off, on-off? Dasselbe für Anlegen und löschen?}\\
Zur Auswertung der Daten sollten alle Ausgangspositionen, Vorgänge und Ergebnisse dokumentiert werden. Für einen Testfall kann hierfür zuerst der Kontakt aufgeschrieben werden. Dann die Operation, z.B. `bearbeiten des Kontakts Amilia` zusammen mit dem Verbindungsstatus (von Client und Server). Ebenso muss der Kontakt aufgeschrieben werden, wie er nach der Synchronisation aussieht. \todo{Lösung anbieten: .log--Datei?}
\highlight{Es stellt sich die Frage wie oft ein Testfall durchlaufen werden muss um eine sinnvolle Aussage treffen zu können}.