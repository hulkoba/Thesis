\chapter{\label{chap:konzeption}Konzeption}
Die erarbeiteten Anforderungen zur Untersuchung der Konfliktmanagementstrategien offlinefähiger Systeme werden in diesem Kapitel für die Konzeption angewendet.\\
Es soll für jede zu untersuchende Technologie ein Prototyp entwickelt werden. Im Rahmen dieser Arbeit entsteht ein Prototyp der \sc{Redux Offline} verwendet und einer in dem \sc{PouchDB} und \sc{CouchDB} eingesetzt wird. \todo{Entscheidung begründen?} Bis zu einem gewissen Status, nämlich dem der Verweindung der Technologien, sind beide Prototypen -- bis auf den Namen-- identisch. \highlight{Vielleicht hier erstmal klären dass die Prototypen bis zu einem gewissen Status identisch sind und deswegen von ihnen im Singular geschrieben wird?.}\\
Beginnend mit dem Aufbau der exemplarischen Anwendung werden in den folgenden Abschnitten die  Architektur  und schließlich \todo{blabla} aufgeführt.\\
% Den Raum aller möglichen Lösungen anhand der Anforderungen auf die in irgendeinem Sinn beste / geeignetste einzuschränken:\\
% App entwickeln, in der es einen `Verbindung unterbrechen` Knopf gibt ', oder ob es aus irgendwelchen Erwägungen notwendig sein könnte, das über eine separate Instanz zu machen.
%
% Aufbau
%
\section{Voraussetzungen?}
Als Grundstein beider Pro
\tt{create-react-app}
\section{Anwendungsaufbau}
%
% Architekturf
%
\sub{Architektur}
Container\\
Header\\
Liste\\
Form\\
Konflikt--Dialog\\
%
% Testfälle
%
\section{Testfälle}
Folgende Testfälle werden während der Entwicklung stetig durchgeführt. Das erfolgreiche Bestehen dieser Tests ist eine notwendige Qualitätseigenschaft / Voraussetzung der zu entwickelnden Prototypen.
\begin{itemize}
  	\item Netzwerkstatus anzeigen ? 
    \item Folgende Punkte müssten durch ServiceWorker usw sowohl on- als auch offline möglich sein. 
    \item Kontakt anlegen 
    \item Kontakt bearbeiten (Name, Email oder Telefonnummer)
    \item Kontakt löschen 
    \item lesen vorhandener Kontakte (lokal und Server)
\end{itemize}
%
% Entwicklungsumgebung
%
% \section{Entwicklungsumgebung}
% Für die Erstellung des Simulators wurde folgende Soft- und Hardware verwendet:
% \subsubsection{Software}
% \begin{itemize}
% 	\item Sublime Editor 3
% 	\item Visual Studio Code
% 	\item git, Version 2.7.4 zur Versionsverwaltung
% 	\item Inkscape, Version 0.92 zur Erstellung von Diagrammen und Zeichnungen
% \end{itemize}
% \subsubsection{Hardware}
% \begin{itemize}
% 	\item Tuxedo (Intel\textsuperscript{\textregistered} Core\tm i7-6500U, 2,50GHz x 4, 7,7 GB RAM) als ersten Entwicklungsrechner
% 	      (Betriebssystem: Ubuntu\footnote{ Download unter \url{https://www.ubuntu.com/download/desktop}} 16.06, 64-bit-Version)
% 	\item Lenovo Thinkpad X200 (Intel\textsuperscript{\textregistered} Core\tm   2 Duo, 2,40GHz, 8GB RAM) als zweiten Entwicklungsrechner (Betriebssystem: Debian 7.8, 64-bit-Version)
% \end{itemize}7