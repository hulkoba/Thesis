%
% LOCAL DB
%
\sub{Datenspeicherung}
Redux Offline speichert, wie in \autoref{sub:reduxpersist} beschrieben, alle im Redux Store verwalteten Daten im Local Storage.
\autoref{fig:local-rdx} zeigt alle gespeicherten Daten im Local Storage.
%
\begin{figure}[H]
  \centering
  \includegraphics[width=\textwidth]{impl/localRdx}
  \grayRule
  \caption[Gespeicherte Daten im Local Storage]{Gespeicherte Daten des Prototypen \it{amilia-rdx} im Local Storage,\\Screenshot: Developer Tools im Firefox Browser}
  \label{fig:local-rdx}
\end{figure}
% 
Der Prototyp \it{amilia-qouch} nutzt zur lokalen Datenspeicherung PouchDB.
PouchDB speichert die von NutzerInnen generierten Daten in IndexedDB, vgl. \autoref{chap:pouch}. In \autoref{fig:local-qouch} sind die gespeicherten Daten in der IndexedDB zu sehen.
%
\begin{figure}[H]
  \centering
  \includegraphics[width=\textwidth]{impl/localQouch}
  \grayRule
  \caption[Gespeicherte Daten in IndexedDB]{Gespeicherte Daten des Prototypen aus \it{amilia-qouch} in IndexedDB,\\Screenshot: Developer Tools im Firefox Browser}
  \label{fig:local-qouch}
\end{figure}
%
Es ist zu beachten, dass der Prototyp \it{amilia-rdx} den gesamten \gls{App}state der Anwendung persistiert und nicht nur die erstellten Kontakte, so wie es beim Prototypen \it{amilia-qouch} der Fall ist.
%
% SYNC
%
\sub{Datenbanksynchronisation}
Zwischen PouchDB und CouchDB können Daten in Echtzeit synchronisiert werden.
Um die Live--Replikation zu aktivieren, muss im Synchronisationsaufruf der Parameter \tt{live: true} gesetzt sein.
Bricht die Internetverbindung ab, stoppt auch die Synchronisation.
Dank der angegebenen Parameter \tt{retry: true} versucht PouchDB die Synchronisation solange neu zu starten, bis die Anwendung wieder mit dem Internet verbunden ist. \autoref{code:sync} zeigt die Implementation der Datenbankensynchronisation im Prototypen \it{amilia-qouch}.
%
\begin{center}
  \lstinputlisting[language=REACT,
  numbers=left,xleftmargin=20pt,
  firstline=1,lastline=5,
  framexleftmargin=15pt,
  caption={Synchronisation zwischen PouchDB und CouchDB im Prototyp \it{amilia-qouch}},
  label=code:sync]{code/sync.js}
\end{center}
% 
Redux Offline nimmt den EntwicklerInnen auch Arbeit bei der Datensynchronisation ab.
Alle Daten, die sich in der Queue befinden, werden automatisch an den Server gesendet, sobald eine Internetverbindung besteht.
% Ist der Client offline, werden Aktionen solange versucht auszuführen, bis die Anwendung wieder mit dem internet verbunden ist.
Das funktioniert jedoch nicht, wenn der Server nicht erreichbar ist.
In der Dokumentation von Redux Offline steht, dass die Aktion so oft wiederholt wird, bis die Anwendung wieder mit dem Internet verbunden ist ~\cite{giving-up}.
Allerdings wird die Aktion abgebrochen und die \sc{rollback} Aktion gefeuert, wenn der Server nicht verfügbar ist.
\begin{center}
  \lstinputlisting[language=REACT,
  numbers=left,xleftmargin=20pt,
  firstline=6,
  framexleftmargin=15pt,
  caption={Discard--Konfiguration für amilia-rdx},
  label=code:discard]{code/sync.js}
\end{center}
%
Die Discard--Konfiguration bestimmt wann eine Aktion abgebrochen und wann sie immer wieder neu gestartet wird.
Im \autoref{code:discard} ist in den Zeilen 1 bis 4 abzulesen wie diese Konfiguration überschrieben werden kann.
Nun wird die Aktion nur abgebrochen, wenn der Server verfügbar ist und einen \gls{HTTP}--Status von 400 bis 500 zurückgibt.
Die 4xx Statuscodes werden geliefert, wenn die Ursache des Scheiterns der Anfrage clientseitig verantwortet wird. Der \gls{HTTP}--Statuscode 500 wird bei einem internen Serverfehler als Antwort geliefert. 
In den darauffolgenden Zeilen wird die eigens implementierte Konfiguration mit der Standardkonfiguration von Redux Offline zusammengeführt.
Alle Daten werden nun auch nach einem Serverausfall an den Server gesendet.\\
Im Gegensatz zum \it{amilia-qouch}--Prototypen werden die Daten nur in eine Richtung automatisch gesendet.
Es gibt keine Implementierung einer automatischen Replikation der Daten vom Server zum lokalen Speicher.