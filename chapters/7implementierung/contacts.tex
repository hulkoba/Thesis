Das Herzstück der hier zu entwickelnden Anwendungen sind die Contacts--Komponenten.
Sie beinhalten die zentralen Funktionalitäten und entscheiden welche Anzeigeelemente gerendert werden und welche nicht.
% In beiden Prototypen sind sie auf zwei Dateien aufgeteilt.
\tt{Contacts.js} ist die Viewkomponente und somit für das Rendern der anderen Komponenten zuständig. 
Die Containerkomponente ist die Datei \tt{ContactsContainer.js}. 
Hier sind Funktionen implementiert über die der State manipuliert werden kann.\\
% Im Prototypen \it{amilia-qouch} besteht sie aus der Datei \tt{Contacts.js} in der ein interner State festgelegt ist.
%
Im Prototypen \it{amilia-rdx} werden in \tt{actions.js} die Aktionen definiert, die über die Containerkomponente \tt{ContactsContainer.js} von jeder Komponente aufgerufen werden können um den \gls{App}state zu manipulieren.
Die Manipulation wird in \tt{reducer.js} behandelt, wo die geänderte Kopie des \gls{App}status wieder an den Store zurückgegeben wird.
%  und sie wieder bei der \tt{Contacts.js} ankommt.
% Der Einfachheit halber wird im Folgenden von der Contacts--Komponente geschrieben, wobei der Zusammenschluss dieser soeben beschriebenen Dateien gemeint ist.\\\\
%
% 
Die Methode \tt{componentDidMount} ist die dritte im React Komponenten--Lebenszyklus\footnote{ \url{https://reactjs.org/docs/react-component.html\#the-component-lifecycle} -- Zugriff: 28.07.1018} und wird aufgerufen sobald die Anwendung an die DOM Elemente gemountet ist.
Hier werden die Kontakte geladen und im \gls{App}state gespeichert.
Sobald sicht der \gls{App}state ändert, wird die \tt{render} Funktion der Komponente aufgerufen.
\autoref{code:render} zeigt die Renderfunktion in \tt{ContactsContainer.js} des Prototypen \it{amilia-qouch}.
%
\begin{center}
  \lstinputlisting[language=REACT,numbers=left,xleftmargin=20pt,framexleftmargin=15pt,
  lastline=25,
  caption={Die Renderfunktion in \tt{ContactsContainer.js} des Prototypen \it{amilia-qouch}},
  label=code:render]{code/render.js}
\end{center}
%
Der Header wird in den Zeilen 5 bis 7 gerendert.
In Zeile 6 wird ihm die Information \tt{isOpen} übergeben, die aussagt, ob der Editiermodus aktiv ist, also ob das Formular gerade geöffnet ist oder nicht.
Ihm wird auch die Funktion \tt{toggleEdit} übergeben, welche genau diese Statuseigenschaft wechselt.\\
%
Die Zeilen 9 bis 14 wird, sofern der Wert \tt{hasConflict} im State auf \tt{true} gesetzt ist, der Konfliktdialog gerendert.
Der Dialog existiert nur im \it{amilia-qouch} Prototypen.
Ihm werden die beiden konfliktbehafteten Versionen des Kontakts und eine Funktion zum Löschen der verlierenden Version übergeben. 
Näheres dazu wird im \autoref{chap:konfliktmanagement} erklärt.\\
%
In den Zeilen 16 bis 23 wird die Contacts--Viewkomponente gerendert.
Ihr werden alle Eigenschaften und Funktionen übergeben, die für ihre Kindkomponenten notwendig sind.
\autoref{code:renderview} zeigt die Renderfunktion der Viewkomponente.\\
In den Zeilen 2 bis 4 werden der kürzeren Schreibweise halber die übergebenen Eigenschaften genommen und den entsprechenden Variablen zugewiesen.
% 
%
\begin{center}
  \lstinputlisting[language=REACT,numbers=left,xleftmargin=20pt,framexleftmargin=15pt,
  firstline=30,
  caption={Die Renderfunktion in \tt{Contacts.js} beider Prototypen},
  label=code:renderview]{code/render.js}
\end{center}
%
% 
In den Zeilen 7 bis 19 wird anhand der \tt{isOpen}--Information entschieden ob das Formular oder die Liste gerendert wird.
%
Das Kontaktformular ist aufgeteilt in eine Container-- und eine Viewkomponente, weswegen hier in \tt{Contacts.js} die Containerkomponente gerendert wird.
% Wo in der Viewkomponende alle beinhalteten Elemente nur dargestellt werden, besitzt die Containerkomponente die Logik.
Der \tt{FormContainer} bekommt in Zeile 12 den zu bearbeitenden Kontakt.
Ist dieser leer, wird über das Formular ein neuer angelegt.
Des Weiteren bekommt er ebenfalls die \tt{toggleEdit()}--Funktion, eine zum Anlegen und eine zum Bearbeiten des Kontakts (Zeilen 9 bis 11).\\
%
Der Liste werden alle im \gls{App}state gespeicherten Kontakte, die durch den Container durchgereicht werden, gegeben.
Außerdem bekommt sie die \tt{toggleEdit()}--Funktion und eine zum Löschen eines Kontakts.
Es ist zu beachten, dass hier bei der \tt{toggleEdit()} das \tt{bind(this, null)} fehlt. Das liegt daran, dass wenn man aus der Liste das Formular öffnet, man den ''Bearbeiten''--Knopf betätigt hat und der zu bearbeitende Kontakt im Formular geladen wird.
Dieser Kontakt wird dann für die Dauer seiner Bearbeitung im \tt{state.editView.contact} gespeichert.\\\\
%
%
In \autoref{chap:persist} wurde beschrieben, wie die Daten in den Prototypen angelegt, bearbeitet und gelöscht werden. Darauf verweisend wird hier nun gezeigt, wie diese Funktionen zum Einsatz kommen.