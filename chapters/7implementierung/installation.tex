Beide entwickelten Prototypen sind als öffentliche Repositories auf GitHub\footnote{Software--Entwicklungs--Plattform \url{https://github.com/}} zu finden. 
Um sie zu installieren müssen folgende Schritte ausgeführt werden.
%
%
\sub{\label{chap:install:qouch}amilia-qouch}
Für die Ausführung dieser Anwendung muss CouchDB installiert sein auf der ein Admin mit dem Passwort ''admin'' besteht.
Zur Installation von CouchDB kann die Anleitung auf \url{http://docs.couchdb.org/en/2.1.2/install/index.html} verwendet werden.\\
Dann müssen die \gls{CORS} aktiviert werden. Das ist eine Webtechnologie die es Webanwendungen erlaubt Ressourcen von einer anderen Domain zu benutzen.
% PouchDb läuft auf ner anderen Domain als CouchDB
Es gibt bereit ein Skript, das genau das macht. Dazu müssen die folgenden Befehle ausgeführt werden.
%
\lstset{language=sh, caption={},belowcaptionskip=0.3\baselineskip,xleftmargin=15pt,framexleftmargin=5pt}
\begin{lstlisting}
npm install -g add-cors-to-couchdb
add-cors-to-couchdb
\end{lstlisting}
%
Waren diese Schritte erfolgreich, kann die Anwendung installiert werden.\\\\
1. Zuerst muss das Repository kopiert werden:
\begin{lstlisting}
git clone git@github.com:hulkoba/amilia-qouch.git
# oder
git clone https://github.com/hulkoba/amilia-qouch.git
\end{lstlisting}
2. Dann wird in das Verzeichnis navigiert um dort alle Abhängigkeiten zu installieren.
\begin{lstlisting}
cd amilia-qouch && npm install
\end{lstlisting}
3. Durch den Aufruf
\begin{lstlisting}
npm start
\end{lstlisting}
wird die Anwendung gestartet und läuft auf \url{http://localhost:3000/}.
%
%
%
\sub{amilia-rdx}
1. Auch hier muss das Repository zuerst kopiert werden:
\lstset{language=sh, caption={},belowcaptionskip=0.3\baselineskip}
\begin{lstlisting}
git clone git@github.com:hulkoba/amilia-rdx.git
# oder
git clone https://github.com/hulkoba/amilia-rdx.git
\end{lstlisting}
2. Schritt zwei ist identisch mit dem in der Anleitung auf \hyperref[chap:install:qouch]{\it{amilia-qouch}}\\
3. Durch die Aufrufe
\begin{lstlisting}
npm run server
npm start
\end{lstlisting}
in separaten Terminalfenstern wird zuerst der Server und dann die Anwendung gestartet.
Die Anwendung läuft nun auf \url{http://localhost:3000/}.