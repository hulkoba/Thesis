Beide entwickelten Prototypen sind auf der beigelegten CD zu finden.
Für die Ausführung beider Prototypen ist die Installation von Node.js eine Voraussetzung.
Ein Node.js--Installationsprogramm steht unter \url{https://nodejs.org/en/download/} zum Download bereit.
Um die Prototypen zu starten müssen folgende Schritte ausgeführt werden.
%
%
\sub{\label{chap:install:qouch}amilia-qouch}
Für die Ausführung dieser Anwendung muss CouchDB auf dem Anwendungsgerät installiert sein.
Zur Installation von CouchDB kann die Anleitung auf \url{http://docs.couchdb.org/en/2.1.2/install/index.html} verwendet werden.
Es muss außerdem ein User ''admin'' mit dem Passwort ''admin'' existieren, um auf die Datenbank zugreifen zu können.\\
Dann müssen die \gls{CORS} Optionen aktiviert werden.
Das ist eine Webtechnologie, die es Webanwendungen erlaubt, Ressourcen von einer anderen, entfernten Domain zu benutzen.
% PouchDb läuft auf einer anderen Domain als CouchDB
Es gibt bereits ein Skript, welches dies durchführt. Dazu müssen folgende Befehle ausgeführt werden.
%
\lstset{language=sh, caption={},belowcaptionskip=0.3\baselineskip,xleftmargin=15pt,framexleftmargin=5pt}
\begin{lstlisting}
npm install -g add-cors-to-couchdb
add-cors-to-couchdb
\end{lstlisting}
%
Waren diese Schritte erfolgreich, kann die Anwendung gestartet werden.\\\\
1. Zuerst muss der Ordner \it{amilia-qouch} auf den Computer kopiert werden.\vspace{0.3cm}\\
2. Dann werden dort mit dem folgenden Befehl alle Abhängigkeiten installiert.
\begin{lstlisting}
npm install
\end{lstlisting}\vspace{0.3cm}
%
3. Durch den Aufruf
\begin{lstlisting}
npm start
\end{lstlisting}
wird die Anwendung gestartet und läuft auf \url{http://localhost:3000/}.
%
%
%
\sub{amilia-rdx}
1. Auch hier muss der Ordner \it{amilia-qouch} auf den Computer kopiert werden.\vspace{0.3cm}\\
2. Schritt zwei ist identisch mit dem in der Anleitung auf \hyperref[chap:install:qouch]{\it{amilia-qouch}}\vspace{0.3cm}\\
3. Durch die Aufrufe
\begin{lstlisting}
npm run server
npm start
\end{lstlisting}\vspace{0.3cm}
in separaten Terminalfenstern wird zuerst der Server und dann die Anwendung gestartet.
Die Anwendung läuft nun auf \url{http://localhost:3000/}.