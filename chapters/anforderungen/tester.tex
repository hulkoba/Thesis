\sub{TesterInnen Perspektive}
Die folgende Tabelle zeigt die Software-Anforderungen an eine offlinefähige Kontaktliste aus Perspektive der TesterInnen.
\begin{longtable}[c]{@{}
	>{\columncolor[HTML]{CFFCC2}}l ll@{}}
	\toprule
	\multicolumn{1}{p{0.15\textwidth}}{\cellcolor[HTML]{cffcc2}\textbf{ID}}
	                                                                   & \multicolumn{1}{p{0.85\textwidth}}{\cellcolor[HTML]{cffcc2}\textbf{Anforderung aus TesterInnenperspektive}} \\
	\hline \noalign{\vskip 0.1cm}
	\endfirsthead
	%
	\endhead
	%
	\multicolumn{1}{l}{\cellcolor[HTML]{cffcc2}\textbf{User-Story 14}} &
	\multicolumn{1}{p{0.85\textwidth}}
	{Ich als TesterIn möchte sicherstellen, dass der Netzwerkstatus der Anwendung änderbar ist, um zwischen offline und online zu wechseln.}\\
	\midrule
	%
	\multicolumn{1}{l}{\cellcolor[HTML]{cffcc2}\textbf{User-Story 15}} &
	\multicolumn{1}{p{0.85\textwidth}}
	{Ich als TesterIn möchte die Anwendung auf mindestens zwei Geräten verwenden, um Kontakte gleichzeitig zu bearbeiten zu können.}\\
	\midrule
	%
	\multicolumn{1}{l}{\cellcolor[HTML]{cffcc2}\textbf{User-Story 16}} &
	\multicolumn{1}{p{0.85\textwidth}}
	{Ich als TesterIn möchte Konflikte forcieren können, um das Verhalten der Anwendung zu testen.}\\
	\midrule
	%
	\multicolumn{1}{l}{\cellcolor[HTML]{cffcc2}\textbf{User-Story 17}} &
	\multicolumn{1}{p{0.85\textwidth}}
	{Ich als TesterIn möchte einen Eintrag editieren können wenn 1. beide Client und Server online sind, 2. entweder Client oder Server offline ist oder 3. beide Parteien offine sind. \todo{daraus 3 Stories machen?}}\\
	\midrule
	%
	\multicolumn{1}{l}{\cellcolor[HTML]{cffcc2}\textbf{User-Story 18}} &
	\multicolumn{1}{p{0.85\textwidth}}
	{Ich als TesterIn möchte die Testfälle detailliert dokumentieren, um sie auswerten zu können.}\\
	% end
	\bottomrule \cellcolor[HTML]{FFFFFF}
	\vspace{0.1cm}\\
	\noalign{\hspace{0.0525\textwidth}\grayRule}
	\caption{Anforderungen aus TesterInnenperspektive}
	\label{tab:test}\\
\end{longtable}
