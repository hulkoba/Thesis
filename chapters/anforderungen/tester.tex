\sub{Terster Perspektive}
Die folgende Tabelle zeigt die Software-Anforderungen an eine offlinefähige Kontaktliste aus Perspektive der TesterInnen.
\begin{longtable}[c]{@{}
	>{\columncolor[HTML]{CFFCC2}}l ll@{}}
	\toprule
	\multicolumn{1}{p{0.15\textwidth}}{\cellcolor[HTML]{cffcc2}\textbf{ID}}
	                                                                   & \multicolumn{1}{p{0.85\textwidth}}{\cellcolor[HTML]{cffcc2}\textbf{Anforderung aus TesterInnenperspektive}} \\
	\hline \noalign{\vskip 0.1cm}
	\endfirsthead
	%
	\endhead
	%
	\multicolumn{1}{l}{\cellcolor[HTML]{cffcc2}\textbf{User-Story 14}}  &
	\multicolumn{1}{p{0.85\textwidth}}
	{Ich als TesterIn möchte die Daten lokal und auf dem Server speichern, um deren Erreichbarkeit unabhängig vom Internetstatus zu gewährleisten.}\\
	\midrule
	%
	\multicolumn{1}{l}{\cellcolor[HTML]{cffcc2}\textbf{User-Story 15}}  &
	\multicolumn{1}{p{0.85\textwidth}}
	{Ich als TesterIn möchte ich nur die Adressbucheinträge oder deren Aktualisierungen laden, die sich nicht schon auf dem Endgerät befinden, um Datentraffic und Ladezeiten zu sparen.}\\
	\midrule
	%
	\multicolumn{1}{l}{\cellcolor[HTML]{cffcc2}\textbf{User-Story 16}}  &
	\multicolumn{1}{p{0.85\textwidth}}
	{Ich als TesterIn möchte ich jeden Eintrag identifizieren, um jedem Adressbucheintrag Operationen zuzuweisen und einzelne Kontakte zu finden.}\\
	\midrule
	%
	\multicolumn{1}{l}{\cellcolor[HTML]{cffcc2}\textbf{User-Story 17}}  &
	\multicolumn{1}{p{0.85\textwidth}}
	{Ich als TesterIn möchte ich jeden Eintrag versionieren, um zu wissen ob wann ein Eintrag bearbeitet wurde.}\\
	% end
	\bottomrule \cellcolor[HTML]{FFFFFF}
	\vspace{0.1cm}\\
	\noalign{\hspace{0.0525\textwidth}\grayRule}
	\caption{Anforderungen aus TesterInnenperspektive}
	\label{tab:test}\\
\end{longtable}
