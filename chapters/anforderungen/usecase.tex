\section{Anwendungsfälle}
Aus den in Kapitel \ref{chap:szenarien} erarbeiteten Szenarien ergeben sich die folgenden \b{drei} Use-Cases, die von der Anwendung erfüllt werden sollen.\\
\begin{table}[H]
\centering
  \begin{tabular}{@{}>{\columncolor[HTML]{cffcc2}}l ll@{} p{0.1\textwidth}p{0.4\textwidth}p{0.4\textwidth}} \toprule
\multicolumn{1}{c}{\cellcolor[HTML]{cffcc2}\textbf{ID}}
& \multicolumn{1}{c}{\cellcolor[HTML]{cffcc2}\textbf{Anwendungsfall}}
& \multicolumn{1}{c}{\cellcolor[HTML]{cffcc2}\textbf{Beschreibung}} \\
\hline
% UC1
\multicolumn{1}{l}{\cellcolor[HTML]{cffcc2}\textbf{UC1}} & \multicolumn{1}{p{0.35\textwidth}}{Ich.., um ...}
& \multicolumn{1}{p{0.55\textwidth}}{Es passiert das und das.} \\
\midrule
% UC2
\multicolumn{1}{l}{\cellcolor[HTML]{cffcc2}\textbf{UC2}} & \multicolumn{1}{p{0.35\textwidth}}{Ich.., um ...}
& \multicolumn{1}{p{0.55\textwidth}}{Es passiert das und das.} \\
\midrule
% UC3
\multicolumn{1}{l}{\cellcolor[HTML]{cffcc2}\textbf{UC3}} & \multicolumn{1}{p{0.35\textwidth}}{Ich ..., um d...}
& \multicolumn{1}{p{0.55\textwidth}}{Es passiert das und das.}\\
%\midrule
\bottomrule \cellcolor[HTML]{FFFFFF} \vspace{0.1cm}
\end{tabular}
\grayRule
  \caption{Anwendungsfälle}
  \label{tab:usecase}
\end{table}

Dann Use-Case-Diagramm