\chapter{\label{chap:konzeption}Konzeption}
Die erarbeiteten Anforderungen an ... werden in diesem Kapitel für die Konzeption angewendet.
Beginnend mit dem Aufbau der Anwendung werden in den folgenden Abschnitten das Design, die von der Anwendung genutzten Daten, Anwendungsfälle,
die Architektur und schließlich die Komponenten der Entwicklumgsumgebung aufgeführt.
\section{Anwendungsaufbau}
\section{Datengrundlage}

\section{Anwendungsfälle}
Aus den in Kapitel \ref{chap:anforderungen} beschriebenen Anforderungen und den in Kapitel \ref{chap:szenarien} erarbeiteten Szenarien ergeben sich die folgenden \textit{zahl} Use-Cases, die von der Anwendung erfüllt werden sollen.

\section{Architektur}
%\clearpage
\section{Die graphische Oberfläche}

% TESTFÄLLE
\section{Testfälle}
Folgende Testfälle werden während der Entwicklung stetig durchgeführt. Das erfolgreiche Bestehen dieser Tests ist eine notwendige Qualitätseigenschaft der zu entwickelnden Applikation.


%
% Entwicklungsumgebung
%
%\clearpage
\section{Entwicklungsumgebung}
Für die Erstellung der Smartphone-Applikation wurde folgende Soft- und Hardware verwendet:
\subsubsection{Software}
\begin{itemize}
	\item Android Studio\footnote{ Download unter \url{http://developer.android.com/sdk/index.html}} Version 1.1. Enthält Android Studio IDE, Android \gls{SDK}-Tools, Android 5.0 Plattform, Android 5.0 Emulator System Image mit Google \glspl{API}
	\item Sublime Editor 3
	\item git, Version 2.7.4 zur Versionsverwaltung
	\item Google Drive zur Erstellung der Diagramme, Zeichnungen und Grafiken
\end{itemize}
\subsubsection{Hardware}
\begin{itemize}
	\item Tuxedo (Intel\textsuperscript{\textregistered} Core\texttrademark i7-6500U, 2,50GHz x 4, 7,7 GB RAM) als ersten Entwicklungsrechner
	(Betriebssystem: Ubuntu\footnote{ Download unter \url{https://www.ubuntu.com/download/desktop}} 16.06, 64-bit-Version)
	\item Lenovo Thinkpad X200 (Intel\textsuperscript{\textregistered} Core\texttrademark   2 Duo, 2,40GHz, 8GB RAM) als zweiten Entwicklungsrechner (Betriebssystem: Debian 7.8, 64-bit-Version)
	\item Android Testgeräte: Samsung Galaxy Note 2, Samsung Nexus S, LG Nexus 4, LG Nexus 5, HTC Desire HD
\end{itemize}
