\chapter{\label{chap:konzeption}Konzeption}
Die erarbeiteten Anforderungen an ... werden in diesem Kapitel für die Konzeption angewendet.
Beginnend mit dem Aufbau der Anwendung werden in den folgenden Abschnitten die Anwendungsfälle,
die Architektur und schließlich die Komponenten der Entwicklumgsumgebung aufgeführt.
%
% Aufbau
%
\section{Anwendungsaufbau}
%
% Architekturf
%
\section{Architektur}
%
% Testfälle
%
\section{Testfälle}
Folgende Testfälle werden während der Entwicklung stetig durchgeführt. Das erfolgreiche Bestehen dieser Tests ist eine notwendige Qualitätseigenschaft der zu entwickelnden Applikation.
%
% Entwicklungsumgebung
%
\section{Entwicklungsumgebung}
Für die Erstellung des Simulators wurde folgende Soft- und Hardware verwendet:
\subsubsection{Software}
\begin{itemize}
  \item Sublime Editor 3
  \item git, Version 2.7.4 zur Versionsverwaltung
  \item Inkscape, Version 0.92 zur Erstellung von Diagrammen und Zeichnungen
\end{itemize}
\subsubsection{Hardware}
\begin{itemize}
  \item Tuxedo (Intel\textsuperscript{\textregistered} Core\tm i7-6500U, 2,50GHz x 4, 7,7 GB RAM) als ersten Entwicklungsrechner
        (Betriebssystem: Ubuntu\footnote{ Download unter \url{https://www.ubuntu.com/download/desktop}} 16.06, 64-bit-Version)
  \item Lenovo Thinkpad X200 (Intel\textsuperscript{\textregistered} Core\tm   2 Duo, 2,40GHz, 8GB RAM) als zweiten Entwicklungsrechner (Betriebssystem: Debian 7.8, 64-bit-Version)
  \item Testgeräte...
\end{itemize}
