\sub{Realm}
Backend für mobile Anwendungen (Java, Swift, C\#, JS). Realm Datenbank oder Realm Platform(= DB+ Object Server).
Schreiben groß `OIffline First Experience` überall hin (webseite, whitepaper...)
\begin{itemize}
  \item lokale DB, plattformübergreifend
  \item Object Server fungiert als Middleware-Komponente in der mobilen \gls{App}-Architektur und managed die Datensynchronisation, eventhandling und Integration in Legacy-Systeme. Kann Daten effizient und simultan synchronisieren und \b{löst in Echtzeit, automatisch Konflikte}
  \item \b{Key-features:} Datensynchronisation in Echtzeit, Skalierbarkeit, Cross-Platform Datenmodell, Eventhandling, regelmäßige Backups, Datenintegrations API, Datensicherheit
  \item \b{key mobile use cases:} Reactive app architectures, Offline-first experiences, mobilizing legacy apis, realtime collaboration
\end{itemize}~\cite{realm_whitepaper}

\begin{itemize}
  \item Realtime Data Sync (sendet automatisch Änderungen in Echtzeit)
  \item Daten-sync-Protokoll komprimiert die marginalen Änderungen (statt das ganze Objekt) in Binärformat und übergibt sie zwischen Gerät und Server.
  \item synchronisiert die spezifischen Operationen zusammen mit den Daten
  \item Diese zusätzlichen Informationen erfassen genau das, was man beabsichtigt hat, \b{sodass das System Konflikte automatisch auflösen kann}. Dies führt zu einer vorhersagbaren Synchronisation ohne manuellen Eingriff, der die Leistung beeinträchtigt.
  \item (Objektorientierte) Datenbank auf dem Gerät
  \item Echtzeit Synchronisation
  \item Konfliktlösung benutzt OT (und vorgegebene Regeln): Man kann custom Konfliktlösungs-Regeln erstellen
  \item Unterstützung von Transaktionen? Ist das nicht normal?  -- Konfliktlösung passiert auf \b{Transaktionsebene}
  \item Wenn Änderungen aufgrund einer unterbrochenen Netzwerkverbindung offline gehen oder das Gerät leer ist, gehen keine Daten verloren.
\end{itemize}~\cite{realm_offline_whitepaper}