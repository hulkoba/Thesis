  \sub{Operational Transformation}
    \gls{OT} ist eine weit verbreitete Technologie zur Unterstützung von Funktionalitäten in \gls{kollaborativ}er Software. Sie stammt aus einer im Jahre 1989 veröffentlichten Forschungsarbeit und wurde ursprünglich nur für die gemeinsame Bearbeitung von Klartext-Dokumenten entwickelt~\cite{ot_paper}. Später ermöglichte weitere Forschung \gls{OT} durch Unterstützung von \textit{Sperrungen, Konfliktlösungen, Benachrichtigungen, Bearbeitung von Baumstrukturierten Dokumenten, ...} zu verbessern und erweitern. Im Jahr 2009: Google Wave, Google Docs\\
    % Ideen: einfach und mathematisch elegant
    \highlight{Es wird das Problem untersucht, dass \gls{OT} in einer idealen Umgebung löst und dadurch zu einem funktionierendem Algorithmus gelangt.\\
    Ziel: Mehrere BenutzerInnen können gleichzeitig an einem Dokument arbeiten, sehen Änderungen der anderen in Echtzeit (live), ohne dass einer Verzögerung durch die \gls{Netzwerklatenz} verursacht wird. Gleichzeitig auftretende Mehrfachänderungen sollen nicht zu unterschiedlichen Dokumentenzuständen führen.}
    \subsub{Funktionsweise}
      \Gls{kollaborativ}e Systeme, die \gls{OT} verwenden, benutzen normalerweise den replizierten Dokumentenspeicher. Das heißt jeder \b{Client} verfügt über eine eigene Kopie des Dokuments.
      Jede Änderung an einem freigegebenen Dokument wird als Operation dargestellt. Operationen sind Repräsentationen von Änderungen an einem Dokument. (Beispielsweise: Füge 'Hello world!' an Position 0 in das Textdokument ein).  Eine Operation zeichnet im Wesentlichen den Unterschied zwischen einer und der nachfolgenden Version eines Dokuments auf. Die Anwendung einer Operation auf das aktuelle Dokument führt zu einem neuen Dokumentstatus.
      Die Operationen erfolgen auf lokalen Kopie und die Änderungen werdenn an alle anderen \b{Clients} weitergegeben.  Wenn ein \b{Client} die Änderungen von einem anderen \b{Client} empfängt, werden die Änderungen normalerweise \b{vor} ihrer Ausführung transformiert.
      \highlight{Die Transformation stellt sicher, dass anwendungsabhängige Konsistenzkriterien (Invarianten) von allen Standorten gepflegt werden. }\\
      Es gibt die Operationen \b{Einfügen}\\
      Das Einfügen besteht aus dem eingefügten Text und dessen Position im Dokument (\tt{insert('h', 0)}).
      Für die Position kann ein Koordinatensystem ermittelt werden (Zeilennummer: Position in Zeile oder einfacher: Dokument wie eine Folge von Zeichen behandeln, also einfach einen nullbasierten Index vergeben.)\\
      und \b{Löschen}\\
      Löschen(5,6) = löscht 5 Zeichen, beginnend bei Position 6.
      Mehr benötigt man nicht, denn update = delete \& insert\\\\
      Um gleichzeitige Operationen zu behandeln, gibt es eine Funktion (normalerweise \texttt{Transform} genannt), die zwei Operationen übernimmt, die auf denselben Dokumentstatus angewendet wurden (aber auf verschiedenen Clients).
      Daraus wird eine neue Operation berechnet, die nach der zweiten Operation angewendet werden kann. Diese behält die erste beabsichtigte Änderung der Operation.

      Des Weiteren unterstützt \gls{OT} Operationen wie \tt{update}, \tt{point}, \tt{lock}.
      \paragraph{Beispiel}:
      Benutzer A fügt an Position 12 das Zeichen 'A' ein
      Benutzer B fügt am Anfang des Dokuments ein 'B' ein.
      Die konkurrierenden Operationen sind daher Einfügen (12, 'A') und Einfügen (0, 'B').
      Wenn wir die Operation von B einfach an Client A senden und dort anwenden würden, gäbe es ein Problem.
      Aber wenn die Operation von A an B gesendet, und angewandt wird nachdem Operation B angewandt wurde ist, würde das Zeichen 'A' eine Position zu weit links von der korrekten Position eingefügt werden.
      Dokumentstatus A und Dokumentstatus B sind nicht identisch.\\
      Daher muss A's \tt{insert(12, 'A')} gegen die Operation von B transformiert werden. So wird berücksichtigt, dass B ein Zeichen vor der Position 12 eingefügt hat (die die Operation \tt{insert(13, 'A')} erzeugt.)\\
      Diese neue Operation kann auf Dokument B nach B's Operation angewandt werden.
      Die Grundidee von \gls{OT} besteht darin, die Parameter einer Editieroperation gemäß den Auswirkungen \b{zuvor ausgeführter} konkurrierender Operationen anzupassen, so dass die transformierte Operation die korrekte Wirkung erzielen und die Dokumentenkonsistenz aufrechterhalten kann.
      \subsub{Struktur?}
      Transformations- 1. Kontrolle, 2. Eigenschaften, Bedingungen, 3. Funktionen\\
      Vorteile der Trennung?
      \subsub{Kritik}
      False-Tie puzzle? \it{A Generic Operation Transformation Scheme for Consistency Maintenance in Real-time Cooperative Editing Systems} und \it{Achieving convergence, causality-preservation, and intention-preservation in real-time cooperative editing systems}\\\\
      Während der klassische OT-Ansatz, Operationen durch ihre Versätze im Text zu definieren, einfach und natürlich zu sein scheint, werfen real verteilte Systeme ernsthafte Probleme auf. Nämlich, dass sich die Operationen mit endlicher Geschwindigkeit fortpflanzen, die Zustände der TeilnehmerInnen sind oft verschieden, so dass die resultierenden Kombinationen von Zuständen und Operationen extrem schwer vorherzusehen und zu verstehen sind.
      Wie Li und Li es ausdrückten: ``Aufgrund der Notwendigkeit, eine komplizierte Fallabdeckung in Betracht zu ziehen, sind formale Beweise sehr kompliziert und fehleranfällig, selbst für OT-Algorithmen, die nur zwei charakteristische Primitive behandeln (Einfügen und Löschen)``~\cite{ot-critic}.\\
      Damit OT funktioniert, muss jede einzelne Änderung an den Daten erfasst werden: ``Einen Schnappschuss des Zustands zu erhalten, ist normalerweise trivial, aber das Erfassen von Bearbeitungen ist eine ganz andere Sache. [...] Der Reichtum moderner Benutzerschnittstellen kann dies problematisch machen, besonders in einer browserbasierten Umgebung`` ~\cite{diff_sync}.
      \subsub{Konsitenzmodelle???  CC Modell, CCI, CSM, CA}