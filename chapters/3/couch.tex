\section{Das CouchDB Replikationsmodell}
\it{Das CouchDB Replikationsmodell erlaubt eine nahtlose, peer-to-peer (direkte) Datensynchronisation zwischen beliebig vielen Geräten. Das CouchDB Replication Protokoll ist in CouchDB selbst implementiert, dass die Serverkomponente abdeckt. Dann gibt es das PouchDB-Projekt, das dasselbe Protokoll in JavaScript implementiert, das auf Browser- und Node.js-Anwendungen abzielt. das deckt Ihre Kunden und dev-Server ab. Schließlich gibt es Couchbase Mobile und Cloudant Sync, die auf iOS und Android laufen und das CouchDB Sync-Protokoll in Objective-C bzw. Java implementieren.}
%
% Couch
%
\sub{CouchDB}
Vektoruhr~\footnote{\url{https://en.wikipedia.org/wiki/Vector_clock}} \\
content addressable versions: Idee: Nimm den Objektinhalt (content) und jag ihn durch eine \gls{Hashfunktion}\\
Apache Couch\tm
%
% Pouch
%
\sub{PouchDB}
PouchDB ist eine Open-Source-JavaScript-Datenbank, die so konzipiert wurde, dass sie im Browser läuft. PouchDB ermöglicht es Anwendungen zu erstellen, die sowohl offline als auch online funktionieren. Daten können lokal gespeichert werden, sodass alle Funktionen der Anwendung auch im Offline-Modus zur Verfügung stehen.
Daten werden unabhängig von der nächsten Anmeldung (des nächsten Onlinezugangs) zwischen \b{Clients}, CouchDB oder kompatiblen Servern synchronisiert.
PouchDB läuft auch in Node.js\footnote{JavaScript Laufzeitumgebung, steht unter \url{https://nodejs.org/en/download/} zum Download bereit} und kann als direkte Schnittstelle zu CouchDB-kompatiblen Servern verwendet werden~\cite{pouch}.\\\\