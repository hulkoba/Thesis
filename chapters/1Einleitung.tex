\chapter{\label{chap:einleitung}Einführung}
\begin{quote}
  We live in a disconnected \& battery powered world, but our technology and best practices are a leftover from the always connected \& steadily powered past.
  \cite{offlinefirst}
\end{quote}

Heutzutage besitzen mehr als fünf Milliarden Menschen ein Mobiltelefon und drei Milliarden haben Zugang zum Internet~\cite{dev-report}.\\\\

langsame Verbindungen, Unterbrechungen. Auch bei 3G und 4G ist die Latenz schrecklich (Lie-Fi?) -> Offline-First Apps können eine bessere User experience bieten.

\section{Motivation}
Ich möchte eine offlinefähige (mobile?) Anwendung entwickeln und stelle mir folgende Fragen. \\
Welche Software/ Framework benutze ich dazu?\\
Auf was muss ich bei der Auswahl achten?\\
Was erwarte ich von einer offline fähigen App?\\
(funktioniert und kein Datenverlust) --> Synchronisation und Konfliktmanagement
%\clearpage
\section{Zielstellung}
Wichtig: Offline nutzbar ohne Datenverlust.\\
Untersuchung des Verhaltens bei  Konflikten (verursacht durch paralleles Arbeiten ohne Internetverbindung).\\
Wie leicht/schwer ist es zu implementieren?
\note{konkret werden. Definition offlinefähig}
