\sub{\label{sub:detect}Verbindungsstatus feststellen und ändern}
Für die Überprüfung der Verbindung zum Server wird das Modus \sc{React Detect Offline} verwendet. Es beobachtet den Online-- und Offlinestatus und bietet zwei Komponenten entsprechend des Status den Inhalt rendern. Der folgende Codeausschnitt zeigt eine Verwendung dieser beiden Komponenten. Ist die Anwendung online, wird `you are online` gerendert. Im anderen Fall ~`you are offline`.
%
\begin{center}
\lstinputlisting[language=REACT,caption={Beispiel einer React Detect Offline Implementierung}, label=code:react-detect]{code/Header.js}
\end{center}
%
Das Modul fragt alle fünf Sekunden die URL \url{https://ipv4.icanhazip.com} ab und rendert je nach Verbindungsstatus die entsprechende Komponente. Verschiedene Parameter wie die URL oder das Poll--Interval können konfiguriert werden~\cite{react-detect}.\\\\
%
Der Verbindungsstatus kann im Browser geändert werden. Die Prototypen, die im Rahmen dieser Arbeit entwickelt werden, sollen in den Browsern Firefox und Chromium laufen.\\
In Firefox lässt sich der Netzwerkstatus über das Einstellungsmenü ändern. Dort kann man entweder unter dem Punkt `Sonstiges` oder dem Punkt `Web-Entwickler` `Offline arbeiten` auswählen und ist vom Internet getrennt. Dieser Status lässt sich über den selben Weg rückgängig machen.\\
In Chrome öffnet man dazu die Entwicklertools, geht auf `Netzwerk` und klickt auf die Checkbox `Offline` am oberen Rand. Dieselbe Checkbox ist auch im `Application`--Tab unter `Service Workers` zu finden.