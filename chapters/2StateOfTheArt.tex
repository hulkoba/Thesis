\chapter{\label{chap:state}Bestehende offlinefähige Systeme / Konzepte}
%
% Software
%
\section{Kollaborative Software}
  \highlight{rauslassen?}
  Kann ich benutzen um kollaborativ zu arbeiten --> \b{was passiert wenn ich offline bin?}
  \sub{Google Docs}
  (benutzt OT)
  \sub{Google Wave}
  (benutzt OT)
  \sub{Dropbox / 'Clouds'}
  \sub{Kollaborative Editoren}
  Wiki: \url{https://en.wikipedia.org/wiki/Collaborative_real-time_editor}\\\\
  \url{https://atom.io/packages/covalent}\\
  \url{https://atom.io/packages/firepad}\\
  Markdown: \url{https://hackmd.io/}\\
  LaTeX: \url{https://www.sharelatex.com/}\\
  Online editor: \url{http://etherpad.org/} (OT)\\
  Mockingbird (tool for creating wireframes): \url{https://gomockingbird.com/home} (OT)\\
  marvelapp? \url{https://marvelapp.com/collaboration/}

  --> Zahl steigend, (nachdem Google die Drive Realtime API veröffentlicht hat, die auf \gls{OT} basiert und es \it{third-party Apps} ermöglicht, dieselbe Zusammenarbeit wie Google Docs zu verwenden)\\\\
  \b{plus} wachsende Anzahl von offen zur Verfügung gestellter Bibliotheken und Frameworks die es ermöglichen, offlinefähige Anwendungen zu programmieren. Siehe Kapitel \ref{sec:frameworks}
%
% Frameworks / Bibliotheken
%
\section{\label{sec:frameworks}Offline-First Frameworks/Bibliotheken}
  Ich möchte aber auch eigenständig Software entwickeln die man vielleicht nicht nur zum Arbeiten nehmen kann, sondern auch um Quatsch zu machen wie Katzengifs zu teilen.
  \sub{redux? react-native? webpack?}
  \Gls{PWA}
  %
  % realm
  %
  \sub{Realm}
  %
  % Datomic
  %
  \sub{Datomic? (Closure)}
  \cite{realm}
  %
  % redux-offline
  %
  \sub{redux-offline}
  github~\cite{redux-offline-gh}
  hackernoon~\cite{redux-offline}
  %
  % redux persist
  %
  \sub{redux-persist}
  localStorage. github~\cite{redux-persist-gh} medium~\cite{redux-persist}
  %
  % react-native-offline
  %
  \sub{react-native-offline}
  github \cite{rn-offline-gh} medium\cite{rn-offline-medium}
  %
  % webpack offline-plugin
  %
  \sub{offline-plugin für webpack}
  github\cite{webpack-gh}
  dev\cite{webpack-dev}
  %
  % hoodie
  %
  \sub{hoodie}
  \cite{hoodie}