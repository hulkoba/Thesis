\chapter{\label{chap:state}Bestehende offlinefähige Systeme / Konzepte}
\section{Kollaborative Software}
Kann ich benutzen um kollaborativ zu arbeiten
\sub{Google Docs}
(benutzt OT)
\sub{Google Wave}
(benutzt OT)
\sub{Dropbox / 'Clouds'}
\sub{Kollaborative Editoren}
Wiki: \url{https://en.wikipedia.org/wiki/Collaborative_real-time_editor}\\\\
\url{https://atom.io/packages/covalent}\\
\url{https://atom.io/packages/firepad}\\
Markdown: \url{https://hackmd.io/}\\
LaTeX: \url{https://www.sharelatex.com/}\\
Online editor: \url{http://etherpad.org/} (OT)\\
Mockingbird (tool for creating wireframes): \url{https://gomockingbird.com/home} (OT)\\
marvelapp? \url{https://marvelapp.com/collaboration/}

--> Zahl steigend, nachdem Google die Drive Realtime API veröffentlicht hat, die auf \gls{OT} basiert und es \it{third-party Apps} ermöglicht, dieselbe Zusammenarbeit wie Google Docs zu verwenden.
\section{Offline-First Frameworks}
Ich möchte aber auch eigenständig Software entwickeln die man vielleicht nicht nur zum Arbeiten nehmen kann, sondern auch um Quatsch zu machen wie Katzengifs zu teilen.
\sub{redux-offline}
\sub{Realm}
\cite{realm}
\sub{hoodie}
\cite{hoodie}