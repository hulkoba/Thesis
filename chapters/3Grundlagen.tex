\chapter{\label{chap:grundlagen}Grundlagen}
Software, mit der ein freigegebenes Dokument zusammen mit anderen über das Internet bearbeitet werden kann, kann überaus wertvoll sein wenn man beispielsweise in einem Team arbeitet. Heutzutage gibt es viele webbasierte Software für simultanes kollaboratives Editieren \textit{(Textdokumente, Tabellenkalkulationen, Präsentationen, Quellcode)}. Im Kapitel \ref{chap:state} werden Produkte und Frameworks vorgestellt, mit deren Hilfe man solche Software erstellen kann.\\
Lassen Sie uns einen Moment nehmen, um genauer zu definieren, was wir unter dem Begriff kollaboratives Echtzeit-Editieren verstehen.
Wir möchten, dass mehrere Personen, die an verschiedenen Computern arbeiten, jederzeit Änderungen an einem auf einem Server gehosteten Dokument vornehmen können. Diese Änderungen werden sofort mit den anderen KollegInnen synchronisiert. Kein Client sollte vor einer Änderung mit dem Server oder einem anderen Client kommunizieren müssen. Insbesondere ist es nicht erforderlich, eine Sperre vom Server zu erhalten um ein Dokument zu bearbeiten und gleichzeitige Editierungen können auftreten. Nachdem alle Änderungen synchronisiert wurden, sollte jeder Client das exakt gleiche Dokument sehen.

%
% Offline-First
%
\section{Offline First}
1. Separate Apps from Data\\
2. Deliver App Code (and make it cachable) {appcache \& ServiceWorkers}\\
3. Save Data Offline {localStrogare / localForage, IndexedDB, other Wrapper}\\
4. Detect Connectivity {navigator.onLine} (Lie-fi)\\
5. Sync Data - Build upon existing solutions --CouchDB/PouchDB | remoteStorage\\
%
% Anforderungen
%
\sub{Anforderungen an Offline-First / PWA?}
1. Lese- und Schreibfähigkeit\\
2. Sync im Hintergrund\\
3. Cache sollte in der Lage sein App-Aktualisierungen zu ``überstehen``
\sub{Konflikte}
%
% Browserstuff
%
\section{Progressive Web Apps?}
\sub{ServiceWorker?}
\sub{localForage || AsyncStorage?}
\sub{IndexedDB?}
%
% Strategien
%
\section{Kollaboration in verteilten Systemen}
Es stellt sich heraus, dass die Implementierung dieser Art von Echtzeit-Zusammenarbeit alles andere als trivial ist.
Im Folgenden werden die drei Strategien \gls{OT}, \gls{CRDT} und \gls{LWW} vorgestellt.
%
% Operational Transform
%
\sub{Operational Transformation}
\gls{OT} ist eine weit verbreitete Technologie zur Unterstützung von Funktionalitäten in \gls{kollaborativ}er Software. Sie stammt aus einer im Jahre 1989 veröffentlichten Forschungsarbeit und wurde ursprünglich nur für die gemeinsame Bearbeitung von Klartext-Dokumenten entwickelt~\cite{ot_paper}. Später ermöglichte weitere Forschung \gls{OT} durch Unterstützung von \textit{Sperrungen, Konfliktlösungen, Benachrichtigungen, Bearbeitung von Baumstrukturierten Dokumenten, ...} zu verbessern und erweitern. Im Jahr 2009: Google Wave, Google Docs\\
% Ideen: einfach und mathematisch elegant
\highlight{Es wird das Problem untersucht, dass \gls{OT} in einer idealen Umgebung löst und dadurch zu einem funktionierendem Algorithmus gelangt.\\
Ziel: Mehrere BenutzerInnen können gleichzeitig an einem Dokument arbeiten, sehen Änderungen der anderen in Echtzeit (live), ohne dass einer Verzögerung durch die \gls{Netzwerklatenz} verursacht wird. Gleichzeitig auftretende Mehrfachänderungen sollen nicht zu unterschiedlichen Dokumentenzuständen führen.}
\subsub{Funktionsweise}
\Gls{kollaborativ}e Systeme, die \gls{OT} verwenden, benutzen normalerweise den replizierten Dokumentenspeicher. Das heißt jeder \b{Client} verfügt über eine eigene Kopie des Dokuments.
Jede Änderung an einem freigegebenen Dokument wird als Operation dargestellt. Operationen sind Repräsentationen von Änderungen an einem Dokument. (Beispielsweise: Füge 'Hello world!' an Position 0 in das Textdokument ein).  Eine Operation zeichnet im Wesentlichen den Unterschied zwischen einer und der nachfolgenden Version eines Dokuments auf. Die Anwendung einer Operation auf das aktuelle Dokument führt zu einem neuen Dokumentstatus.
Die Operationen erfolgen auf lokalen Kopie und die Änderungen werdenn an alle anderen \b{Clients} weitergegeben.  Wenn ein \b{Client} die Änderungen von einem anderen \b{Client} empfängt, werden die Änderungen normalerweise \b{vor} ihrer Ausführung transformiert.
\highlight{Die Transformation stellt sicher, dass anwendungsabhängige Konsistenzkriterien (Invarianten) von allen Standorten gepflegt werden. }\\

Es gibt die Operationen \b{Einfügen}\\
Das Einfügen besteht aus dem eingefügten Text und dessen Position im Dokument (\tt{insert('h', 0)}).
Für die Position kann ein Koordinatensystem ermittelt werden (Zeilennummer: Position in Zeile oder einfacher: Dokument wie eine Folge von Zeichen behandeln, also einfach einen nullbasierten Index vergeben.)\\
und \b{Löschen}\\
Löschen(5,6) = löscht 5 Zeichen, beginnend bei Position 6.
Mehr benötigt man nicht, denn update = delete \& insert\\\\
Um gleichzeitige Operationen zu behandeln, gibt es eine Funktion (normalerweise \texttt{Transform} genannt), die zwei Operationen übernimmt, die auf denselben Dokumentstatus angewendet wurden (aber auf verschiedenen Clients).
Daraus wird eine neue Operation berechnet, die nach der zweiten Operation angewendet werden kann. Diese behält die erste beabsichtigte Änderung der Operation.

Des Weiteren unterstützt \gls{OT} Operationen wie \tt{update}, \tt{point}, \tt{lock}.
\paragraph{Beispiel}:
Benutzer A fügt an Position 12 das Zeichen 'A' ein
Benutzer B fügt am Anfang des Dokuments ein 'B' ein.
Die konkurrierenden Operationen sind daher Einfügen (12, 'A') und Einfügen (0, 'B').
Wenn wir die Operation von B einfach an Client A senden und dort anwenden würden, gäbe es ein Problem.
Aber wenn die Operation von A an B gesendet, und angewandt wird nachdem Operation B angewandt wurde ist, würde das Zeichen 'A' eine Position zu weit links von der korrekten Position eingefügt werden.
Dokumentstatus A und Dokumentstatus B sind nicht identisch.\\
Daher muss A's \tt{insert(12, 'A')} gegen die Operation von B transformiert werden. So wird berücksichtigt, dass B ein Zeichen vor der Position 12 eingefügt hat (die die Operation \tt{insert(13, 'A')} erzeugt.)\\
Diese neue Operation kann auf Dokument B nach B's Operation angewandt werden.
Die Grundidee von \gls{OT} besteht darin, die Parameter einer Editieroperation gemäß den Auswirkungen \b{zuvor ausgeführter} konkurrierender Operationen anzupassen, so dass die transformierte Operation die korrekte Wirkung erzielen und die Dokumentenkonsistenz aufrechterhalten kann.
\subsub{Struktur?}
Transformations- 1. Kontrolle, 2. Eigenschaften, Bedingungen, 3. Funktionen\\
Vorteile der Trennung?
\subsub{Kritik}
False-Tie puzzle? \it{A Generic Operation Transformation Scheme for Consistency Maintenance in Real-time Cooperative Editing Systems} und \it{Achieving convergence, causality-preservation, and intention-preservation in real-time cooperative editing systems}\\\\
Während der klassische OT-Ansatz, Operationen durch ihre Versätze im Text zu definieren, einfach und natürlich zu sein scheint, werfen real verteilte Systeme ernsthafte Probleme auf. Nämlich, dass sich die Operationen mit endlicher Geschwindigkeit fortpflanzen, die Zustände der TeilnehmerInnen sind oft verschieden, so dass die resultierenden Kombinationen von Zuständen und Operationen extrem schwer vorherzusehen und zu verstehen sind.
Wie Li und Li es ausdrückten: ``Aufgrund der Notwendigkeit, eine komplizierte Fallabdeckung in Betracht zu ziehen, sind formale Beweise sehr kompliziert und fehleranfällig, selbst für OT-Algorithmen, die nur zwei charakteristische Primitive behandeln (Einfügen und Löschen)``~\cite{ot-critic}.\\
Damit OT funktioniert, muss jede einzelne Änderung an den Daten erfasst werden: ``Einen Schnappschuss des Zustands zu erhalten, ist normalerweise trivial, aber das Erfassen von Bearbeitungen ist eine ganz andere Sache. [...] Der Reichtum moderner Benutzerschnittstellen kann dies problematisch machen, besonders in einer browserbasierten Umgebung`` ~\cite{diff_sync}.
\subsub{Konsitenzmodelle???  CC Modell, CCI, CSM, CA}
%
% CRDT
%
\sub{Conflict-free replicated data type}
Die Idee von \glspl{CRDT} ist, dass jeder "Typ" (wie ein Einkaufswagen) mit Intelligenz handelt, um Konflikte automatisch zu lösen.\\\\
CRDTs sind Objekte, die ohne teure Synchronisation aktualisiert werden können. Sie konvergieren schließlich, wenn alle gleichzeitigen Aktualisierungen kommutativ\footnote{Unverändert bei Vertauschen der Operanden} sind und wenn alle Aktualisierungen schließlich von jeder Replik ausgeführt werden ~\cite{crdt_shapiro}.
Um diese Garantien zu geben, müssen diese Objekte bestimmte Kriterien erfüllen, welche im Folgenden beschrieben werden ~\cite{crdt_shapiro2}.
% In ihren Arbeiten betrachten Shapiro et. al zwei Replikationsmodelle in einem letztendlich konsistentem verteilten System: den zusatndsbasierten und den operationsbasierten Ansatz. Basierend auf dem Replikationsmodell definieren Sie zwei Arten von CRDTs: CvRDT (vonvergent replicated data type) und CmRDT (kommutativ replizierter Datantyp). Interessanterweise zeigen sie, dass diese beiden Replikationsmodelle und diese beiden Arten von CRDTs äquivalent sind.
\subsub{Zustandbasierter Ansatz}
Wenn ein Replikat ein Update von einem Client empfängt, aktualisiert es zuerst seinen lokalen Status und dann, einige Zeit später, seinen \b{vollständigen Status}. So sendet jedes Replikat gelegentlich seinen vollständigen Status an ein anderes Replikat im System. Um ein Replikat, das den Status eines anderen Replikats empfängt, wendet eine \b{Zusammenführungsfunktion} (merge) an, um den empfangenen Status mit dem lokalen Status zusammenzuführen. Entsprechend sendet dieses Replikat gelegentlich auch seinen Status an ein anderes Replikat, sodass jedes Update schließlich alle Replikate im System erreicht.
\subsub{Operationsbasierter Ansatz (OT?)}
Bei diesem Ansatz sendet ein Replikat seinen vollständigen Status (kann groß sein) nicht an ein anderes Replikat. Stattdessen sendet es nur den \b{Aktualisierungsvorgang} an \b{alle} anderen Replikate im System und erwartet von ihnen, dass sie das Update auf sich anwenden.\\
Da es sich um einen Sendevorgang handelt, wenn zwei Updates $u1$ und $u2$, bei einem Replikat $i$ angewendet werden und diese Updates an zwei Replikate $r1$ und $r2$ gesendet werden, können diese Updates in unterschiedlicher Reihenfolge bei diesen replikaten ankommen. $r1$ kann sie in der Reihenfolge $u1, u2$ empfangen, während bei $r2$ die Updates in umgekehrter Reihenfolge ($u2, u1$) ankommen können. Sind die Aktualisierungen \b{kommutativ}, können die Repliken zussamengeführt werden, egal in welcher Reihenfolge die Updates bei ihnen ankommen - der resultierende Zustand ist derselbe. In diesem Modell wird ein Objekt, für das alle gleichzeitigen Aktualisierungen kommutativ sind, CmRDT (commutative replicated data type - kommutativ replizierter Datentyp) genannt. \\\\
\b{Beispiel:...}\\
CRDTs befassen sich mit einem interessanten und grundlegendem Problem in verteilten Systemen, haben jedoch eine wichtige Einschränkung: "Da ein CRDT konstruktionsbedingt keinen Konsens verwendet, hat der Ansatz starke Einschränkungen; Dennoch sind einige interessante und nicht-triviale CRDTs bekannt" ~\cite{crdt_shapiro2}. Die Einschränkung ist, dass die CRDT-Adresse nur einen Teil des Problemraums betrifft, da nicht alle möglichen Aktualisierungsoperationen kommutativ sind und daher nicht alle Probleme in CRDTs umgewandelt werden können. Auf der anderen Seite können CRDTs für einige Arten von Anwendungen durchaus nützlich sein, da sie eine nette Abstraktion zur Implementierung repliziter verteilter Systeme bieten und gleichzeitig theoretische Konsistenzgarantien bieten.
%
% Last-write-wins
%
\sub{Last-Write-Wins (LWW) || Blockieren?}
NoSQL bietet dies, indem es einen Zeitstempel von irgendeiner Art beibehält, der ihnen hilft, zu entscheiden, welcher Schreibvorgang zuletzt kam. Datenbanken wie DynamoDB\footnote{\url{https://aws.amazon.com/de/dynamodb/faqs/What_is_a_readwrite_capacity_unit}} oder Cassandra \footnote{\url{https://docs.datastax.com/en/cassandra/3.0/cassandra/dml/dmlConfigConsistency.html}} verwenden LWW, um Schreibvorgänge zu verarbeiten. Bei diesen Leuten erfordern stark konsistente Schreibvorgänge das Schreiben in ein Quorum von Shards, wodurch Sie mehr Geld kosten.
%
% Couch Pouch
%
\section{CouchDB/PouchDB}
\sub{CouchDB}
\sub{PouchDB}
%
% Couch Pouch
%
