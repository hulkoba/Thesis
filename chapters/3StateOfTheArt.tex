\chapter{\label{chap:state}Bestehende offlinefähige Systeme / Konzepte}
\note{Harte, weiche, mittlere probleme (verschiedene Stufen von offlinefähig)}\\
\note{Native Apps sind offlinefähig}\\
bei nativen Apps ist das Problem bei der Datenverteilung
% %
% % Software
% %
% \section{Kollaborative Software raus?}
% Kann ich benutzen um kollaborativ zu arbeiten --> \b{was passiert wenn ich offline bin?}
% \sub{Google Docs}
% (benutzt OT)
% \sub{Google Wave}
% (benutzt OT)
% \sub{Kollaborative Editoren}
% Wiki: \url{https://en.wikipedia.org/wiki/Collaborative_real-time_editor}\\\\
% \url{https://atom.io/packages/covalent}\\
% \url{https://atom.io/packages/firepad}\\
% Markdown: \url{https://hackmd.io/}\\
% LaTeX: \url{https://www.sharelatex.com/}\\
% Online editor: \url{http://etherpad.org/} (OT)\\
% Mockingbird (tool for creating wireframes): \url{https://gomockingbird.com/home} (OT)\\
% --> Zahl steigend, (nachdem Google die Drive Realtime API veröffentlicht hat, die auf \gls{OT} basiert und es \it{third-party Apps} ermöglicht, dieselbe Zusammenarbeit wie Google Docs zu verwenden)\\\\
% \b{plus} wachsende Anzahl von offen zur Verfügung gestellter Bibliotheken und Frameworks die es ermöglichen, offlinefähige Anwendungen zu programmieren. Siehe Kapitel \ref{sec:frameworks}
%
% Frameworks / Bibliotheken
%
\section{\label{sec:frameworks}Offline-First Frameworks/Bibliotheken}
Ich möchte aber auch eigenständig Software entwickeln die man vielleicht nicht nur zum Arbeiten nehmen kann, sondern auch um Quatsch zu machen wie Katzengifs zu teilen.
\sub{git?}
%
% realm
%
\section{\label{sub:realm}Realm}
Realm ist eine Backendtechnologie für mobile Anwendungen und umfasst die Realm Datenbank und den Realm Object Server. Beide Technologien sind quelloffen, jedoch nicht kostenfrei~\cite{realm}.\\
Die Realm Datenbank ist eine objektorientierte, plattformübergreifende lokale Datenbank die eine Echtzeitsynchronisation mit dem Realm Object Server.
Der Object Server fungiert als \gls{Middleware}-Komponente in der mobilen \gls{App} und handhabt unter anderem die Ereignisbehandlung und Datensynchronisation. Im Zusammenspiel ermöglichen die beiden Technologien die Erstellung von offlinefähigen, kollaborativen, mobilen Anwendungen~\cite{realm_whitepaper}.\\\\
Zur Offline First Funktionalität stellt Realm eine umfassende Lösung bereit.
Die lokale Realm Datenbank unterstützt die Echtzeitsynchronisation von Daten sodass alle Änderungen sofort automatisch gesendet werden. Das Synchronisationsprotokoll komprimiert statt dem gesamten Objekt nur die marginalen Änderungen und synchronisiert sie auf dem Endgerät und dem Server. Zusätzlich zu den Daten werden die spezifischen Operationen erfasst. \todo{Siehe OT}. 
Wird beispielsweise ein Kontakt bearbeitet, wird neben den geänderten Daten die Information \it{update} mitgesendet.
Dank dieser zusätzlichen Information kann der Aktionswunsch genau erfasst werden sodass das System eventuelle Konflikte automatisch auflösen kann. Das hat zur Folge, dass die Synchronisation keinen manuellen Eingriff bedarf, der die Leistung des Systems beeinträchtigen könnte.
Zusätzlich zu dem \gls{OT} Algorithmus benutzt Realm vorgegebene Regeln zur automatischen Konfliktlösung. Es besteht die Möglichkeit eigene Regeln zu definieren.\\
Darüber hinaus passiert die interne Konfliktlösung auf Transaktionsebene. Das heißt der Vorgang ist nur erfolgreich wenn er auch vollständig und fehlerfrei ist. Andererseits wird er zurückgesetzt. Das gewährleistet die Konsistenz der Daten und verhindert deren Verlust wenn Änderungen aufgrund einer unterbrochenen Netzwerkverbindung nicht stattfinden können~\cite{realm_offline_whitepaper}.
%
% redux-offline
%
\section{\label{sub:reduxoffline}Redux Offline}
%``Persistenter Redux store für \it{reasonaboutable}\tm ~Offline-First Anwendungen``. \\
Redux Offline kann nur zusammen mit Redux verwendet werden~\cite{redux-req}. Deswegen ist für die Verwendung von Redux Offline die Implementierung von Redux eine Voraussetzungsbedingung.
%
% Redux
%
% Alles beginnt mit dem Aufruf von \tt{store.dispatch(action)} von jeder beliebigen Stelle in der Anwendung. Die Aktion die beschreibt was passiert heißt \tt{toggleEdit} und sieht im Beispiel des Ansichtswechsels aus wie in Zeile drei bis fünf.\\
% Der \tt{Store} ruft nun den \tt{Reducer} auf
\sub{\label{chap:redux}Redux}
Redux ist eine JavaScript Bibliothek, die Probleme im Zusammenhang mit dem Zustand, dem sogenannten \gls{App}state einer Anwendung, löst.
Redux ist eine Bibliothek zur Zustandsverwaltung in JavaScriptanwendungen.
Es gibt einen zentralen Ort, in dem der Zustand der App gespeichert ist, auf den von jeder Komponente aus zugegriffen werden kann.
Dieser Ort wird Store genannt und jede Applikation hat genau einen davon. 
Als einzige Informationsquelle für den Store als zentralen Speicher dienen Aktionen. 
Aktionen beschreiben was passiert ist und senden Daten von der Anwendung an den Store.
Eine Aktion könnte beispielsweise die Information beinhalten, dass es eine Aktualisierung gab und auf welchem Objekt diese Aktualisierung stattgefunden hat.
Der dritte wichtige Bestandteil von Redux sind die Reducer. Sie spezifizieren, wie der \gls{App}state sich als Reaktion auf die Aktionen ändert~\cite{redux}.\\
Der Datenfluss in der Reduxarchitektur ist unidirektional. Zur Veranschaulichung wird anhand der folgenden Abbildung der Redux Datenfluss beschrieben.
%
\begin{figure}[H]
  \centering
  \includegraphics[width=0.8\textwidth]{redux-flow}
  \grayRule
  \caption{Redux Datenfluss}
  \label{fig:rdx-dataflow}
\end{figure}
% 
Zuerst sendet die View eine Aktion an den Store. Dieser empfängt die Aktion und schickt sie zusammen mit dem Applikationsstatus an den Reducer.
Der Reducer erstellt eine Kopie des Status, verändert diese und schickt sie wieder zurück an den Store.
Der Store ersetzt nun den alten mit dem neuen Status und löst ein erneutes Rendern der View aus ~\cite{reduxflow}.
% 
% Redux Offline
% 
\sub{Redux Offline}
Redux Offline erweitert Redux um einen persistenten Store mit Offline-First Technologie und ist kompatibel mit allen *View Frameworks wie React\footnote{JavaScript Bibliothek: \url{https://reactjs.org/}}, Vue\footnote{JavaScript Framework: \url{https://vuejs.org/}}, oder Angular\footnote{JavaScript Framework: \url{https://www.angular.io}}~\cite{redux-offline-compabilaty}.
Es umfasst unter anderem netzwerkfähige \gls{API}-Aufrufe, das Persistieren des Zustands der Anwendung, das Speichern von Aktionen, die Behandlung von Fehlern und erneute Versuche, die Verbindung wieder herzustellen.
Redux Offline verspricht nicht, die Webanwendung komplett offlinefähig zu machen. Um \gls{Assets} zwischenzuspeichern, muss zusätzlich noch ein ServiceWorker implementiert sein ~\cite{redux-offline-gh}.\\
Die Idee hinter Redux Offline ist, dass der Redux Store die Datenbank ersetzt~\cite{redux-offline}.
Bei jeder Änderung wird der Redux Store auf dem Datenträger gespeichert, und bei jedem Start automatisch neugeladen. Für das Speichern der Daten in einer lokalen Datenbank wird intern \hyperref[sub:reduxpersist]{Redux Persist} verwendet.\\\\
%
%
Eine mit Redux Offline erstellte Anwendung funktioniert ohne weitere Codeimplementierung offline im Lesemodus, da das Lesen und Schreiben aus der lokalen Datenbank bereits eingebunden ist.
Damit die Anwendung auch im Schreibmodus offline funktioniert, müssen einige Anpassungen vorgenommen werden.
Sämtliche Daten der Anwendung können nur über Aktionen manipuliert werden. 
Alle netzwerkgebundenen Aktionen werden in einem storeinternem \gls{Queue} gespeichert und müssen mit einem Metaattribut dekoriert werden, um offline arbeiten zu können. Durch die Metaattribute weiß die Anwendung, was vor der eigentlichen Ausführung der Aktion und was danach zu tun ist. 
Es gibt drei Metadaten, die Redux Offline interpretieren kann:\\
\tt{meta.offline.effect} - Die initiale Aktion wird ausgeführt. Hier kann eine URL angegeben werden, die Redux Offline anfragen soll.\\
\tt{meta.offline.commit} - Hier wird die Aktion definiert, die ausgeführt wird sobald die Netzwerkanfrage erfolgreich ist.\\
\tt{meta.offline.rollback} - Hier kann die Aktion angegeben werden, die bei permanent fehlgeschlagener Internetverbindung oder wenn der Server einen Serverfehler zurückgibt, gefeuert wird.
Dann fügt Redux Offline dem \gls{App}state automatisch ein \tt{offline} Objekt hinzu. Dort wird unter anderem ein Array namens \tt{outbox} verwaltet.
Dieses Array repräsentiert den \gls{Queue}. Hier werden die Aktionen inklusive Metadaten gespeichert, um bei bestehender Internetverbindung abgearbeitet zu werden~\cite{redux-offline-docs}.
Die von Jani Eväkallio erstellte Grafik \ref{fig:redux-offline} veranschaulicht die oben erklärte Architektur.
%
\begin{figure}[h]
  \centering
  \includegraphics[width=0.8\textwidth]{redux-offline-new}
  \grayRule
  \caption[Redux Offline]{Redux Offline Architektur~Quelle:~\cite{redux-offline}}
  \label{fig:redux-offline}
\end{figure}
%
Links ist eine Aktion zu sehen, die etwas machen möchte. Sie hat ein Metaattribut das weitere Aktionen definiert, eine Aktion für den Erfolg und eine für den Fehlschlag von `DO\_STUFF`.
In der Mitte ist der Store zu sehen.
Der Store kennt den Netzwerkstatus und umfasst den \gls{Queue} namens \tt{outbox} in dem Aktionen mitsamt ihrer Metafelder gespeichert werden.
Rechts befindet sich das \gls{API}, das über die \gls{Middleware} mit dem Store kommuniziert.\\
Wird die Aktion 'DO\_STUFF' gefeuert gelangt sie in den Store, damit dieser den \gls{App}State aktualisieren kann und wird ersteinmal im \gls{Queue} gespeichert.
Ist die Anwendung online, wird sie sofort abgearbeitet.
Wenn nicht, wird sie im \gls{Queue} gespeichert bis die Anwendung wieder eine Verbindung zum Internet hat.
% \subsub{Konflikte}
%
% redux persist
%
\sub{\label{sub:reduxpersist}Redux Persist}
Redux Persist ist eine Bibliothek, die als Wrapper für den Redux Store funktioniert.
Mit Redux Persist wird der State automatisch lokal, per default im LocalStorage, gespeichert~\cite{redux-persist}.
Es kann konfiguriert werden wo die Daten gespeichert werden.
Hierfür gibt es diverse Möglichkeiten, wie zum Beispiel im SessionStorage, per localForage oder in Dateisystemen~\cite{redux-persist-gh}.
LocalForage ist eine Bibliothek, mit der Daten in IndexedDB oder WebSQL gespeichert werden können.
Wenn der Browser die Speichermöglichkeiten nicht unterstützt, wird der LocalStorage genommen~\cite{localforage}.\\
Es ist auch möglich einen eigenen Speicher zu konfigurieren.
Die einzige Voraussetzung hierfür ist, das \gls{API} muss die Standardmethoden \tt{setItem}, \tt{getitem} und \tt{removeItem} implementieren und Promises unterstützen~\cite{redux-persist-gh}.
%
% redux optimist
%
% \sub{redux-optimist}

%
% react-native-offline
%
\section{React Native Offline}
React Native ist ein Framework mit dem native, mobile Apps mit JavaScript und React gebaut werden können~\cite{rn}.\\
Die Bibliothek React Native Offline erweitert das Framework um Offlinefunktionalität.
React Native Offline unterstützt die Behandlung des Netzwerkstatus.
Dieser kann einmalig oder regelmäßig abgefragt werden, um je nach Status z.B. einen anderen Inhalt zu rendern.\\
Zusammen mit Redux implementiert, können weitere Fähigkeiten genutzt werden.
Genau wie Redux Offline hat React Native Offline einen Offline \gls{Queue}, in dem Aktionen gespeichert werden können.
Allerdings nur solche Aktionen, die fehlgeschlagen sind weil die Anwendung nicht mit dem Internet verbunden ist.
Auch hier wird der Aktion ein Metaattribut gegeben. Dieses hat die Felder \tt{retry} und \tt{dismiss}.
Das erste Feld erwartet ein Boolean. Hier kann angegeben werden ob die Aktion noch einmal bei bestehender Internetverbindung ausgeführt werden soll oder nicht.
Das Feld \tt{dismiss} erwartet ein Array. Hier können Aktionen angegeben werden, die das Wiederholen der Aktion abbrechen~\cite{rn-offline-gh}.\\\\
Ein Beispiel im folgenden Listing soll die Funktionsweise der Metaattribute besser beschreiben.
%
\lstset{language=REACT,
  caption={Beispiel einer React Native Offline Aktion},
  label={code:rno}}
\begin{lstlisting}
const action = {
  type: 'FETCH_CONTACT',
  contact,
  meta: {
    retry: true,
    dismiss: ['CANCEL']
  }
}
\end{lstlisting}
%
Bei Aufruf dieser Aktion soll ein Kontakt geladen werden.
Im Metafeld ist \tt{retry} auf \tt{true} gesetzt.
Wurde die Aktion im Offlinemodus versucht auszuführen, wird sie erneut aufgerufen sobald die Anwendung wieder einen Internetzugang hat.
Die Aktionen die in \tt{dismiss} angegeben werden, unterbrechen diesen Vorgang.
Wurde also der Kontakt im Offlinemodus angefragt und dann die Aktion ''CANCEL'' gefeuert, wird ''FETCH\_CONTACT'' aus dem Queue gelöscht und nicht erneut ausgeführt.


%
% webpack offline-plugin
%
\sub{offline-plugin für webpack}
webpack ist ein JavaScript `Bundler`, packt JavaScript-Dateien und oder \gls{Assets} für die Verwendung in Browsern.blabla\\
Bietet offline experience für webpack Projekte. Benutzt \sc{ServiceWorker} und \sc{AppCache} unter der Haube --> cached nur (gebündelten) von webpack generierten \gls{Assets}. Für die anderen Dateien (z.B. index.html die nicht gebundled wird oder Modul von CDN) braucht mal ein \tt{html-plugin} oder benutzt die \tt{externals} :\\\\
\tt{const OfflinePlugin = require('offline-plugin')\\
  const offline = new OfflinePlugin\\\\
  const offline = new OfflinePlugin({\\
    externals: ['index.html'],\\
    })}\\
Es gibt diverse config-options für \sc{ServiceWorker} und AppCache...~\cite{webpack-gh}
% dev\cite{webpack-dev}


%
% hoodie
%
\section{hoodie}
Hoodie ist eine JavaScript Bibliothek für offlinefähige Webapplikationen, die ein komplettes Backend zur Verfügung stellt. Wird Hoodie für die Entwicklung einer Webanwendung verwendet, muss also lediglich das Frontend implementiert werden. Den Rest erledigt die Bibliothek. Über eine integrierte Programmierschnittstelle kommuniziert die Anwendung mit dem von Hoodie zur Verfügung gestelltem Backend. Über das \gls{API} können unter Anderen BenutzerInnen authentifiziert, Daten gespeichert und synchronisiert werden~\cite{hoodie}.\\
Anhand der Abbildung \ref{fig:hoodie} wird erklärt wie Hoodie funktioniert.
\begin{figure}[H]
  \centering
  \includegraphics[width=0.8\textwidth]{hoodie}
  \grayRule
  \caption[Hoodie Architektur]{Hoodie Architektur~Quelle:~\cite{hoodie-how}}
  \label{fig:hoodie}
\end{figure}
Im Frontend--Bereicht ist die App zu sehen die über das Hoodie \gls{API} mit dem lokalen Speicher kommuniziert. Die Anwendung spricht niemals direkt mit dem Server oder der Datenbank. Für die lokale Speicherung der Daten benutzt Hoodie intern PouchDB, was wiederum IndexedDB verwendet. Durch das lokale Speichern sind die Daten auch offline verfügbar. Dann werden über eine \gls{REST} Schnittstelle mit einer CouchDB synchronisiert. CouchDB ist eine Datenbank mit der Superkraft des Synchronisierens und in Hoodie haben alle AnwenderInnen ihre eigene private CouchDB. Hinter der Datenbank befindet sich ein kleiner Server der auf die Daten in der CouchDB reagiert, die wiederum die Änderungen an den Client schickt ~\cite{hoodie-how}.
So können NutzerInnen nur auf ihre eigenen Daten zugreifen. Wenn es mehrere Geräte gibt, die mit einem Account assoziiert werden, werden die Änderungen von einem Gerät zuerst auf die serverseitige CouchDB synchronisiert, um dann von dort in die lokalen Datenbanken der anderen Geräte zu gelangen.\\
Dadurch dass das Frontend und das Backend nicht direkt miteinander sprechen, ist die Funktionalität beider Komponenten auch dann gewährleistet, wenn die Verbindung unterbrochen wird.
%
% Couch
%
\sub{\label{sec:couch}CouchDB}
Apache CouchDB\tm ist ein \gls{DBMS} das seit 2005 als freie Software entwickelt wird. Die dokumentenorientierte \gls{DB} funktioniert sowohl als einzelne Instanz, als auch im Cluster, in dem ein Datenbanksserver auf einer beliebig großen Anzahl an Servern oder \glspl{VM} ausgeführt werden kann. So kann die Datenschicht beliebig skaliert werden, um die Anforderungen vieler BenutzerInnen zu erfüllen. CouchDB verwendet das \gls{HTTP}--Protokoll und \gls{JSON} als Datenformat, weswegen es mit jeder Webfähigen Anwendung kompatibel ist. CouchDB wird über ein \gls{REST}ful \gls{HTTP} \gls{API} angesprochen. Mit den für \gls{REST}ful Services standardisierten Methoden z. B. GET, POST, PUT, DELETE können die Daten abgerufen und. manipuliert werden.\\
Das implementierte Replikationsmodell erlaubt die Synchronisation bzw. bidirektionale Replikation zu verschiedenen Geräten ist genau die Besonderheit, die CouchDB als eine Offline--Datenbank auszeichnet. Dessen Funktionsweise wird in \autoref{sec:replication} detailliert beschrieben. Dieses Protokoll ist die Grundlage für Offline First Anwendungen.
Das Replikations-API von CouchDB bietet die Möglichkeit, eine Datenbank kontinuierlich oder selbstgesteuert mit einer anderen zu synchronisieren.
So kann beispielsweise eine CouchDB-Instanz auf dem Mobiltelefon und eine auf dem Laptop bestehen und beide können sich bei bestehender Internetverbindung synchronisieren. Da so die gespeicherten Daten aus dem lokalen Speicher gelesen werden, sind ein schnelles Interface und eine geringe Latenz die positive Folge. Wenn Konflikte auftreten, beispielsweise durch gleichzeitiges Bearbeiten eines Dokuments von zwei Personen ohne Netzwerkverbindung, werden diese als solche markiert, jedoch nicht von selbst aufgelöst. So gehen keine Daten verloren und es liegt an der benutzenden Person diese zu lösen. \todo{Referenz git?}\\
CouchDB ist für Server konzipiert. Für Browser gibt es \hyperref[sub:pouch]{PouchDB} und für native iOS- und Android--\glspl{App} wurde Couchbase Lite entwickelt. Alle können Daten miteinander replizieren und verwenden das CouchDB Replikationsprotokoll~\cite{couch}.
\todo{CouchDB Replikationsmodell hier?}
%
% Pouch
%
\sub{\label{sub:pouch}PouchDB}
\todo{Was benutzt Poch wann? -- IndexedDB, Web SQL, LevelDB}
Als Ergänzung zu CouchDB kann PouchDB verwendet werden. PouchDB ist eine Open--Source--JavaScript--Datenbank, die so konzipiert wurde, dass sie im Browser läuft. PouchDB ermöglicht es Anwendungen zu erstellen, die sowohl offline als auch online funktionieren. Daten können lokal gespeichert werden, sodass alle Funktionen der Anwendung auch im Offline--Modus zur Verfügung stehen.
Daten werden unabhängig von der nächsten Anmeldung (des nächsten Onlinezugangs) zwischen \b{Clients}, CouchDB oder kompatiblen Servern synchronisiert.
PouchDB läuft auch in Node.js\footnote{JavaScript Laufzeitumgebung, steht unter \url{https://nodejs.org/en/download/} zum Download bereit} und kann als direkte Schnittstelle zu CouchDB--kompatiblen Servern verwendet werden~\cite{pouch}.
