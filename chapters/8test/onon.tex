In der ersten Testreihe werden Aktionen ausgeführt, während beide Geräte mit dem Internet verbunden sind.
Diese Tests zeigen, dass die Anwendung grundsätzlich auf mehreren Geräten einsatzbereit ist.
%
\subsub{Kontakt anlegen}
Ein Kontakt mit dem Namen ''Amilia Pond'' wird in beiden Browsern angelegt.\\\\
\b{Erwartetes Ergebnis}\\
Der Kontakteintrag mit dem Namen ''Amilia Pond'' existiert zwei mal auf jedem Gerät.\\
\b{Tatsächliches Ergebnis \it{amilia-qouch}}\\
Der Kontakteintrag mit dem Namen ''Amilia Pond'' existiert zwei mal auf jedem Gerät.\\
\b{Tatsächliches Ergebnis \it{amilia-rdx}}\\
Der Kontakteintrag mit dem Namen ''Amilia Pond'' existiert zwei mal auf jedem Gerät.
%
\subsub{Kontakt bearbeiten}
Ein Kontakt mit dem Namen ''Amilia Pond'' wird in beiden Browsern bearbeitet.
Zuerst wird im Firefox die E-Mail Adresse ''amilia@pond.de'' hinzugefügt und gespeichert.
Dann wird im Chrome die E--Mail Adresse des Kontakts auf ''amilia.pond@rory.de'' geändert und gespeichert.\\\\
\b{Erwartetes Ergebnis}\\
Ein Kontakteintrag mit dem Namen ''Amilia Pond'' hat die E--Mail Adresse ''amilia.pond@rory.de''.
\b{Tatsächliches Ergebnis \it{amilia-qouch}}\\
Ein Kontakteintrag mit dem Namen ''Amilia Pond'' hat die E--Mail Adresse ''amilia.pond@rory.de''.
\b{Tatsächliches Ergebnis \it{amilia-rdx}}\\
Ein Kontakteintrag mit dem Namen ''Amilia Pond'' hat die E--Mail Adresse ''amilia.pond@rory.de''.
%
\subsub{Kontakt löschen}
Der Kontakt mit dem Namen ''Amilia Pond'' ohne E--Mail Adresse, wird im Firefox gelöscht. Der andere Adressbucheintrag wird im Chrome gelöscht.\\\\
\b{Erwartetes Ergebnis}\\
Die Adressliste ist leer. Es ist kein gespeicherter Kontakt vorhanden.\\
\b{Tatsächliches Ergebnis \it{amilia-qouch}}\\
Die Adressliste ist leer. Es ist kein gespeicherter Kontakt vorhanden.\\
\b{Tatsächliches Ergebnis \it{amilia-rdx}}\\
Die Adressliste ist leer. Es ist kein gespeicherter Kontakt vorhanden.