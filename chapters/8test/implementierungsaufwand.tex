%
Setup der \gls{App} ist irrelevant und identisch wegen Create React App.
%
%
\subsub{Einbinding der Technologien}
\begin{description}[leftmargin=0.5cm,style=nextline]
  \item[amilia-qouch]
  gering, nur Installation und Anwendung der Pouch API\\
  Neue Instanz, Synchronisation ( 1 Zeile) und CRUD
  \item[amilia-rdx]
  komplex, da Redux eingesetzt und verstanden werden muss. Für komplexere Apps mag das generell besser sein, aber für den einfachen Anwendungsfall zu viel Overhead. Einbindung von Redux Offline war allerdings wenig aufwändig sobald Redux implementiert war
\end{description}
%
\subsub{Lesbarkeit des Codes}
\begin{description}[leftmargin=0.5cm,style=nextline]
  \item[amilia-qouch]
  nicht mehr als erwartet und auch verständlich wenn man mit den Technologien nicht vertraut ist. Es liest sich so wie man es verwendet dank unmissverständlicher API.
  %
  \item[amilia-rdx] 
  Um den Code ohne Probleme lesen zu können muss man mit Redux vertraut sein. Ist das der Fall, ist trotzdem ein Blick in die Dokumentation von Redux Offline hilfreich, weil die Metaattribute nicht für sich selbst sprechen (effect, commit, rollback).
\end{description}
%
  \subsub{Menge des geschriebenen Quellcodes}
  Beide Prototypen wurden mit React Create App erstellt. Die Menge des Codes die dadurch erzeugt wurde, wurde nicht mitgezählt. Genausowenig wie Kommentare und das geschriebene \gls{CSS}.
  \begin{description}[leftmargin=0.5cm,style=nextline]
    \item[amilia-qouch]
    Dem \it{amilia-qouch} Prototypen wurden 371 Zeilen Code hinzugefügt.
    \item[amilia-rdx] 
    Dem \it{amilia-rdx} Prototypen wurden 586 Zeilen Code hinzugefügt.
\end{description}

% D1 Ich als EntwicklerIn möchte die Daten lokal und auf dem Server speichern, um deren Erreichbarkeit unabhängig vom Internetstatus zu gewährleisten.
% D2 Ich als EntwicklerIn möchte ich nur die Adressbucheinträge oder deren Aktualisierungen laden, die sich nicht schon auf dem Endgerät befinden, um Datentraffic und Ladezeiten zu sparen.
% D3 Ich als EntwicklerIn möchte ich jeden Eintrag identifizieren, um jedem Adressbucheintrag Operationen zuzuweisen und einzelne Kontakte zu finden.
% D4 Ich als EntwicklerIn möchte, dass jeden Eintrag versioniert ist, um zu wissen ob wann ein Eintrag bearbeitet wurde.
% D5 Ich als EntwicklerIn möchte, dass alle von NutzerInnen vorgenommenen Änderungen beim System ankommen und keine Daten verloren gehen.
% D6 Ich als EntwicklerIn möchte auftretende Konflikte effizient speichern, um mit ihnen umgehen können. Mit ihnen umgehen heißt: selbstständig oder von User lösen, zum konfliktfreien Zustand gelangen
% D7 Ich als EntwicklerIn möchte eine Technologie verwenden die leicht zu verstehen und implemenieren ist, um den Arbeitsaufwand gering zu halten.
% D8 Ich als EntwicklerIn möchte sauberen und verständlichen Code schreiben, um die Les– und Wartbarkeit zu erhöhen.