% D7, D8
Beim Vergleich des Implementierungsaufwands werden verschiedene Punkte berücksichtigt.
Es wird der Arbeitsaufwand betrachtet, der benötigt wird die verwendete Technologie zu verstehen und zu benutzen.
Ebenfalls von hoher Wichtigkeit ist die Lesbarkeit des geschriebenen Quellcodes.
Sauberer und verständlicher Code ist einfacher les-- und somit auch wartbarer. Interessant ist auch die Menge des geschriebenen Codes.
%
%
\subsub{Einbindung der Technologien}
\begin{description}[leftmargin=0cm,style=nextline]
  \item[amilia-qouch]
  Der Aufwand, eine Anwendung mit PouchDB und CouchDB zu schreiben, ist gering.
  Man muss nur beide Datenbanken installieren und im Code instanziieren.
  PouchDB dient als Schnittstelle zu CouchDB, weswegen nur das simpel gehaltene \gls{API} von PouchDB benutzt wird.
  Die Dokumentationen beider Technologien sind sehr ausführlich und tragen ausgesprochen gut zum Verständnis der Funktionsweise bei.

  \item[amilia-rdx]
  Der Aufwand, eine Anwendung mit Redux Offline zu schreiben, ist deutlich komplexer, da Redux eingesetzt und verstanden werden muss.
  Die Einbindung von Redux Offline war allerdings nur minimal aufwändig, sobald Redux einmal implementiert war.
  Redux ist sehr gut dokumentiert und bietet viele Beispiele zur Anwendung.
  Die Dokumentation von Redux Offline besteht aus einer verschachtelten Readme Datei, was die Lesbarkeit und somit das Verständnis der Funktionsweise der Technologie erschwert.
\end{description}
%
%
\subsub{Lesbarkeit des Codes}
\begin{description}[leftmargin=0cm,style=nextline]
  \item[amilia-qouch]
  Auch wenn man mit den Technologien nicht vertraut ist, lässt sich der Code problemlos lesen.
  Für die Synchronisation beider Technologien reicht eine Zeile Code und umgesetzte \gls{CRUD}--Operationen sind dank unmissverständlicher API eindeutig nachvollziehbar.
  %
  \item[amilia-rdx]
  Um den Code ohne Probleme lesen zu können, sollte man mit Redux und dessen Datenfluss vertraut sein.
  Ist das der Fall, ist ein Blick in die Dokumentation von Redux Offline hilfreich, da die verwendeten Metaattribute der Aktionen nicht unbedingt eindeutig sind.
  Sind diese Dinge klar, ist auch der Code von \it{amilia-rdx} gut lesbar und begreiflich.
\end{description}
%
  \subsub{Menge des geschriebenen Quellcodes}
  Beide Prototypen wurden mit React Create App erstellt.
  Die Menge des hieraus entstandenen Codes, wurde nicht mitgezählt. Ebensowenig wurden die Kommentare und das geschriebene \gls{CSS} gezählt.
\begin{description}[leftmargin=0cm,style=nextline]
    \item[amilia-qouch]
    Es wurden 388 Zeilen Code hinzugefügt.
    \item[amilia-rdx]
    Es wurden 544 Zeilen Code hinzugefügt.
\end{description}
