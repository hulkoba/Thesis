In der dritten Testreihe werden Aktionen ausgeführt, während beide Browser mit dem Internet, aber nicht mit dem Server verbunden sind.
Nachdem in beiden Browsern die beschriebene Aktion vollendet wurde, wird der Server wieder gestartet, sodass die Daten synchronisiert werden können. 
%
\subsub{Kontakt anlegen}
Ein Kontakt mit dem Namen ''Amilia Pond'' wird in beiden Browsern angelegt.\\\\
\b{Erwartetes Ergebnis}\\
Es existieren zwei Kontakteinträge mit dem Namen ''Amilia Pond'' auf jedem Gerät.\\
\b{Tatsächliches Ergebnis \it{amilia-qouch}}\\
Es existieren zwei Kontakteinträge mit dem Namen ''Amilia Pond'' auf jedem Gerät.\\
\b{Tatsächliches Ergebnis \it{amilia-rdx}}\\
Es existieren zwei Kontakteinträge mit dem Namen ''Amilia Pond'' auf jedem Gerät.
%
\subsub{Kontakt bearbeiten}
Ein Kontakt mit dem Namen ''Amilia Pond'' wird in beiden Browsern bearbeitet. Im Firefox wird dem Kontakt die E-Mail Adresse ''amilia@pond.de'' hinzugefügt und gespeichert.
Im Chrome wird dem Kontakt die E--Mail--Adresse ''amilia.pond@rory.de'' gegeben und gespeichert.\\\\
\b{Erwartetes Ergebnis}\\
Es entsteht ein Konflikt, der von einem Menschen gelöst werden kann.\\
\b{Tatsächliches Ergebnis \it{amilia-qouch}}\\
Ein Konflikt ist entstanden und wurde gespeichert. Ein Dialog öffnet sich mit allen notwendigen Informationen.
Es kann entschieden werden, welche E--Mail--Adresse übernommen werden soll.\\
\b{Tatsächliches Ergebnis \it{amilia-rdx}}\\
Der Kontakt hat die E-Mail Adresse ''amilia@pond.de''.
%
\subsub{Kontakt bearbeiten und denselben Kontakt löschen}
Der Kontakt mit dem Namen ''Amilia Pond'' wird in einem Browser bearbeitet, in dem anderen gelöscht.
Zuerst wird im Firefox der Name des Kontakts zu ''Rory Pond'' geändert und gespeichert.
Danach wird der Kontakt im Chrome gelöscht.\\\\
\b{Erwartetes Ergebnis}\\
Auf beiden Geräten ist ein Kontakt mit dem Namen ''Rory Pond'' gespeichert.\\
\b{Tatsächliches Ergebnis \it{amilia-qouch}}\\
Auf beiden Geräten ist ein Kontakt mit dem Namen ''Rory Pond'' gespeichert.\\
\b{Tatsächliches Ergebnis \it{amilia-rdx}}\\
Die Kontaktliste ist leer.
%
\subsub{Kontakt löschen}
Der Kontakt mit dem Namen ''Amilia Pond'' wird in beiden Browsern gelöscht.\\\\
\b{Erwartetes Ergebnis}\\
Die Kontaktliste ist leer.\\
\b{Tatsächliches Ergebnis \it{amilia-qouch}}\\
Die Kontaktliste ist leer.\\
\b{Tatsächliches Ergebnis \it{amilia-rdx}}\\
Die Kontaktliste ist leer.