  %
\sub{Auswertung der manuellen Tests}
Zur Auswertung der Testergebnisse werden die Testreihen gruppiert.
Die zweite und dritte Testreihe befassen sich mit der Offlinefähigkeit der Prototypen.
Es spielt hierbei keine Rolle, ob ein oder zwei der Geräte während der Testdurchläufe offline sind. Deswegen werden diese Testreihen in der Auswertung zusammengefasst.
%
%
%
\subsub{\label{chap:F1}Kollaborationsfähigkeit}
In der ersten Testreihe sind beide Geräte die ganze Zeit mit dem Internet verbunden.
Das Hauptaugenmerk lag auf der allgemeinen Funktionalität des jeweiligen Prototypen auf zwei Geräten.
In der \autoref{tab:test:tests} ist abzulesen, dass beide Prototypen alle Tests dieser Reihe eindeutig bestanden haben.
Damit kann bestätigt werden, dass beide Prototypen auf zwei Geräten funktionieren, solange diese online sind.
%
%
%
\subsub{Offlinefunktionalität und Konfliktmanagement}
Während der zweiten und dritten Testreihe waren entweder ein oder beide Geräte für die Ausführung bestimmter Aktionen offline.
Der Fokus dieser Tests lag auf dem eventuell eintretenden Datenverlust im Konfliktfall.
\autoref{tab:test:tests} zeigt, dass das Anlegen und Löschen weder bei \it{amilia-qouch}, noch bei \it{amilia-rdx} ein Problem darstellte.
Beim Anlegen eines Kontakts war dieser auf beiden Geräten doppelt gelistet. Auf diesem Wege ging kein Kontakt verloren.
% Es wurde kein Algorithmus auf Anwendungsebene implementiert, der Dopplungen in 
Auch das Löschen eines Kontakts erfolgte reibungslos. Der gelöschte Kontakt war nach der Synchronisation auf keinem der Geräte mehr vorhanden.\\
%
Sowohl in der zweiten, als auch in der dritten Testreihe sind im Redux Offline--Prototyp beim Bearbeiten, bzw. beim Bearbeiten und Löschen eines Kontakts, Daten verloren gegangen.
Wo im Prototypen \it{amilia-qouch} die aufgetretenen Konflikte gespeichert wurden, um sie auf Benutzungsebene zu lösen, ging im anderen Protoypen \it{amilia-rdx} einer der konfliktbehafteten Datensätze verloren.
Bei gleichzeitigem Bearbeiten eines Kontakts gewann die Version, die zuletzt synchronisiert wurde, die andere wurde überschrieben.
% \todo{Die Daten werden bei Redux Offline in der \gls{Queue} gespeichert und dort nacheinander angearbeitet. Das gleicht der Konfliktmanagementstrategie \gls{LWW}.}\\
Wurde ein Kontakt auf einem Gerät bearbeitet und auf dem anderen gelöscht, war es in jedem Fall nicht mehr in der Liste. Löschungen haben immer Vorrang, wodurch ebenfalls Daten verloren gehen.
% 
%
\subsub{Beobachtungen}
Zusammenfassend lässt sich sagen, dass Redux Offline per se nicht auf mehreren Geräten offlinefähig ist.
Bei den Testfällen in \autoref{chap:impl:test}, die während der Entwicklung stetig -- auf einem Gerät -- durchgeführt wurden, bestand \it{amilia-rdx} jeden Test.
Auf nur einem Gerät ist es jedoch schwierig, Konflikte zu erzeugen. Deswegen ist die \gls{LWW} Strategie von Redux Offline für die Verwendung auf einem Gerät akzeptabel.\\
Der Prototyp \it{amilia-qouch} hingegen hat alle Tests erwartungsgemäß bestanden.
Immer wenn es zu einem Konflikt gekommen war, konnte dieser über den Konfliktdialog einfach und vor allem korrekt gelöst werden.
In keinem Testdurchlauf gingen Daten verloren.
%
%
%
%
\sub{WIP Erfüllung der Anforderungen}
% Zur Auswertung der Testergebnisse werden die Testreihen gruppiert.
Die zweite und dritte Testreihe befassen sich mit der Offlinefähigkeit der Prototypen.
Es spielt hierbei keine Rolle, ob ein oder zwei der Geräte während der Testdurchläufe offline sind. Deswegen werden diese Testreihen in der Auswertung zusammengefasst.
%
%
%
\subsub{\label{chap:F1}Kollaborationsfähigkeit}
In der ersten Testreihe sind beide Geräte die ganze Zeit mit dem Internet verbunden.
Das Hauptaugenmerk lag auf der allgemeinen Funktionalität des jeweiligen Prototypen auf zwei Geräten.
In der \autoref{tab:test:tests} ist abzulesen, dass beide Prototypen alle Tests dieser Reihe eindeutig bestanden haben.
Damit kann bestätigt werden, dass beide Prototypen auf zwei Geräten funktionieren, solange diese online sind.
%
%
%
\subsub{Offlinefunktionalität und Konfliktmanagement}
Während der zweiten und dritten Testreihe waren entweder ein oder beide Geräte für die Ausführung bestimmter Aktionen offline.
Der Fokus dieser Tests lag auf dem eventuell eintretenden Datenverlust im Konfliktfall.
\autoref{tab:test:tests} zeigt, dass das Anlegen und Löschen weder bei \it{amilia-qouch}, noch bei \it{amilia-rdx} ein Problem darstellte.
Beim Anlegen eines Kontakts war dieser auf beiden Geräten doppelt gelistet. Auf diesem Wege ging kein Kontakt verloren.
% Es wurde kein Algorithmus auf Anwendungsebene implementiert, der Dopplungen in 
Auch das Löschen eines Kontakts erfolgte reibungslos. Der gelöschte Kontakt war nach der Synchronisation auf keinem der Geräte mehr vorhanden.\\
%
Sowohl in der zweiten, als auch in der dritten Testreihe sind im Redux Offline--Prototyp beim Bearbeiten, bzw. beim Bearbeiten und Löschen eines Kontakts, Daten verloren gegangen.
Wo im Prototypen \it{amilia-qouch} die aufgetretenen Konflikte gespeichert wurden, um sie auf Benutzungsebene zu lösen, ging im anderen Protoypen \it{amilia-rdx} einer der konfliktbehafteten Datensätze verloren.
Bei gleichzeitigem Bearbeiten eines Kontakts gewann die Version, die zuletzt synchronisiert wurde, die andere wurde überschrieben.
% \todo{Die Daten werden bei Redux Offline in der \gls{Queue} gespeichert und dort nacheinander angearbeitet. Das gleicht der Konfliktmanagementstrategie \gls{LWW}.}\\
Wurde ein Kontakt auf einem Gerät bearbeitet und auf dem anderen gelöscht, war es in jedem Fall nicht mehr in der Liste. Löschungen haben immer Vorrang, wodurch ebenfalls Daten verloren gehen.
% 
%
\subsub{Beobachtungen}
Zusammenfassend lässt sich sagen, dass Redux Offline per se nicht auf mehreren Geräten offlinefähig ist.
Bei den Testfällen in \autoref{chap:impl:test}, die während der Entwicklung stetig -- auf einem Gerät -- durchgeführt wurden, bestand \it{amilia-rdx} jeden Test.
Auf nur einem Gerät ist es jedoch schwierig, Konflikte zu erzeugen. Deswegen ist die \gls{LWW} Strategie von Redux Offline für die Verwendung auf einem Gerät akzeptabel.\\
Der Prototyp \it{amilia-qouch} hingegen hat alle Tests erwartungsgemäß bestanden.
Immer wenn es zu einem Konflikt gekommen war, konnte dieser über den Konfliktdialog einfach und vor allem korrekt gelöst werden.
In keinem Testdurchlauf gingen Daten verloren.
%
% F1 Die Anwendung muss unabhängig vom Netzwerkstatus und auf mindestens zwei Geräten funktionieren.
% F2 Die Anwendung soll fähig sein den Netzwerkstatus zu ändern.
% F2 Die Anwendung muss den Netzwerkstatus erkenntlich machen.
% F3 Die Anwendung muss fähig sein die Kontakte unabhängig vom Netzwerkstatus zu laden, sofern diese einmal aus dem Netzwerk geladen wurden.
% F4 Die Anwendung muss die Möglichkeit bieten, unabhängig vom Netzwerkstatus einen Kontakt anzulegen, zu bearbeiten oder zu löschen.
% F5 Die Anwendung muss alle Kontakte sowohl lokal als auch serverseitig persisitieren, identifizieren können und versionieren.
% F6 Die Anwendung muss die lokal gespeicherten Kontakte mit denen auf der Serverdatenbank persistierten synchronisieren, sobald die Anwendung mit dem Internet verbunden ist.
% F7 Die Anwendung muss die Möglichkeit bieten die Konfliktmanagementstrategien der zu untersuchenden Technologien zu evaluieren.
% F8 Die Anwendung soll Konflikte speichern, sofern diese auftreten.
% F9 Die Anwendung muss die Möglichkeit bieten die Konflikte zu lösen, sofern diese auftreten.
% F10 Die Anwendung muss sicherstellen, dass auf keinen Fall Daten verloren gehen.

% T4 Ich als TesterIn möchte Konflikte forcieren, um das Verhalten der Anwendung zu evaluieren.
%
\sub{Implementierungsaufwand}
% D7, D8
Beim Vergleich des Implementierungsaufwands werden verschiedene Punkte Berücksichtigt.
Es wird der Arbeitsaufwand betrachtet, der benötigt wird die verwendete Technologie zu verstehen und zu benutzen.
Ebenfalls von hoher Wichtigkeit ist die Lesbarkeit des geschriebenen Quellcodes.
Sauberer und verständlicher Code ist einfacher les-- und somit auch wartbarer. Interessant ist auch die Menge des geschriebenen Codes.
%
\subsub{Einbinding der Technologien}
\begin{description}[leftmargin=0cm,style=nextline]
  \item[amilia-qouch]
  Der Aufwand eine Anwendung mit PouchDB und CouchDB zu schreiben, ist gering.
  Man muss nur beide Technologien installieren und im Code instanziieren.
  PouchDB dient als Schnittstelle zu CouchDB, weswegen nur das simpel gehaltene \gls{API} von PouchDB benutzt wird.
  Die Dokumentationen beider Technologien sind sehr ausfühlich und tragen ausgeprochen zum Verständnis der Funktionsweise bei.

  \item[amilia-rdx]
  Der Aufwand eine Anwendung mit Redux Offline zu schreiben, ist deutlich komplexer, da Redux eingesetzt und verstanden werden muss.
  Die Einbindung von Redux Offline war allerdings minimal aufwändig, sobald Redux einmal implementiert war.
  Redux ist sehr gut Dokumentiert und bietet viele Beispiele zur Anwendung.
  Die Dokumentation von Redux Offline besteht aus einer verschachtelten Readme Datei, was die Lesbarkeit und somit das Verständnis der Funktionsweise der Technologie erschwert.
\end{description}
%
\subsub{Lesbarkeit des Codes}
\begin{description}[leftmargin=0cm,style=nextline]
  \item[amilia-qouch]
  Auch wenn man mit den Technologien nicht vertraut ist, lässt sich der Code problemlos lesen.
  Für die Synchronisation beider Technologien reicht eine Zeile Code und umgesetzten \gls{CRUD} Operationen sind dank unmissverständlicher API eindeutig nachvollziehbar.
  %
  \item[amilia-rdx] 
  Um den Code ohne Probleme lesen zu können sollte man mit Redux und dessen Datenfluss vertraut sein.
  Ist das der Fall, ist ein Blick in die Dokumentation von Redux Offline hilfreich, da die verwendeten Metaattribute der Aktionen nicht unbedingt eindeutig sind.
  Sind diese Dinge klar, ist auch der Code von \it{amilia-rdx} gut lesbar und begreiflich.
\end{description}
%
  \subsub{Menge des geschriebenen Quellcodes}
  Beide Prototypen wurden mit React Create App erstellt.
  Die Menge des Codes die dadurch erzeugt wurde, wurde nicht mitgezählt. Ebensowenig wurden die Kommentare und das geschriebene \gls{CSS} gezählt.
\begin{description}[leftmargin=0cm,style=nextline]
    \item[amilia-qouch]
    Es wurden 371 Zeilen Code hinzugefügt.
    \item[amilia-rdx] 
    Es wurden 586 Zeilen Code hinzugefügt.
\end{description}
% 
% Welche Strategie wird verwendet? Welche Technologie ist nun besser geeignet?