\sub{WIP Implementierungsaufwand}

\subsub{Setup der \gls{App}}
Irrelevant und identisch wegen Create React App.

%
\subsub{Einbinding der Technologien}
\begin{description}[leftmargin=0.5cm,style=nextline]
  \item[amilia-qouch]
  gering, nur Installation und Anwendung der Pouch API\\
  Neue Instanz, Synchronisation ( 1 Zeile) und CRUD
  \item[amilia-rdx]
  komplex, da Redux eingesetzt und verstanden werden muss. Für komplexere Apps mag das generell besser sein, aber für den einfachen Anwendungsfall zu viel Overhead. Einbindung von Redux Offline war allerdings wenig aufwändig sobald Redux implementiert war
\end{description}
%
\subsub{Lesbarkeit des Codes}
\begin{description}[leftmargin=0.5cm,style=nextline]
  \item[amilia-qouch]
  nicht mehr als erwartet und auch verständlich wenn man mit den Technologien nicht vertraut ist. Es liest sich so wie man es verwendet dank unmissverständlicher API.
  %
  \item[amilia-rdx] 
  Um den Code ohne Probleme lesen zu können muss man mit Redux vertraut sein. Ist das der Fall, ist trotzdem ein Blick in die Dokumentation von Redux Offline hilfreich, weil die Metaattribute nicht für sich selbst sprechen (effect, commit, rollback).
\end{description}
%
  \subsub{Menge des geschriebenen Quellcodes}
  Beide Prototypen wurden mit React Create App erstellt. Die Menge des Codes die dadurch erzeugt wurde, wurde nicht mitgezählt. Genausowenig wie Kommentare und das geschriebene \gls{CSS}.
  \begin{description}[leftmargin=0.5cm,style=nextline]
    \item[amilia-qouch]
    Dem \it{amilia-qouch} Prototypen wurden 371 Zeilen Code hinzugefügt.
    \item[amilia-rdx] 
    Dem \it{amilia-rdx} Prototypen wurden 586 Zeilen Code hinzugefügt.
\end{description}
  %
  %
  %
  %
\sub{WIP Offlinefunktionalität}
\subsub{Offlinefähigkeit auf einem Gerät}
\subsub{Offlinefähigkeit auf zwei Geräten}
Wichtig: Redux Offline ist per se nicht auf mehreren Geräten nutzbar. Im Gegensatz zu Couch. Bei Redux Offline muss das Backend, das dafür obligatorisch ist, selbst implementiert werden.
%
\sub{WIP Konfliktmanagement}
Delete always wins\\
last (syynchronisiertes) update wins == random loss of user data!\\
Qouch: einfaches Handling...

% Das sind die funktionalen ANforderungen aus Kapitel 5.2
% \begin{longtable}[c]{@{}
% >{\columncolor[HTML]{CFFCC2}}l ll@{}}
% \toprule
%     \multicolumn{1}{p{0.05\textwidth}}{\cellcolor[HTML]{cffcc2}\textbf{ID}}
%     & \multicolumn{1}{p{0.9\textwidth}}{\cellcolor[HTML]{cffcc2}\textbf{Funktionale Anforderungen}}\\ \hline \noalign{\vskip 0.1cm}
% \endfirsthead
% \endhead
% %
% % 
%   \multicolumn{1}{p{0.05\textwidth}}{\cellcolor[HTML]{cffcc2}\textbf{F1}} &
%   \multicolumn{1}{p{0.9\textwidth}}
%   {Die Anwendung muss unabhängig vom Netzwerkstatus und auf mindestens zwei Geräten funktionieren.}\\
%   \midrule
%   %
%   \multicolumn{1}{p{0.05\textwidth}}{\cellcolor[HTML]{cffcc2}\textbf{F2}} &
%   \multicolumn{1}{p{0.9\textwidth}}
%   {Die Anwendung soll fähig sein den Netzwerkstatus zu ändern.}\\
%   \midrule
%   %
%   \multicolumn{1}{p{0.05\textwidth}}{\cellcolor[HTML]{cffcc2}\textbf{F3}} &
%   \multicolumn{1}{p{0.9\textwidth}}
%   {Die Anwendung muss fähig sein die Kontakte unabhängig vom Netzwerkstatus zu laden, sofern diese einmal aus dem Netzwerk geladen wurden.}\\
%   \midrule
%   %
%   \multicolumn{1}{p{0.05\textwidth}}{\cellcolor[HTML]{cffcc2}\textbf{F4}} &
%   \multicolumn{1}{p{0.9\textwidth}}
%   {Die Anwendung muss die Möglichkeit bieten, unabhängig vom Netzwerkstatus einen Kontakt anzulegen, zu bearbeiten oder zu löschen.}\\
%   \midrule
%   %
%   \multicolumn{1}{p{0.05\textwidth}}{\cellcolor[HTML]{cffcc2}\textbf{F5}} &
%   \multicolumn{1}{p{0.9\textwidth}}
%   {Die Anwendung muss alle Kontakte sowohl lokal als auch serverseitig persisitieren, identifizieren können und versionieren.}\\
%   \midrule
%     %
%   \multicolumn{1}{p{0.05\textwidth}}{\cellcolor[HTML]{cffcc2}\textbf{F6}} &
%   \multicolumn{1}{p{0.9\textwidth}}
%   {Die Anwendung muss die lokal gespeicherten Kontakte mit denen auf der Serverdatenbank persistierten synchronisieren, sobald die Anwendung mit dem Internet verbunden ist.}\\
%   \midrule
%   %
%   \multicolumn{1}{p{0.05\textwidth}}{\cellcolor[HTML]{cffcc2}\textbf{F7}} &
%   \multicolumn{1}{p{0.9\textwidth}}
%   {Die Anwendung muss die Möglichkeit bieten die Konfliktmanagementstrategien der zu untersuchenden Technologien zu evaluieren.}\\
%   \midrule
%     %
%   \multicolumn{1}{p{0.05\textwidth}}{\cellcolor[HTML]{cffcc2}\textbf{F8}} &
%   \multicolumn{1}{p{0.9\textwidth}}
%   {Die Anwendung soll Konflikte speichern, sofern diese auftreten.}\\
%   \midrule
%   %
%   \multicolumn{1}{p{0.05\textwidth}}{\cellcolor[HTML]{cffcc2}\textbf{F9}} &
%   \multicolumn{1}{p{0.9\textwidth}}
%   {Die Anwendung muss die Möglichkeit bieten die Konflikte zu lösen, sofern diese auftreten.}\\
%   \midrule
%   %
%   \multicolumn{1}{p{0.05\textwidth}}{\cellcolor[HTML]{cffcc2}\textbf{F10}} &
%   \multicolumn{1}{p{0.9\textwidth}}
%   {Die Anwendung muss sicherstellen, dass auf keinen Fall Daten verloren gehen.}\\
%   %
%   % end
%   \bottomrule \cellcolor[HTML]{FFFFFF}
%   \vspace{0.1cm}\\
%   \noalign{\hspace{0.0525\textwidth}\grayRule}
%   \caption{Funktionale Anforderungen}
%   \label{tab:funcreq}\\
% \end{longtable}