  %
\sub{\label{chap:testauswertung}Auswertung der manuellen Tests}
Zur Auswertung der Testergebnisse werden die Testreihen gruppiert.
Die zweite und dritte Testreihe befassen sich mit der Offlinefähigkeit der Prototypen.
Es spielt hierbei keine Rolle, ob ein oder zwei der Geräte während der Testdurchläufe offline sind. Deswegen werden diese Testreihen in der Auswertung zusammengefasst.
%
%
%
\subsub{\label{chap:F1}Kollaborationsfähigkeit}
In der ersten Testreihe sind beide Geräte die ganze Zeit mit dem Internet verbunden.
Das Hauptaugenmerk lag auf der allgemeinen Funktionalität des jeweiligen Prototypen auf zwei Geräten.
In der \autoref{tab:test:tests} ist abzulesen, dass beide Prototypen alle Tests dieser Reihe eindeutig bestanden haben.
Damit kann bestätigt werden, dass beide Prototypen auf zwei Geräten funktionieren, solange diese online sind.
%
%
%
\subsub{Offlinefunktionalität und Konfliktmanagement}
Während der zweiten und dritten Testreihe waren entweder ein oder beide Geräte für die Ausführung bestimmter Aktionen offline.
Der Fokus dieser Tests lag auf dem eventuell eintretenden Datenverlust im Konfliktfall.
\autoref{tab:test:tests} zeigt, dass das Anlegen und Löschen weder bei \it{amilia-qouch}, noch bei \it{amilia-rdx} ein Problem darstellte.
Beim Anlegen eines Kontakts war dieser auf beiden Geräten doppelt gelistet. Auf diesem Wege ging kein Kontakt verloren.
% Es wurde kein Algorithmus auf Anwendungsebene implementiert, der Dopplungen in 
Auch das Löschen eines Kontakts erfolgte reibungslos. Der gelöschte Kontakt war nach der Synchronisation auf keinem der Geräte mehr vorhanden.\\
%
Sowohl in der zweiten, als auch in der dritten Testreihe sind im Redux Offline--Prototyp beim Bearbeiten, bzw. beim Bearbeiten und Löschen eines Kontakts, Daten verloren gegangen.
Wo im Prototypen \it{amilia-qouch} die aufgetretenen Konflikte gespeichert wurden, um sie auf Benutzungsebene zu lösen, ging im anderen Protoypen \it{amilia-rdx} einer der konfliktbehafteten Datensätze verloren.
Bei gleichzeitigem Bearbeiten eines Kontakts gewann die Version, die zuletzt synchronisiert wurde, die andere wurde überschrieben.
% \todo{Die Daten werden bei Redux Offline in der \gls{Queue} gespeichert und dort nacheinander angearbeitet. Das gleicht der Konfliktmanagementstrategie \gls{LWW}.}\\
Wurde ein Kontakt auf einem Gerät bearbeitet und auf dem anderen gelöscht, war es in jedem Fall nicht mehr in der Liste. Löschungen haben immer Vorrang, wodurch ebenfalls Daten verloren gehen.
% 
%
\subsub{Beobachtungen}
Zusammenfassend lässt sich sagen, dass Redux Offline per se nicht auf mehreren Geräten offlinefähig ist.
Bei den Testfällen in \autoref{chap:impl:test}, die während der Entwicklung stetig -- auf einem Gerät -- durchgeführt wurden, bestand \it{amilia-rdx} jeden Test.
Auf nur einem Gerät ist es jedoch schwierig, Konflikte zu erzeugen. Deswegen ist die \gls{LWW} Strategie von Redux Offline für die Verwendung auf einem Gerät akzeptabel.\\
Der Prototyp \it{amilia-qouch} hingegen hat alle Tests erwartungsgemäß bestanden.
Immer wenn es zu einem Konflikt gekommen war, konnte dieser über den Konfliktdialog einfach und vor allem korrekt gelöst werden.
In keinem Testdurchlauf gingen Daten verloren.
%
%
%
%
\sub{\label{chap:auswertunganforderungen}Erfüllung der Anforderungen}
In \autoref{chap:funktionaleanforderungen} wurden die funktionalen Anforderungen an eine offlinefähige, kollaborative Anwendung erarbeitet.
Es wird überprüft, welche dieser Anforderungen prototypisch erüllt wurden und welche nicht.
%
\begin{description}[leftmargin=0cm,style=nextline]
  \item[F1 Die Anwendung muss auf mindestens zwei Geräten funktionieren.]
    Die Auswertung der ersten Testreihe in \autoref{chap:F1} zeigt, dass beide Prototypen auf zwei Geräten funktionieren, sofern beide Geräte mit dem Internet verbunden sind.
    Da die Testdurchläufe nur auf zwei Geräten durchgeführt wurden, ist keine Aussage darüber zu treffen, ob die Anwendung auch auf mehr als zwei Geräten uneingeschränkt funktioniert.\\
  %
  \item[F2 Die Anwendung soll fähig sein, den Netzwerkstatus zu ändern.]
    Diese Anforderung ist indirekt erfüllt. Die Prototypen wurden für die Browser Firefox und Chrome entwickelt.
    Beide ermöglichen es in den Einstellungen, den Offlinemodus zu aktivieren und so die Anwendung vom Internet zu trennen.
    Auch durch das Ausschalten des Servers kann die Verbindung zwischen den beiden Geräten, auf denen die Anwendung läuft, unterbrochen werden.\\
%
  \item[F3 Die Anwendung muss den Netzwerkstatus erkenntlich machen.]
    Im Header eines jeden Prototypen wird der aktuelle Netzwerkstatus der Anwendung korrekt angezeigt.
    Dafür wird das externe Modul React Detect Offline verwendet, dem die Adresse des Servers übergeben wird.
    Über diese Adresse wird alle fünf Sekunden geprüft, ob der Server erreichbar ist.
    Aus letzterem folgt, dass mit einer leichten zeitlichen Abweichung in der Anzeige zu rechnen ist.\\
%
  \item[F4 Die Anwendung muss fähig sein, die Kontakte unabhängig vom Netzwerkstatus zu laden, sofern diese einmal aus dem Netzwerk geladen wurden.]
    Einmal geladen, werden alle Kontakte in beiden Prototypen aus dem lokalen Speicher gelesen, wenn die Anwendung offline ist oder der Server nicht läuft.
    Der \it{amilia-qouch} Prototyp liest die Kontakte dann aus der IndexedDB und der \it{amilia-rdx} Prototyp aus dem Local Storage.\\
%
  \item[F5 Die Anwendung muss die Möglichkeit bieten, unabhängig vom Netzwerkstatus einen Kontakt anzulegen, zu bearbeiten oder zu löschen.]
    Während der Entwicklung wurden stetig Tests zur Offlinefunktionalität auf einem Gerät durchgeführt (siehe \autoref{chap:impl:test}).
    Diese bestätigen, dass in jedem Prototypen unabhängig vom Netzwerkstatus jederzeit ein Kontakt angelegt, bearbeitet oder gelöscht werden kann.\\
%
  \item[F6 Die Anwendung muss alle Kontakte sowohl lokal als auch serverseitig persistieren. Die Konntakte müssen identifiziert und versioniert werden.]
    Auch diese Anforderung erfüllen beide Prototypen. Die lokale Speicherung erfolgt, wie bereits beschrieben, im Local Storage oder in IndexedDB.
    Serverseitig werden die Kontakte im Prototypen \it{amilia-qouch} in einer CouchDB persistiert.
    Da Redux Offline keine serverseitige Datenbank zur Verfügung stellt, werden die Daten in diesem Prototypen in einer serverseitigen \gls{JSON} Datei gespeichert.
    Die Identifizierung der Kontakte erfolgt in beiden Prototypen über einen Zeitstempel.
    In \it{amilia-rdx} erfolgt die Versionierung ebenfalls über den Zeitstempel. Der Prototyp \it{amilia-qouch} verwendet hierzu eine Revisionsnummer.
    Diese Revisionsnummer ist der Dreh-- und Angelpunkt für das Konfliktmanagement in CouchDB.\\
%
  \item[F7 Die Anwendung muss die lokal gespeicherten Kontakte mit denen in der Serverdatenbank persistierten synchronisieren, sobald die Anwendung mit dem Internet verbunden ist.]
    Ist die Anwendung offline oder der Server ausgeschaltet, werden alle Aktionen die während dieses Zeitraums durchgeführt werden, mit der Serverdatenbank, bzw. der \gls{JSON} Datei synchronisiert. Diese Funktionalität wird von allen in den Prototypen verwendeten Technologien bereitgestellt.\\
%
  \item[F8 Die Anwendung muss die Möglichkeit bieten, die Konfliktmanagementstrategien der zu untersuchenden Technologien zu evaluieren.]
  Um die Konfliktmanagementstrategien der verwendeten Technologien untersuchen zu können, müssen Konflikte herbeigeführt werden.
  Die Möglichkeit, die Anwendung auf zwei Geräten nutzen zu können und diese vom Internet zu trennen, macht den Vorgang durchführbar.
  Die Visualisierung des Netzwerkstatus im Header der Kontaktliste vereinfacht diesen Vorgang.
  Die in \autoref{chap:manuelletests} durchgeführten manuellen Tests waren nur aufgrund dieser Funktionalitäten ausführbar. Die Auswertung erfolgte in \autoref{chap:auswertung}, wodurch diese Anforderung erfüllt ist.\\
%
  \item[F9 Die Anwendung soll Konflikte speichern, sofern diese auftreten.]
    Redux Offline bietet keine Möglichkeit Konflikte zu speichern, weswegen diese Funktionalität für \it{amilia-rdx} nicht umgesetzt werden konnte.
    Im anderen Prototypen werden die Revisionsnummern der konfliktbehafteten Kontakte im Kontakt gespeichert, wodurch der Prototyp \it{amilia-qouch} diese Anforderung erfüllt.\\
%
  \item[F10 Die Anwendung muss die Möglichkeit bieten, Konflikte zu lösen, sofern diese auftreten.]
    Wenn Konflikte auftreten, werden sie in beiden Prototypen gelöst. Im Redux Offline Prototypen wird dazu im Groben die \gls{LWW} Strategie verwendet.
    Bei Löschungen ist es egal, welche Änderung zuletzt durchgeführt wurde. Löschungen gewinnen immer.
    Im CouchDB Prototypen werden Konflikte zuerst automatisch durch den in \autoref{chap:replication} beschriebenen Algorithmus gelöst.
    Da dieser Algorithmus die gewinnende Version mehr oder weniger zufällig auswählt, werden alle konfliktbehafteten Versionen gespeichert.
    So können sie auf Anwendungsebene gelöst werden, was über den Modaldialog passiert.\\
%
  \item[F11 Die Anwendung muss sicherstellen, dass auf keinen Fall Daten verloren gehen.]
    Durch das praktische Konfliktmanagement in CouchDB gehen im Prototypen keine Daten verloren.
    In \it{amilia-rdx} gehen in jeder getesteten Konfliktsituation Daten verloren.
    Durch die automatische Festlegung der gewinnenden Version wird die andere überschrieben, unabhängig davon, welche die richtige Version ist.
\end{description}
%
\sub{Implementierungsaufwand}
% D7, D8
Beim Vergleich des Implementierungsaufwands werden verschiedene Punkte Berücksichtigt.
Es wird der Arbeitsaufwand betrachtet, der benötigt wird die verwendete Technologie zu verstehen und zu benutzen.
Ebenfalls von hoher Wichtigkeit ist die Lesbarkeit des geschriebenen Quellcodes.
Sauberer und verständlicher Code ist einfacher les-- und somit auch wartbarer. Interessant ist auch die Menge des geschriebenen Codes.
%
\subsub{Einbinding der Technologien}
\begin{description}[leftmargin=0cm,style=nextline]
  \item[amilia-qouch]
  Der Aufwand eine Anwendung mit PouchDB und CouchDB zu schreiben, ist gering.
  Man muss nur beide Technologien installieren und im Code instanziieren.
  PouchDB dient als Schnittstelle zu CouchDB, weswegen nur das simpel gehaltene \gls{API} von PouchDB benutzt wird.
  Die Dokumentationen beider Technologien sind sehr ausfühlich und tragen ausgeprochen zum Verständnis der Funktionsweise bei.

  \item[amilia-rdx]
  Der Aufwand eine Anwendung mit Redux Offline zu schreiben, ist deutlich komplexer, da Redux eingesetzt und verstanden werden muss.
  Die Einbindung von Redux Offline war allerdings minimal aufwändig, sobald Redux einmal implementiert war.
  Redux ist sehr gut Dokumentiert und bietet viele Beispiele zur Anwendung.
  Die Dokumentation von Redux Offline besteht aus einer verschachtelten Readme Datei, was die Lesbarkeit und somit das Verständnis der Funktionsweise der Technologie erschwert.
\end{description}
%
\subsub{Lesbarkeit des Codes}
\begin{description}[leftmargin=0cm,style=nextline]
  \item[amilia-qouch]
  Auch wenn man mit den Technologien nicht vertraut ist, lässt sich der Code problemlos lesen.
  Für die Synchronisation beider Technologien reicht eine Zeile Code und umgesetzten \gls{CRUD} Operationen sind dank unmissverständlicher API eindeutig nachvollziehbar.
  %
  \item[amilia-rdx] 
  Um den Code ohne Probleme lesen zu können sollte man mit Redux und dessen Datenfluss vertraut sein.
  Ist das der Fall, ist ein Blick in die Dokumentation von Redux Offline hilfreich, da die verwendeten Metaattribute der Aktionen nicht unbedingt eindeutig sind.
  Sind diese Dinge klar, ist auch der Code von \it{amilia-rdx} gut lesbar und begreiflich.
\end{description}
%
  \subsub{Menge des geschriebenen Quellcodes}
  Beide Prototypen wurden mit React Create App erstellt.
  Die Menge des Codes die dadurch erzeugt wurde, wurde nicht mitgezählt. Ebensowenig wurden die Kommentare und das geschriebene \gls{CSS} gezählt.
\begin{description}[leftmargin=0cm,style=nextline]
    \item[amilia-qouch]
    Es wurden 371 Zeilen Code hinzugefügt.
    \item[amilia-rdx] 
    Es wurden 586 Zeilen Code hinzugefügt.
\end{description}
% 
% Welche Strategie wird verwendet? Welche Technologie ist nun besser geeignet?