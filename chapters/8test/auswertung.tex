\sub{WIP Implementierungsaufwand}
% D7, D8
Beim Vergleich des Implementierungsaufwands werden verschiedene Punkte Berücksichtigt.
Es wird der Arbeitsaufwand betrachtet, der benötigt wird die verwendete Technologie zu verstehen und zu benutzen.
Ebenfalls von hoher Wichtigkeit ist die Lesbarkeit des geschriebenen Quellcodes.
Sauberer und verständlicher Code ist einfacher les-- und somit auch wartbarer. Interessant ist auch die Menge des geschriebenen Codes.
%
\subsub{Einbinding der Technologien}
\begin{description}[leftmargin=0cm,style=nextline]
  \item[amilia-qouch]
  Der Aufwand eine Anwendung mit PouchDB und CouchDB zu schreiben, ist gering.
  Man muss nur beide Technologien installieren und im Code instanziieren.
  PouchDB dient als Schnittstelle zu CouchDB, weswegen nur das simpel gehaltene \gls{API} von PouchDB benutzt wird.
  Die Dokumentationen beider Technologien sind sehr ausfühlich und tragen ausgeprochen zum Verständnis der Funktionsweise bei.

  \item[amilia-rdx]
  Der Aufwand eine Anwendung mit Redux Offline zu schreiben, ist deutlich komplexer, da Redux eingesetzt und verstanden werden muss.
  Die Einbindung von Redux Offline war allerdings minimal aufwändig, sobald Redux einmal implementiert war.
  Redux ist sehr gut Dokumentiert und bietet viele Beispiele zur Anwendung.
  Die Dokumentation von Redux Offline besteht aus einer verschachtelten Readme Datei, was die Lesbarkeit und somit das Verständnis der Funktionsweise der Technologie erschwert.
\end{description}
%
\subsub{Lesbarkeit des Codes}
\begin{description}[leftmargin=0cm,style=nextline]
  \item[amilia-qouch]
  Auch wenn man mit den Technologien nicht vertraut ist, lässt sich der Code problemlos lesen.
  Für die Synchronisation beider Technologien reicht eine Zeile Code und umgesetzten \gls{CRUD} Operationen sind dank unmissverständlicher API eindeutig nachvollziehbar.
  %
  \item[amilia-rdx] 
  Um den Code ohne Probleme lesen zu können sollte man mit Redux und dessen Datenfluss vertraut sein.
  Ist das der Fall, ist ein Blick in die Dokumentation von Redux Offline hilfreich, da die verwendeten Metaattribute der Aktionen nicht unbedingt eindeutig sind.
  Sind diese Dinge klar, ist auch der Code von \it{amilia-rdx} gut lesbar und begreiflich.
\end{description}
%
  \subsub{Menge des geschriebenen Quellcodes}
  Beide Prototypen wurden mit React Create App erstellt.
  Die Menge des Codes die dadurch erzeugt wurde, wurde nicht mitgezählt. Ebensowenig wurden die Kommentare und das geschriebene \gls{CSS} gezählt.
\begin{description}[leftmargin=0cm,style=nextline]
    \item[amilia-qouch]
    Es wurden 371 Zeilen Code hinzugefügt.
    \item[amilia-rdx] 
    Es wurden 586 Zeilen Code hinzugefügt.
\end{description}
  %
  %
  %
\sub{WIP Offlinefunktionalität}
\subsub{Offlinefähigkeit auf einem Gerät}
\subsub{Offlinefähigkeit auf zwei Geräten}
Wichtig: Redux Offline ist per se nicht auf mehreren Geräten nutzbar. Im Gegensatz zu Couch. Bei Redux Offline muss das Backend, das dafür obligatorisch ist, selbst implementiert werden.

% F1 Die Anwendung muss unabhängig vom Netzwerkstatus und auf mindestens zwei Geräten funktionieren.
% F2 Die Anwendung soll fähig sein den Netzwerkstatus zu ändern.
% F2 Die Anwendung muss den Netzwerkstatus erkenntlich machen.
% F3 Die Anwendung muss fähig sein die Kontakte unabhängig vom Netzwerkstatus zu laden, sofern diese einmal aus dem Netzwerk geladen wurden.
% F4 Die Anwendung muss die Möglichkeit bieten, unabhängig vom Netzwerkstatus einen Kontakt anzulegen, zu bearbeiten oder zu löschen.
% F5 Die Anwendung muss alle Kontakte sowohl lokal als auch serverseitig persisitieren, identifizieren können und versionieren.
% F6 Die Anwendung muss die lokal gespeicherten Kontakte mit denen auf der Serverdatenbank persistierten synchronisieren, sobald die Anwendung mit dem Internet verbunden ist.
% F7 Die Anwendung muss die Möglichkeit bieten die Konfliktmanagementstrategien der zu untersuchenden Technologien zu evaluieren.
% F8 Die Anwendung soll Konflikte speichern, sofern diese auftreten.
% F9 Die Anwendung muss die Möglichkeit bieten die Konflikte zu lösen, sofern diese auftreten.
% F10 Die Anwendung muss sicherstellen, dass auf keinen Fall Daten verloren gehen.

% T4 Ich als TesterIn möchte Konflikte forcieren, um das Verhalten der Anwendung zu evaluieren.
%
\sub{WIP Konfliktmanagement}
Delete always wins\\
last (syynchronisiertes) update wins == random loss of user data!\\
Qouch: einfaches Handling...
