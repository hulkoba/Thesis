\chapter{\label{chap:implementierung}WIP Implementierung der Prototypen}
In diesem Kaptitel wird nach dem in Kapitel \ref{chap:konzeption} präsentiertem Lösungsweg die detaillierte Beschreibung der technischen Realisierung der Prototypen vorgestellt.\\
Nach der Beschreibung der Realisierung der grundlegenden Funktionen der Anwendung wird auf die Umsetzung der Offlinefunktionalität und des Konfliktmanagements eingegangen.
Im Zuge dessen wird der implementierte Algorithmus zur Lösung auftretender Konflikte vorgestellt.
%
%
%
\section{Die Contacts Komponente}
Die Herzstück jeder zu entwickelnden Anwendung ist die Komponente \tt{Contacts}.
Sie beinhaltet die zentralen Funktionalitäten und entscheidet welche Anzeigeelemente gerendert werden und welche nicht.
Im Prototypen \it{amilia-qouch} besteht sie aus der Datei \tt{Contacts.js} in der ein interne State festgelegt ist. Hier sind außerdem Funktionen implementiert über die der State manipuliert werden kann.\\
Durch den in \autoref{chap:redux} beschriebenen, Redux--spezifischen Datenfluss ist sie auf mehrere Dateien aufgeteilt.
\tt{Contacts.js} ist für das Rendern der anderen Komponenten zuständig. 
In \tt{actions.js} werden die Aktionen definiert, die über die Containerkomponente \tt{ContactsContainer.js} von jeder Komponente aufgerufen werden können um den \gls{App}state zu manipulieren.
Die Manipulation wird in der \tt{reducer.js} behandelt, wo die geänderte Kopie wieder zurück an den \tt{Store} gegeben wird und sie wieder bei der \tt{Contacts.js} ankommt.
Der Einfachheit halber wird im Folgenden von der \tt{Contacts} Komponente geschrieben, wobei für den Redux Prototypen der Zusammenschluss dieser soeben beschriebenen Dateien gemeint ist.\\\\
Die Methode \tt{componentDidMount} ist die dritte im React Komponenten--Lebenszyklus\footnote{ \url{https://reactjs.org/docs/react-component.html\#the-component-lifecycle} -- Zugriff: 28.07.1018} und wird aufgerufen sobald die Anwendung gemountet ist.
Hier werden werden die Kontakte geladen und im \gls{App}state gespeichert.
Sobald sicht der \gls{App}state ändert wird die \tt{render} Funktion der Komponente aufgerufen. \autoref{code:render} zeigt die Renderfunktion der \tt{Contacts} Komponente.
%
\begin{center}
  \lstinputlisting[language=REACT,numbers=left,xleftmargin=20pt,framexleftmargin=15pt,caption={Die Renderfunktion der Contacts Komponente}, label=code:render]{code/render.js}
\end{center}
%

%
%
%
\section{Offlinefunktionalität}
Die Prototypen \it{amilia-qouch} und \it{amilia-rdx} sind vollständig offline verwendbar.
Alle in \autoref{chap:offlinefirst} beschriebenen Grundvoraussetzungen werden von beiden Prototypen erfüllt und sämtliche umgesetzten Funktionen sind sowohl mit als auch ohne Internetverbindung durchführbar.
%
%
% LOCAL DB
%
\sub{Datenspeicherung}
\todo{Cachen der Assets geht erst nach dem Build -> ServiceWorker}\autoref{chap:aufbau}\\
Redux Offline speichert, wie in \autoref{sub:reduxpersist} bereits beschrieben, alle im Redux Store verwalteten Daten im LocalStorage. \autoref{fig:local-rdx} zeigt alle gespeicherten, nutzerInnengenerierten Daten im LocalStorage.
%
\begin{figure}[H]
  \centering
  \includegraphics[width=\textwidth]{impl/localRdx}
  \grayRule
  \caption[Gespeicherte Daten im LocalStorage]{Gespeicherte Daten des Prototypen amilia-rdx im LocalStorage,\\Screenhot: Developer Tools im Firefox Browser}
  \label{fig:local-rdx}
\end{figure}
% 
Der Prototyp \it{amilia-qouch} nutzt zur lokalen Datenspeicherung PouchDB. PouchDB speichert die von NutzerInnen generierten Daten in IndexedDB, vgl. \autoref{chap:pouch}. In \autoref{fig:local-qouch} sind die gespeicherten Daten in der IndexedDB zu sehen.
%
\begin{figure}[H]
  \centering
  \includegraphics[width=\textwidth]{impl/localQouch}
  \grayRule
  \caption[Gespeicherte Daten in IndexedDB]{Gespeicherte Daten des Prototypen aus amilia-qouch in IndexedDB,\\Screenhot: Developer Tools im Firefox Browser}
  \label{fig:local-qouch}
\end{figure}
%
% SYNC
%
\sub{Datenbanksynchronisation}
Zwischen PouchDB und CouchDB können Daten in Echtzeit synchronisiert werden. Um die Live--Replikation zu aktivieren, muss im Synchronisationsaufruf der Parameter \tt{live: true} gesetzt sein.
Bricht die Internetverbindung ab, stoppt auch die Synchronisation.
Dank der angegebenen Parameter \tt{retry: true} versucht PouchDB die Synchronisation solange neuzustarten bis die Anwendung wieder mit dem Internet verbunden ist. \autoref{code:sync} zeigt die Implementation der Datenbankensynchronisation im Prototypen \it{amilia-qouch}.
%
\begin{center}
  \lstinputlisting[language=REACT,
  numbers=left,xleftmargin=20pt,
  firstline=1,lastline=5,
  framexleftmargin=15pt,
  caption={Synchronisation zwischen PouchDB und CouchDB im Prototyp amilia-qouch},
  label=code:sync]{code/sync.js}
\end{center}
Redux Offline nimmt einem auch Arbeit bei der Datensynchronisation ab. Alle Daten die sich im Queue befinden werden automatisch an den Server gesendet, sobald eine Internetverbindung besteht. Das funktioniert jedoch nicht so einfach wenn der Server nicht an ist. In der Dokumentation von Redux Offline steht, dass die Aktion solange versucht wird auszuführen, bis die Anwendung wieder mit dem Internet verbunden ist ~\cite{giving-up}. Allerdings wird die ROLLBACK Aktion gefeuert wenn der Server nicht verfügbar ist und die Aktion wird abgebrochen.
\begin{center}
  \lstinputlisting[language=REACT,
  numbers=left,xleftmargin=20pt,
  firstline=6,
  framexleftmargin=15pt,
  caption={Discard Konfiguration für amilia-rdx},
  label=code:discard]{code/sync.js}
\end{center}
Die Discard Konfiguration bestimmt wann eine Aktion abgebrochen, und wann sie immer wieder neugestartet wird. Im \autoref{code:discard} ist in den Zeilen eins bis vier abzulesen wie diese Konfiguration überschrieben wird. Nun wird die Aktion nur abgebrochen wenn der Server verfügbar ist und einen \gls{HTTP} Status zwischen 400 und 500 zurückgibt.
In den darauffolgenden Zeilen wird die eigens implementierte Konfiguration mit der Standardkonfiguration von Redux Offline zusammengeführt wird.
%
%
%
\section{Konfliktmanagement}
Konflikte werden in beiden Prototypen manuell erzeugt wobei die Vorgehensweise identisch ist.
Wie ein Konflikt herbeigeführt werden kann, wird in \autoref{chap:konzept:test} beschrieben.
Durch das regelmäßige Anfragen an den Server wird dargestellt, ob sich die Anwendung im Onlinestatus befindet oder nicht.
Konflikte können erzeugt werden wenn ein oder beide Geräte auf denen die Anwendung läuft nicht mit dem Internet verbunden ist. Die Anzeige im Header dient der Kontrolle über diesen Status.
Die folgenden Codeausschnitte illustrieren das Konfliktmanagement im Prototypen \it{amilia-qouch}.\\
Konflikte werden in CouchDB gespeichert, damit in der Anwendung entschieden werden kann wie damit umgegangen wird.
Im \autoref{code:conflicts} werden alle Kontakteinträge geladen.
Weil der Parameter in Zeile drei als Option mitgegeben wird, sind Konfliktinformationen für jeden Kontakt verfügbar.
Gibt es unterschiedliche Versionen eines Kontaktes kommt er mit dem Attribut \tt{\_conflicts} beim Client an, eine Liste aus alles korrelierenden Revisionsnummern.
In Zeile sieben werden wird der erste konfliktbehaftete Kontakt der vom Server kommt ermittelt. Wenn es einen Konflikt gibt, wird die Funktion \tt{getConflictRevisions} in Zeile zehn aufgerufen.
%
\begin{center}
  \lstinputlisting[language=REACT,
    firstline=1,lastline=12,
    numbers=left,xleftmargin=20pt,framexleftmargin=15pt,
    caption={Das Laden von konfliktbehafteten Kontakten},
    label=code:conflicts]{code/conflicts.js}
\end{center}
%
\autoref{code:conflicts2} zeigt die Umsetzung der \tt{getConflictRevisions()}.
Hier wird die konkurrierende Version des Kontakts ermittelt. Außerdem wird die Herkunft jeder Version festgestellt und das Öffnen eines Konfliktdialogs eingeleitet.\\
In Zeile zwei werden die Varieblen \tt{contactMe} und \tt{contactYou} initialisiert. Die erste repräsentiert die lokale Version, \tt{contactYou} steht für die Version die aus der CouchDB kommt.
In Zeile vier wird überprüft ob die übergebene Version des Kontakts mit dem zuletzt bearbeiteten übereinstimmt. Entsprechend werden die Variablen \tt{contactMe} und \tt{contactYou} befüllt.
Die übergebene Version ist die von CouchDB festgelegte gewinnende Revision.
Die andere, konkurrierende Version wird in Zeile acht, bzw. in Zeile 13 durch die Übergabe der Revisionsnummer im Parameter geladen.
Die ID ist bei beiden Versionen identisch.\\
Dann wird das Öffnen des Konfliktdialogs durch das Aktualisieren den \gls{App}status initialisiert.
Die beiden Kontaktversionen werden ebenfalls in den State geladen, um im Dialog korrekt angezeigt zu werden.
%
\begin{center}
  \lstinputlisting[language=REACT,
    firstline=14,lastline=38,
    numbers=left,xleftmargin=20pt,framexleftmargin=15pt,
    caption={Das Ermitteln von konkurrierenden Kontaktversionen},
    label=code:conflicts2]{code/conflicts.js}
\end{center}
%
Der Konfliktdialog ist in \autoref{fig:modal} dargestellt.
Im Titel steht der Name der lokalen Version, rot hervorgehoben.
Darunter befinden sich zwei große Knöpfe in unterschiedlichen Farben.
Der erste zeigt die Version des Kontakts, die vom Server kommt, die auf einem anderen Gerät bearbeitet wurde.
Der zweite, untere Knopf beinhaltet alle relevanten Informationen über die lokale Version des konfliktbehafteten Kontakts.
So ist zu erkennen welche der beiden Versionen die eigene ist und worin sie sich von der anderen unterscheidet.
%
\begin{figure}[H]
  \centering
  \includegraphics[width=\textwidth]{impl/Modal}
  \grayRule
  \caption{Konfliktdialog des Prototypen amilia-qouch}
  \label{fig:modal}
\end{figure}
%
Durch das Klicken einer dieser Knöpfe wird entschieden welche der beiden Versionen behalten und welche eliminiert wird.
Wird der Knopf mit der lokalen Version betätigt, wird die in \autoref{code:conflicts3} gelistete Funktion \tt{removeRev} mit der anderen Version im Parameter aufgerufen.
%
\begin{center}
  \lstinputlisting[language=REACT,
    firstline=40,lastline=43,
    numbers=left,xleftmargin=20pt,framexleftmargin=15pt,
    caption={Das Eliminieren der verlierenden Version},
    label=code:conflicts3]{code/conflicts.js}
\end{center}
%
Dort, in Zeile drei, wird die verlierende Revision gelöscht.\\\\
%
%
Der Konfliktdialog konnte für den Protoypen \it{amilia-rdx} nicht umgesetzt werden da Redux offline nicht die Möglichkeit bietet Konflikte zu erkennen, gescheige denn zu speichern.
%
% Installationsanleitung
%
\section{WIP Installationsanleitung}
Beide entwickelten Prototypen sind auf der beigelegten CD zu finden.
Für die Ausführung beider Prototypen ist die Installation von Node.js eine Voraussetzung.
Ein Node.js--Installationsprogramm steht unter \url{https://nodejs.org/en/download/} zum Download bereit.
Um die Prototypen zu starten müssen folgende Schritte ausgeführt werden.
%
%
\sub{\label{chap:install:qouch}amilia-qouch}
Für die Ausführung dieser Anwendung muss CouchDB auf dem Anwendungsgerät installiert sein.
Zur Installation von CouchDB kann die Anleitung auf \url{http://docs.couchdb.org/en/2.1.2/install/index.html} verwendet werden.
Es muss außerdem ein User ''admin'' mit dem Passwort ''admin'' existieren, um auf die Datenbank zugreifen zu können.\\
Dann müssen die \gls{CORS} Optionen aktiviert werden.
Das ist eine Webtechnologie, die es Webanwendungen erlaubt, Ressourcen von einer anderen, entfernten Domain zu benutzen.
% PouchDb läuft auf einer anderen Domain als CouchDB
Es gibt bereits ein Skript, welches dies durchführt. Dazu müssen folgende Befehle ausgeführt werden.
%
\lstset{language=sh, caption={},belowcaptionskip=0.3\baselineskip,xleftmargin=15pt,framexleftmargin=5pt}
\begin{lstlisting}
npm install -g add-cors-to-couchdb
add-cors-to-couchdb
\end{lstlisting}
%
Waren diese Schritte erfolgreich, kann die Anwendung gestartet werden.\\\\
1. Zuerst muss der Ordner \it{amilia-qouch} auf den Computer kopiert werden.\vspace{0.3cm}\\
2. Dann werden dort mit dem folgenden Befehl alle Abhängigkeiten installiert.
\begin{lstlisting}
npm install
\end{lstlisting}\vspace{0.3cm}
%
3. Durch den Aufruf
\begin{lstlisting}
npm start
\end{lstlisting}
wird die Anwendung gestartet und läuft auf \url{http://localhost:3000/}.
%
%
%
\sub{amilia-rdx}
1. Auch hier muss der Ordner \it{amilia-qouch} auf den Computer kopiert werden.\vspace{0.3cm}\\
2. Schritt zwei ist identisch mit dem in der Anleitung auf \hyperref[chap:install:qouch]{\it{amilia-qouch}}\vspace{0.3cm}\\
3. Durch die Aufrufe
\begin{lstlisting}
npm run server
npm start
\end{lstlisting}\vspace{0.3cm}
in separaten Terminalfenstern wird zuerst der Server und dann die Anwendung gestartet.
Die Anwendung läuft nun auf \url{http://localhost:3000/}.

%
% Testfälle
%
\section{\label{chap:impl:test}Testfälle}
Folgende Testfälle zur Offlinefunktionalität an einem Gerät werden während der Entwicklung stetig durchgeführt.
Das erfolgreiche Bestehen dieser Tests ist eine notwendige Qualitätseigenschaft der zu entwickelnden Prototypen.
\begin{description}[leftmargin=0.7cm,style=nextline]
\item[Netzwerkstatus:] 
Die Anwendung muss zu jeder Zeit den korrekten Netzwerkstatus anzeigen.\\
\item[Kontakte lesen:] 
Die Anwendung muss bei jedem Start die Kontakte aus dem lokalen Speicher oder aus der Datenbank bzw. der \gls{JSON} Datei laden.\\
\item[Kontakt anlegen:] 
Die Anwendung muss zu jedem Zeitpunkt in der Lage sein, einen Kontakt mit jedem seiner Attribute anzulegen.
Dazu muss der Kontakt immer lokal gespeichert werden und sobald eine Internetverbindung besteht, persistiert werden.
Das Anlegen eines Kontakts im Offlinestatus ist für die Konfliktherbeiführung erforderlich.\\
\item[Kontakt bearbeiten:] 
Die Anwendung muss zu jedem Zeitpunkt in der Lage sein, einen Kontakt mit jedem seiner Attribute zu bearbeiten.
Ist keine Internetverbindung vorhanden, müssen die Änderungen lokal übernommen und später, sobald sich der Netzwerkstatus ändert, synchronisiert werden.
Das Bearbeiten eines Kontakts im Offlinestatus ist für die Konfliktforcierung erforderlich.\\
\item[Kontakt löschen:] 
Die Anwendung muss zu jedem Zeitpunkt in der Lage sein, einen Kontakt löschen zu können.
Das Löschen eines Kontakts im Offlinestatus ist für die Konfliktherbeiführung erforderlich.
\end{description}