\chapter{\label{chap:implementierung}Implementierung der Prototypen}
In diesem Kaptitel wird nach dem in Kapitel \ref{chap:konzeption} präsentiertem Lösungsweg die detaillierte Beschreibung der technischen Realisierung der Prototypen vorgestellt.\\
Nach der Beschreibung der Konfigurationsdateien wird auf die Umsetzung der Szenarien eingegangen. Im Zuge dessen werden die implementierten Algorithmen vorgestellt, wobei sich der erste mit dem Auffinden der jeweilig nächsten relevanten Ampel befasst und der zweite die jeweils empfohlene Geschwindigkeit berechnet.
\section{package.json?}
Vergleichen? build Vorgang? Ich glaube das würde ich rauslassen (habe mich hier an anderen Arbeiten orientiert)
\section{Hauptmodul Contacts}
wird beim Start ausgeführt\\
Stellt zenrale Funktionen bereit... Erklärung\\
\section{Umsetzung der Szenarien? Anforderungen?}
Umsetzung der Funktionalität? \\
Wie wurde CRUD implementiert?\\
Wie kann ein Konflikt erzeugt werden?\\
(manuell -- siehe \ref{sub:detect})
%
% Installationsanleitung
%
\section{Installationsanleitung}
\todo{build und so}
Beide entwickelten Prototypen sind als öffentliche Repositories auf GitHub\footnote{Software--Entwicklungs--Plattform \url{https://github.com/}} zu finden. 
Um sie zu installieren müssen folgende Schritte ausgeführt werden.
\sub{amilia-qouch}
1. Zuerst muss das Repository kopiert werden:
\lstset{language=sh, caption={},belowcaptionskip=0.3\baselineskip}
\begin{lstlisting}
git clone git@github.com:hulkoba/amilia-qouch.git
# oder
git clone https://github.com/hulkoba/amilia-qouch.git
\end{lstlisting}
2. Dann muss man in das Verzeichnis navigieren und alle Abhängigkeiten installieren.
\begin{lstlisting}
cd amilia-qouch
npm install
\end{lstlisting}
3. Mittels
\begin{lstlisting}
npm start
\end{lstlisting}
wird die Anwendung gestartet.
%
\sub{amilia-rdx}
1. Auch hier muss das Repository zuerst kopiert werden:
\lstset{language=sh, caption={},belowcaptionskip=0.3\baselineskip}
\begin{lstlisting}
git clone git@github.com:hulkoba/amilia-rdx.git
# oder
git clone https://github.com/hulkoba/amilia-rdx.git
\end{lstlisting}
2. Schritt zwei ist identisch mit dem in der \tt{amilia-qouch} Anleitung\\
3. Mittels
\begin{lstlisting}
npm run server
npm start
\end{lstlisting}
wird zuerst der Server, dann die Anwendung gestartet.
%
% Testfälle
%
\section{\label{sec:impl:test}Testfälle}
Folgende Testfälle zur Offlinefunktionalität werden während der Entwicklung stetig durchgeführt. Das erfolgreiche Bestehen dieser Tests ist eine notwendige Qualitätseigenschaft der zu entwickelnden Prototypen.
\begin{description}[leftmargin=0.7cm,style=nextline]
\item[Netzwerkstatus:] 
Die Anwendung muss zu jeder Zeit den korrekten Netzwerkstatus anzeigen.\\
\item[Kontakte lesen:] 
Die Anwendung bei jedem Start die Kontakte aus dem lokalen Speicher oder aus der \it{Datenbank} laden.\\
\item[Kontakt anlegen:] 
Die Anwendung muss zu jedem Zeitpunkt in der Lage sein einen Kontakt mit jedem seiner Attribute anzulegen. Dazu muss er immer lokal gespeichert werden und sobald eine Internetverbindung besteht, persistiert werden.
Das Anlegen eines Kontakts im Offlinestatus ist für die Konfliktforcierung erforderlich.\\
\item[Kontakt bearbeiten:] 
Die Anwendung muss zu jedem Zeitpunkt in der Lage sein einen Kontakt mit jedem seiner Attribute zu bearbeiten. Ist keine Internetverbindung vorhanden, müssen die Änderungen lokal übernommen und später, sobald sich der Netzwerkstatus ändert, synchronisiert werden.
Das Bearbeiten eines Kontakts im Offlinestatus ist für die Konfliktforcierung erforderlich.\\
\item[Kontakt löschen:] 
Die Anwendung muss zu jedem Zeitpunkt in der Lage sein einen Kontakt zu löschen.
Das Löschen eines Kontakts im Offlinestatus ist für die Konfliktforcierung erforderlich.
\end{description}