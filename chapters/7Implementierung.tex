\chapter{\label{chap:implementierung}WIP Implementierung der Prototypen}
In diesem Kaptitel wird nach dem in Kapitel \ref{chap:konzeption} präsentiertem Lösungsweg die detaillierte Beschreibung der technischen Realisierung der Prototypen vorgestellt.\\
\todo{Nach der Beschreibung der Hauptkomponente wird auf die Umsetzung der Offlinefunktionalität und des Konfliktmanagements eingegangen. Im Zuge dessen wird ... vorgestellt...}
%
%
%
\section{WIP Die Kernkomponente Contacts}

Kontakte lesen, anlegen, bearbeiten, löschen
Couch ermittelt Delta über die Revisionsnummer und sendet nur die neuen Daten.
CRUD, toggleEdit, getConflictRevisions, chooseRev()\\
Redux: Contacts= ContactsView, ContactsContainer, reducer/actions
%
%
%
\section{WIP Offlinefunktionalität}
Die Prototypen \it{amilia-qouch} und \it{amilia-rdx} sind vollständig offline verwendbar.
Alle in \autoref{chap:offlinefirst} beschriebenen Grundvoraussetzungen werden von beiden Prototypen erfüllt und sämtliche umgesetzten Funktionen sind sowohl mit als auch ohne Internetverbindung durchführbar.
%
%
\sub{Datenspeicherung}
\todo{Cachen der Assets geht erst nach dem Build -> ServiceWorker}\\
Redux Offline speichert, wie in \autoref{sub:reduxpersist} bereits beschrieben, alle im Redux Store verwalteten Daten im LocalStorage. \autoref{fig:local-rdx} zeigt alle gespeicherten, nutzerInnengenerierten Daten im LocalStorage.
%
\begin{figure}[H]
  \centering
  \includegraphics[width=\textwidth]{impl/localRdx}
  \grayRule
  \caption[Gespeicherte Daten im LocalStorage]{Gespeicherte Daten des Prototypen amilia-rdx im LocalStorage,\\Screenhot: Developer Tools im Firefox Browser}
  \label{fig:local-rdx}
\end{figure}
% 
Der Prototyp \it{amilia-qouch} nutzt zur lokalen Datenspeicherung PouchDB. PouchDB speichert die von NutzerInnen generierten Daten in IndexedDB, vgl. \autoref{chap:pouch}. In \autoref{fig:local-qouch} sind die gespeicherten Daten in der IndexedDB zu sehen.
%
\begin{figure}[H]
  \centering
  \includegraphics[width=\textwidth]{impl/localQouch}
  \grayRule
  \caption[Gespeicherte Daten in IndexedDB]{Gespeicherte Daten des Prototypen aus amilia-qouch in IndexedDB,\\Screenhot: Developer Tools im Firefox Browser}
  \label{fig:local-qouch}
\end{figure}
%
%
\sub{Datenbanksynchronisation}
Redux: discard überchreiben weil ROLLBACK nicht immer retried, obwohl das so in der Dokumentation steht.
Wie wurde CRUD implementiert?\\
%
%
%
\section{WIP Konfliktmanagement}
Wie kann ein Konflikt erzeugt werden?\\
(manuell -- siehe \ref{sub:detect})
\begin{center}
  \lstinputlisting[language=REACT,
    firstline=1,lastline=12,
    numbers=left,xleftmargin=20pt,framexleftmargin=15pt,
    caption={Laden von konfliktbehafteten Kontakten},
    label=code:conflicts]{code/conflicts.js}
\end{center}

\begin{center}
  \lstinputlisting[language=REACT,
    firstline=14,lastline=38,
    numbers=left,xleftmargin=20pt,framexleftmargin=15pt,
    caption={Laden von konfliktbehafteten Kontakten},
    label=code:conflicts2]{code/conflicts.js}
\end{center}

\begin{center}
  \lstinputlisting[language=REACT,
    firstline=40,lastline=43,
    numbers=left,xleftmargin=20pt,framexleftmargin=15pt,
    caption={Laden von konfliktbehafteten Kontakten},
    label=code:conflicts3]{code/conflicts.js}
\end{center}

Modal:
\begin{figure}[H]
  \centering
  \includegraphics[width=\textwidth]{impl/Modal}
  \grayRule
  \caption{Konfliktdialog in Aktion}
  \label{fig:modal}
\end{figure}
Modal ist nicht bei Redux offline enthalten, das es nicht die Möglichkeit bietet Konflikte zu erkennen, gescheige denn zu speichern.
%
%
%
%
% Installationsanleitung
%
\section{WIP Installationsanleitung}
Beide entwickelten Prototypen sind auf der beigelegten CD zu finden.
Für die Ausführung beider Prototypen ist die Installation von Node.js eine Voraussetzung.
Ein Node.js--Installationsprogramm steht unter \url{https://nodejs.org/en/download/} zum Download bereit.
Um die Prototypen zu starten müssen folgende Schritte ausgeführt werden.
%
%
\sub{\label{chap:install:qouch}amilia-qouch}
Für die Ausführung dieser Anwendung muss CouchDB auf dem Anwendungsgerät installiert sein.
Zur Installation von CouchDB kann die Anleitung auf \url{http://docs.couchdb.org/en/2.1.2/install/index.html} verwendet werden.
Es muss außerdem ein User ''admin'' mit dem Passwort ''admin'' existieren, um auf die Datenbank zugreifen zu können.\\
Dann müssen die \gls{CORS} Optionen aktiviert werden.
Das ist eine Webtechnologie, die es Webanwendungen erlaubt, Ressourcen von einer anderen, entfernten Domain zu benutzen.
% PouchDb läuft auf einer anderen Domain als CouchDB
Es gibt bereits ein Skript, welches dies durchführt. Dazu müssen folgende Befehle ausgeführt werden.
%
\lstset{language=sh, caption={},belowcaptionskip=0.3\baselineskip,xleftmargin=15pt,framexleftmargin=5pt}
\begin{lstlisting}
npm install -g add-cors-to-couchdb
add-cors-to-couchdb
\end{lstlisting}
%
Waren diese Schritte erfolgreich, kann die Anwendung gestartet werden.\\\\
1. Zuerst muss der Ordner \it{amilia-qouch} auf den Computer kopiert werden.\vspace{0.3cm}\\
2. Dann werden dort mit dem folgenden Befehl alle Abhängigkeiten installiert.
\begin{lstlisting}
npm install
\end{lstlisting}\vspace{0.3cm}
%
3. Durch den Aufruf
\begin{lstlisting}
npm start
\end{lstlisting}
wird die Anwendung gestartet und läuft auf \url{http://localhost:3000/}.
%
%
%
\sub{amilia-rdx}
1. Auch hier muss der Ordner \it{amilia-qouch} auf den Computer kopiert werden.\vspace{0.3cm}\\
2. Schritt zwei ist identisch mit dem in der Anleitung auf \hyperref[chap:install:qouch]{\it{amilia-qouch}}\vspace{0.3cm}\\
3. Durch die Aufrufe
\begin{lstlisting}
npm run server
npm start
\end{lstlisting}\vspace{0.3cm}
in separaten Terminalfenstern wird zuerst der Server und dann die Anwendung gestartet.
Die Anwendung läuft nun auf \url{http://localhost:3000/}.

%
% Testfälle
%
\section{\label{chap:impl:test}Testfälle}
Folgende Testfälle zur Offlinefunktionalität an einem Gerät werden während der Entwicklung stetig durchgeführt.
Das erfolgreiche Bestehen dieser Tests ist eine notwendige Qualitätseigenschaft der zu entwickelnden Prototypen.
\begin{description}[leftmargin=0.7cm,style=nextline]
\item[Netzwerkstatus:] 
Die Anwendung muss zu jeder Zeit den korrekten Netzwerkstatus anzeigen.\\
\item[Kontakte lesen:] 
Die Anwendung muss bei jedem Start die Kontakte aus dem lokalen Speicher oder aus der Datenbank bzw. der \gls{JSON} Datei laden.\\
\item[Kontakt anlegen:] 
Die Anwendung muss zu jedem Zeitpunkt in der Lage sein, einen Kontakt mit jedem seiner Attribute anzulegen.
Dazu muss der Kontakt immer lokal gespeichert werden und sobald eine Internetverbindung besteht, persistiert werden.
Das Anlegen eines Kontakts im Offlinestatus ist für die Konfliktherbeiführung erforderlich.\\
\item[Kontakt bearbeiten:] 
Die Anwendung muss zu jedem Zeitpunkt in der Lage sein, einen Kontakt mit jedem seiner Attribute zu bearbeiten.
Ist keine Internetverbindung vorhanden, müssen die Änderungen lokal übernommen und später, sobald sich der Netzwerkstatus ändert, synchronisiert werden.
Das Bearbeiten eines Kontakts im Offlinestatus ist für die Konfliktforcierung erforderlich.\\
\item[Kontakt löschen:] 
Die Anwendung muss zu jedem Zeitpunkt in der Lage sein, einen Kontakt löschen zu können.
Das Löschen eines Kontakts im Offlinestatus ist für die Konfliktherbeiführung erforderlich.
\end{description}