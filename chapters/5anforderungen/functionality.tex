%
%
\section{Funktionalität}
Ein offlinefähiges, kollaboratives Adressbuch ist eine Anwendung, mit der mehrere Personen, Kontakte verwalten können.
Sie \it{zeigt} eine Liste von Kontakteinträgen, welche jederzeit -- unabhängig von der Internetverbindung -- von den verwendenden Personen gelesen, bearbeitet, erstellt und gelöscht werden können.
Geschieht eine dieser Operationen offline, werden die Daten bei wieder bestehender Internetverbindung synchronisiert. Im einfachen Fall erfolgt die Synchronisation zwischen der Server und Client. Da die Beispielanwendung kollaborativ ist, erfolgt die sie zwischen allen beteiligten Parteien. Synchronisation erfordert in jedem Fall den Umgang mit Konflikten.\\\\
Bein ersten Start der Anwendung müssen, sofern vorhanden, alle Kontakte geladen werden. Sobald sie einmal geladen sind, sollen sie auch offline verfügbar sein.
Damit ein Datensatz, wie zum Beispiel ein Adressbucheintrag, offline erreichbar ist, sollte er wenigstens so lange auf dem Client gespeichert werden, bis er vollständig beim Server angekommen sind.
Im aktuellen Anwendungsfall bedeutet das, es gibt zwei Kopien des Adressbucheintrags. Eine auf dem Anwendungsgerät -- beispielsweise in einer lokalen Datenbank \todo{ref: grundlagen/lokale Speicherung?}, eine auf dem Server.\\
Danach sollen nur die Einträge geladen werden, die nicht auf dem Gerät existieren. Die Daten würden sonst doppelt geladen werden, der Server hätte mehr zu arbeiten was wiederum die Antwortzeit verlängern würde.
Dazu muss ermittelt werden können, welche Daten neu angelegt oder aktualisiert wurden.
Der Server muss also in der Lage sein die Einträge zu sortieren und nur bestimmte Einträge zu versenden und die Anwendung muss wissen, welche Daten sie bereits hat. Wird ein Eintrag angelegt, bearbeitet oder gelöscht, müssen alle Parteien wissen um welchen Kontakt es sich handelt.
Dazu muss jeder Kontakt mittels einer eindeutigen ID identifiziert werden. Wenn es zwei Kontakteinträge mit derselben ID gibt, muss feststellbar sein, welcher Eintrag der aktuellere ist.\\
Gibt es mehr als zwei Einträge müssen diese sortiert werden, sodass ersichtlich wird welcher der aktuellste oder älteste ist, welcher Eintrag vor oder nach welchem kommt. Dazu muss jeder Kontakt versioniert werden.
%
%
\sub{Konfliktmanagement}
Die in Kapitel \ref{chap:szenarien} erarbeiteten Szenarien zeigen, Konflikte können immer auftreten. Werden Konflikte falsch oder gar nicht behandelt, kann es zu Datenverlust führen.
Aus diesem Grund müssen sie als Teil der Anwendung betrachtet statt ignoriert zu werden.
Im einfachen Konfliktfall kann das System entscheiden welches die konfliktfreie Version ist. So kann zum Beispiel der Kontakt `Amilia Pond` von einer Person eine neue Telefonnummer, von einer anderen eine neue Adresse bekommen.
Die Aktualisierungen finden in unterschiedlichen Bereichen statt und stellen kein Problem dar.\\\\
Die oben erarbeiteten \hyperref[sec:konfliktszenarien]{Konfliktszenarien} beschreiben Konflikte die nicht vom System gelöst werden können.
Diese sollen effizient gespeichert werden. Wichtig hierbei ist die Möglichkeit immer zu einem konfliktfreien Status zu gelangen -- unabhängig davon wie viele Konflikte es gibt.\\
Jeder Fehlerfall muss kommuniziert werden. Wenn es konliktbehaftete Daten gibt muss dies mitgeteilt, und angeboten werden die Konflikte zu lösen. Nur so kann sichergestellt werden, dass keine Daten verloren gehen.
    % \subitem Operationen müssen dem Objekt/Eintrag zugeordnet werden
  % \item Delta berechnen [alle Kontakte -- lokal existierende Kontakte]