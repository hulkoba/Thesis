\sub{Developer Perspektive}
Die folgende Tabelle zeigt die Software-Anforderungen an eine offlinefähige Kontaktliste aus der EntwicklerInnenperspektive.
\begin{longtable}[c]{@{}
	>{\columncolor[HTML]{CFFCC2}}l ll@{}}
	\toprule
	\multicolumn{1}{p{0.15\textwidth}}{\cellcolor[HTML]{cffcc2}\textbf{ID}}
	                                                                   & \multicolumn{1}{p{0.85\textwidth}}{\cellcolor[HTML]{cffcc2}\textbf{Anforderung aus Entwicklungsperspektive}} \\
	\hline \noalign{\vskip 0.1cm}
	\endfirsthead
	%
	\endhead
	%
	\multicolumn{1}{l}{\cellcolor[HTML]{cffcc2}\textbf{User-Story 6}}  &                                                                                                              
	\multicolumn{1}{p{0.85\textwidth}}
	{Ich als EntwicklerIn möchte die Daten lokal und auf dem Server speichern, um deren Erreichbarkeit unabhängig vom Internetstatus zu gewährleisten.}\\
	\midrule
	%
	\multicolumn{1}{l}{\cellcolor[HTML]{cffcc2}\textbf{User-Story 7}}  &                                                                                                              
	\multicolumn{1}{p{0.85\textwidth}}
	{Ich als EntwicklerIn möchte ich nur die Adressbucheinträge oder deren Aktualisierungen laden, die sich nicht schon auf dem Endgerät befinden, um Datentraffic und Ladezeiten zu sparen.}\\
	\midrule
	%
	\multicolumn{1}{l}{\cellcolor[HTML]{cffcc2}\textbf{User-Story 8}}  &                                                                                                              
	\multicolumn{1}{p{0.85\textwidth}}
	{Ich als EntwicklerIn möchte ich jeden Eintrag identifizieren, um jedem Adressbucheintrag Operationen zuzuweisen und einzelne Kontakte zu finden.}\\
	\midrule
	%
	\multicolumn{1}{l}{\cellcolor[HTML]{cffcc2}\textbf{User-Story 9}}  &                                                                                                              
	\multicolumn{1}{p{0.85\textwidth}}
	{Ich als EntwicklerIn möchte ich jeden Eintrag versionieren, um zu wissen ob wann ein Eintrag bearbeitet wurde.}\\
	\midrule
	%
	\multicolumn{1}{l}{\cellcolor[HTML]{cffcc2}\textbf{User-Story 10}} &                                                                                                              
	\multicolumn{1}{p{0.85\textwidth}}
	{Ich als EntwicklerIn möchte dass alle von NutzerInnen vorgenommenen Änderungen beim System ankommen und keine Daten verloren gehen.}\\
	\midrule
	%
	\multicolumn{1}{l}{\cellcolor[HTML]{cffcc2}\textbf{User-Story 11}} &                                                                                                              
	\multicolumn{1}{p{0.85\textwidth}}
	{Ich als EntwicklerIn möchte auftretende Konflikte effizient speichern, um mit ihnen umgehen können. \todo{Mit ihnen umgehen heißt: selbstständig oder von User lösen, zum konfliktfreien Zustand gelangen}}\\
	\midrule
	%
	\multicolumn{1}{l}{\cellcolor[HTML]{cffcc2}\textbf{User-Story 12}} &                                                                                                              
	\multicolumn{1}{p{0.85\textwidth}}
	{Ich als EntwicklerIn möchte eine Technologie verwenden die leicht zu verstehen und implemenieren ist, um den Arbeitsaufwand gering zu halten.}\\
	\midrule
	%
	\multicolumn{1}{l}{\cellcolor[HTML]{cffcc2}\textbf{User-Story 13}} &                                                                                                              
	\multicolumn{1}{p{0.85\textwidth}}
	{Ich als EntwicklerIn möchte sauberen und verständlichen Code schreiben, um die Les-- und Wartbarkeit zu erhöhen.}\\
	% end
	\bottomrule \cellcolor[HTML]{FFFFFF}
	\vspace{0.1cm}\\
	\noalign{\hspace{0.0525\textwidth}\grayRule}
	\caption{Anforderungen aus Entwicklungsperspektive}
	\label{tab:dev}\\
\end{longtable}
