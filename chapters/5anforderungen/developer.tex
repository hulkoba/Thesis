Die folgende Tabelle zeigt die Software-Anforderungen an eine offlinefähige Kontaktliste aus der Perspektive der EntwicklerInnen.
\begin{longtable}[c]{@{}
	>{\columncolor[HTML]{CFFCC2}}l ll@{}}
	\toprule
	\multicolumn{1}{p{0.05\textwidth}}{\cellcolor[HTML]{cffcc2}\textbf{ID}}
	&
	\multicolumn{1}{p{0.9\textwidth}}{\cellcolor[HTML]{cffcc2}\textbf{Anforderung aus Entwicklungsperspektive}} \\
	\hline \noalign{\vskip 0.1cm}
	\endfirsthead
	%
	\endhead
	%
	\multicolumn{1}{p{0.05\textwidth}}{\cellcolor[HTML]{cffcc2}\textbf{D1}} & 
	\multicolumn{1}{p{0.9\textwidth}}
	{Ich als EntwicklerIn möchte die Daten lokal und auf dem Server speichern, um deren Erreichbarkeit unabhängig vom Internetstatus zu gewährleisten.}\\
	\midrule
	%
	\multicolumn{1}{p{0.05\textwidth}}{\cellcolor[HTML]{cffcc2}\textbf{D2}} & 
	\multicolumn{1}{p{0.9\textwidth}}
	{Ich als EntwicklerIn möchte nur die Adressbucheinträge oder deren Aktualisierungen laden, die sich nicht schon auf dem Endgerät befinden, um Daten--Traffic und Ladezeiten zu sparen.}\\
	\midrule
	%
	\multicolumn{1}{p{0.05\textwidth}}{\cellcolor[HTML]{cffcc2}\textbf{D3}} &
	\multicolumn{1}{p{0.9\textwidth}}
	{Ich als EntwicklerIn möchte jeden Eintrag eindeutig identifizieren, um jedem Adressbucheintrag Operationen zuzuweisen und einzelne Kontakte zu finden.}\\
	\midrule
	%
	\multicolumn{1}{p{0.05\textwidth}}{\cellcolor[HTML]{cffcc2}\textbf{D4}} &
	\multicolumn{1}{p{0.9\textwidth}}
	{Ich als EntwicklerIn möchte, dass jeder Eintrag versioniert ist, um zu wissen ob und wann ein Eintrag bearbeitet wurde.}\\
	\midrule
	%
	\multicolumn{1}{p{0.05\textwidth}}{\cellcolor[HTML]{cffcc2}\textbf{D5}} & 
	\multicolumn{1}{p{0.9\textwidth}}
	{Ich als EntwicklerIn möchte, dass alle von NutzerInnen vorgenommenen Änderungen beim System ankommen und keine Daten verloren gehen.}\\
	\midrule
	%
	\multicolumn{1}{p{0.05\textwidth}}{\cellcolor[HTML]{cffcc2}\textbf{D6}} &
	\multicolumn{1}{p{0.9\textwidth}}
	{Ich als EntwicklerIn möchte auftretende Konflikte speichern, um mit ihnen umgehen können.
	% \todo{Mit ihnen umgehen heißt: selbstständig oder von User lösen, zum konfliktfreien Zustand gelangen}
	}\\
	\midrule
	%
	\multicolumn{1}{p{0.05\textwidth}}{\cellcolor[HTML]{cffcc2}\textbf{D7}} &
	\multicolumn{1}{p{0.9\textwidth}}
	{Ich als EntwicklerIn möchte eine Technologie verwenden, die leicht zu verstehen und implemenieren ist, um den Arbeitsaufwand gering zu halten.}\\
	\midrule
	%
	\multicolumn{1}{p{0.05\textwidth}}{\cellcolor[HTML]{cffcc2}\textbf{D8}} &
	\multicolumn{1}{p{0.9\textwidth}}
	{Ich als EntwicklerIn möchte sauberen und verständlichen Code schreiben, um die Les-- und Wartbarkeit zu erhöhen.}\\
	% end
	\bottomrule \cellcolor[HTML]{FFFFFF}
	\vspace{0.1cm}\\
	\noalign{\hspace{0.0525\textwidth}\grayRule}
	\caption{Anforderungen aus Entwicklungsperspektive}
	\label{tab:dev}\\
\end{longtable}