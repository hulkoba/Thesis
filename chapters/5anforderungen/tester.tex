Die folgende Tabelle zeigt die Software-Anforderungen an eine offlinefähige Kontaktliste aus Perspektive der TesterInnen.
\begin{longtable}[c]{@{}
	>{\columncolor[HTML]{CFFCC2}}l ll@{}}
	\toprule
	\multicolumn{1}{p{0.05\textwidth}}{\cellcolor[HTML]{cffcc2}\textbf{ID}}
	&
	\multicolumn{1}{p{0.9\textwidth}}{\cellcolor[HTML]{cffcc2}\textbf{Anforderung aus TesterInnenperspektive}} \\ \hline \noalign{\vskip 0.1cm}
	\endfirsthead
	%
	\endhead
	%
	\multicolumn{1}{p{0.05\textwidth}}{\cellcolor[HTML]{cffcc2}\textbf{T1}} &
	\multicolumn{1}{p{0.9\textwidth}}
	{Ich als TesterIn möchte sicherstellen, dass der Netzwerkstatus der Anwendung änderbar ist, um zwischen offline und online zu wechseln.}\\
  \midrule
	%
	\multicolumn{1}{p{0.05\textwidth}}{\cellcolor[HTML]{cffcc2}\textbf{T2}} &
	\multicolumn{1}{p{0.9\textwidth}}
	{Ich als TesterIn möchte wissen, ob die Anwendung mit dem Internet verbunden ist oder nicht.}\\
	\midrule
	%
	\multicolumn{1}{p{0.05\textwidth}}{\cellcolor[HTML]{cffcc2}\textbf{T3}} &
	\multicolumn{1}{p{0.9\textwidth}}
	{Ich als TesterIn möchte die Anwendung auf mindestens zwei Geräten verwenden, um Kontakte gleichzeitig zu bearbeiten zu können.}\\
	\midrule
	%
	\multicolumn{1}{p{0.05\textwidth}}{\cellcolor[HTML]{cffcc2}\textbf{T4}} &
	\multicolumn{1}{p{0.9\textwidth}}
	{Ich als TesterIn möchte Konflikte forcieren, um das Verhalten der Anwendung zu evaluieren.}\\
	\midrule
	%
	\multicolumn{1}{p{0.05\textwidth}}{\cellcolor[HTML]{cffcc2}\textbf{T5}} &
	\multicolumn{1}{p{0.9\textwidth}}
	{Ich als TesterIn möchte einen Eintrag editieren \it{können} wenn 1. beide Client und Server online sind, 2. entweder Client oder Server offline ist oder 3. beide Parteien offine sind.}\\
	\midrule
	%
	\multicolumn{1}{p{0.05\textwidth}}{\cellcolor[HTML]{cffcc2}\textbf{T6}} &
	\multicolumn{1}{p{0.9\textwidth}}
	{Ich als TesterIn möchte die Testfälle detailliert dokumentieren, um sie auswerten zu kpönnen.}\\
	% end
	\bottomrule \cellcolor[HTML]{FFFFFF}
	\vspace{0.1cm}\\
	\noalign{\hspace{0.0525\textwidth}\grayRule}
	\caption{Anforderungen aus TesterInnenperspektive}
	\label{tab:test}\\
\end{longtable}
