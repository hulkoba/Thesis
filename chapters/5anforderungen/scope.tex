Das Ziel dieser Arbeit ist die Untersuchung des Konfliktmanagements offlinefähiger Technologien.
Dazu soll die beispielhafte Anwendung entwickelt werden, welche an dem Beispiel eines kollaborativen Adressbuchs die Offlinekompatibilität mit dem Schwerpunkt auf das Konfliktmanagement der verwendeten Technologien illustriert.
Ein offlinefähiges, kollaboratives Adressbuch ist eine Anwendung mit der mehrere Personen, Kontakte verwalten können.
Dass diese Anwendung auch ohne Internetzugang funktioniert, ist obligatorisch.
Das Adressbuch beinhaltet eine Liste von Kontakteinträgen, welche jederzeit -- unabhängig von der Internetverbindung -- von den verwendenden Personen gelesen, bearbeitet, erstellt und gelöscht werden können.
Die Offlinefunktionalität soll durch die verwendeten Technologien gegeben sein und wird nicht ausgiebig getestet.
Geschieht eine dieser Operationen offline, werden die Daten bei wieder bestehender Internetverbindung synchronisiert. Im einfachen Fall erfolgt die Synchronisation zwischen der Server und Client.
Da die Beispielanwendung kollaborativ ist, erfolgt die sie zwischen allen beteiligten Parteien.
% Synchronisation erfordert in jedem Fall den Umgang mit Konflikten.
Der Schwerpunkt liegt auf den Konflikten die entstehen können, wenn die Internetverbindung abbricht.
Dabei sind Aspekte aus unterschiedlichen Rollen zu betrachten. So werden die Rollen der AnwenderInnen, der EntwicklerInnen und die der TesterInnen berücksichtigt.\\\\
%
%
Bein ersten Start der Anwendung müssen, sofern vorhanden, alle Kontakte geladen werden. Sobald sie einmal geladen sind, sollen sie auch offline verfügbar sein.
Damit ein Datensatz, wie zum Beispiel ein Adressbucheintrag, offline erreichbar ist, sollte er wenigstens so lange auf dem Client gespeichert werden, bis er vollständig beim Server angekommen sind.
Im aktuellen Anwendungsfall bedeutet das, es gibt zwei Kopien des Adressbucheintrags. Eine auf dem Anwendungsgerät, eine auf dem Server.\\
Danach sollen nur die Einträge geladen werden, die nicht auf dem Gerät existieren. Die Daten würden sonst doppelt geladen werden, der Server hätte mehr zu arbeiten was wiederum die Antwortzeit verlängern würde.
Dazu muss ermittelt werden können, welche Daten neu angelegt oder aktualisiert wurden.
Der Server muss also in der Lage sein die Einträge zu sortieren und nur bestimmte Einträge zu versenden und die Anwendung muss wissen, welche Daten sie bereits hat. Wird ein Eintrag angelegt, bearbeitet oder gelöscht, müssen alle Parteien wissen um welchen Kontakt es sich handelt.
Dazu muss jeder Kontakt mittels einer eindeutigen ID identifiziert werden. Wenn es zwei Kontakteinträge mit derselben ID gibt, muss feststellbar sein, welcher Eintrag der aktuellere ist.\\
Gibt es mehr als zwei Einträge müssen diese sortiert werden, sodass ersichtlich wird welcher der aktuellste oder älteste ist, welcher Eintrag vor oder nach welchem kommt. Dazu muss jeder Kontakt versioniert werden.\\\\
Um eine Aussage darüber zu treffen, welches der untersuchten Technologien besser für die Entwicklung einer offlinefähigen Anwendung geeignet ist und warum, ist eine Testumgebung mitzuentwickeln. Hier soll die Möglichkeit geschaffen werden, Konflikte durch gleichzeitiges Bearbeiten eines Kontakts im Offlinestatus herbeizuführen und die Testfälle auswerten zu können.\\
Es wird ermittelt welche Strategie zur Konfliktlösung von der Technologie verwendet wird, ob die Funktionalität der Anwendung auch bei auftretenden Konflikten gewährleistet ist und ob dabei Daten verloren gehen.
Dabei wird auch der Aufwand betrachtet, der aufgebracht werden muss die Technologie zu verwenden und wie leicht der geschriebene Quellcode zu verstehen ist.\\
% was Dein Projekt umfassen soll und was nicht. Im Prinzip eine konkretere Version Deiner Zielstellung.
% Hier legst Du auch fest,  was Du untersuchen willst und was nicht.
\todo{Dinge, die von vornherein ausgeschlossen werden: \\
Nur auf 2 Geräten testen\\
perfektes UI\\
Kein Merge von Konfliktversionen. Nur entweder, oder.\\
Implementierungen von Push notifications bei Redux}\\\\
Die in \autoref{chap:konfliktszenarien} erarbeiteten Szenarien zeigen, Konflikte können immer auftreten. Werden Konflikte falsch oder gar nicht behandelt, kann es zu Datenverlust führen.
Aus diesem Grund müssen sie als Teil der Anwendung betrachtet statt ignoriert zu werden.
Im einfachen Konfliktfall kann das System entscheiden welches die konfliktfreie Version ist.
So kann zum Beispiel der Kontakt `Amilia Pond` von einer Person eine neue Telefonnummer, von einer anderen eine neue Adresse bekommen.
Die Aktualisierungen finden in unterschiedlichen Bereichen statt und sind deshalb unproblematisch zuzuordnen.\\\\
Die oben erarbeiteten \hyperref[chap:konfliktszenarien]{Konfliktszenarien} beschreiben die Konflikte die nicht vom System gelöst werden können.
Diese sollen effizient gespeichert werden.
% \it{Wichtig hierbei ist die Möglichkeit immer zu einem konfliktfreien Status zu gelangen -- unabhängig davon wie viele Konflikte es gibt.}\\
Jeder Fehlerfall muss kommuniziert werden. Wenn es konliktbehaftete Daten gibt muss dies mitgeteilt und angeboten werden die Konflikte zu lösen. Nur so kann sichergestellt werden, dass keine Daten verloren gehen.