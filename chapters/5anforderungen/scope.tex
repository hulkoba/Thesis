\section{Umfang}
Das Ziel dieser Arbeit ist die Untersuchung des Konfliktmanagements offlinefähiger Technologien. Dazu soll die beispielhafte Anwendung eines kollaborativen Adressbuchs betrachtet werden. Dass diese Anwendung auch ohne Internetzugang funktioniert, ist obligatorisch.
Der Schwerpunkt liegt auf den Konflikten die entstehen können, wenn die Internetverbindung abbricht. Dabei sind Aspekte aus unterschiedlichen Rollen zu betrachten. So werden die Rollen der AnwenderInnen, der EntwicklerInnen und die der TesterInnen berücksichtigt.\\\\
Es soll eine Applikation entwickelt werden, welche an dem Beispiel eines kollaborativen Adressbuchs die Offlinekompatibilität mit dem Schwerpunkt auf das Konfliktmanagement der verwendeten Technologien illustriert.  Die Anwendung umfasst eine Liste von Kontakten, welche einzeln bearbeitet werden können. Die Offlinefunktionalität soll durch die verwendeten Technologien gegeben sein.\\
Um eine Aussage darüber zu treffen, welches der untersuchten Technologien besser für die Entwicklung einer offlinefähigen Anwendung geeignet ist und warum, ist eine Testumgebung mit zu entwickeln. Hier soll die Möglichkeit geschaffen werden, Konflikte durch gleichzeitiges Bearbeiten eines Kontakts im Offlinestatus zu herbeiführen und die Testfälle auswerten zu können.\\
Es wird ermittelt welche Strategie zur Konfliktlösung von der Technologie verwendet wird, ob die Funktionalität der Anwendung auch bei auftretenden Konflikten gewährleistet ist und ob dabei Daten verloren gehen.
Dabei wird auch der Aufwand betrachtet, der aufgebracht werden muss die Technologie zu verwenden und wie leicht der geschriebene Quellcode zu verstehen ist.\\
% was Dein Projekt umfassen soll und was nicht. Im Prinzip eine konkretere Version Deiner Zielstellung.
% Hier legst Du auch fest,  was Du untersuchen willst und was nicht.
\todo{Dinge, die von vornherein ausgeschlossen werden} und \todo{Bezug zu Szenarien und Anwendungsfällen?}